%! TEX root = lecture0_note.tex
\title{Lambda Calculus and Types, 0}
\subtitle{Preliminary}
\begin{document}
\begin{frame}
\maketitle
\end{frame}

The aim of this course is to briefly grasp some idea of lambda
calculus which is the formalisation of functional programming, and unfortunately
the computability aspect is omitted due to time constraints. Another aspect we
will not discuss is the semantics of lambda calculus. It is one of the
long-standing subject in theoretical computer science but it is very deep
and requires some knowledge of modern mathematics. 

For further details about lambda calculus, the reader is referred
to~\cite{Barendregt1984,Barendregt1992,Barendregt1984a}. 

\section{Relation}
This section is for self-reading only. Please feel free to ignore this section
if you know \emph{relation (in mathematics)} and relevant notions.

\begin{definition}[Set-theoretic]
  A (set-theoretic) \emph{relation}~$R$ (on a set~$X$) is a subset~$R
  \subseteq X \times X$, and the expression ``$x_1 \mathbin{R} x_2$'' indicates
  $(x_1, x_2) \in R$.  Whenever there is no danger of confusion, we may suppress
  the domain, i.e.\ $X$, of a relation. 
\end{definition}

The above definition is conventional and standard in mathematics, but not as
intuitive as it could be. For programmer, the following is perhaps more
straightforward:
\begin{definition}[Functional]
  A \emph{relation}~$R$ is a Boolean-valued function 
  \[
    R\colon X \times X \to \{T, F\},
  \]
  and the expression ``$x_1 \mathbin{R} x_2$'' indicates $R(x_1, x_2) = T$.
\end{definition}

As $R$ being a function, it tells whether the given arguments are related in the
way ``expected''. For example, $R$ can be a comparison operator:
\[
  {\tt less\_than}\colon \mathbb{N} \times \mathbb{N} \to \{T, F\}
\]
which returns $T$ if $x_1 < x_2$, or otherwise $F$.\footnote{%
  The relation ${\tt less\_than}$ can be defined inductively without circular
  reference (how?).}

We also name some special classes of relations. For example,
a relation~$R$ on a set~$X$ is called
\begin{inparaenum}[(a)]
  \item \emph{reflexive} if it is always the case that $x\mathbin{R}x$ for
    any~$x \in X$;
  \item \emph{symmetric} if whenever $x_1 \mathbin{R} x_2$ we have $x_2
    \mathbin{R} x_1$
  \item \emph{transitive} if for any $x_1, x_2, x_3 \in X$ with $x_1 \mathbin{R}
    x_2$ and $x_2 \mathbin{R} x_3$ we always have $x_1 \mathbin{R} x_3$;
  \item an \emph{equivalence relation} if it is reflexive, symmetric, and
    transitive.
\end{inparaenum}

Finally, it is also a useful to construct a certain relation out of another
relation by adding elements. 
\begin{definition}
  Given a relation~$R$, 
  define the following relations inductively:
  \begin{enumerate}
    \item the reflexive closure $R^r$ by\\
      \begin{minipage}[c]{.33\textwidth} 
      \begin{prooftree}
        \AXC{\phantom{X}}
        \UIC{$x \mathbin{R^r} x$}
      \end{prooftree}
      \end{minipage}
      and
      \begin{minipage}{.33\textwidth}
        \begin{prooftree}
          \AXC{$x_1 \mathbin{R} x_2$}
          \RightLabel{;}
          \UIC{$x_1 \mathbin{R^r} x_2$}
        \end{prooftree}
      \end{minipage}
    \item the symmetric closure~$R^s$ by\\
      \begin{minipage}{.33\textwidth}
        \begin{prooftree}
          \AXC{$x_1 \mathbin{R} x_2$}
          \UIC{$x_1 \mathbin{R^s} x_2$}
        \end{prooftree}
      \end{minipage}
        and
        \begin{minipage}{.33\textwidth}
          \begin{prooftree}
            \AXC{$x_1 \mathbin{R} x_2$}
          \RightLabel{;}
            \UIC{$x_2 \mathbin{R^s} x_1$}
          \end{prooftree}
        \end{minipage}
    \item the transitive closure~$R^t$ by\\
      \begin{minipage}{.33\textwidth}
        \begin{prooftree}
          \AXC{$x_1 \mathbin{R} x_2$}
          \UIC{$x_1 \mathbin{R^t} x_2$}
        \end{prooftree}
      \end{minipage}
      and 
      \begin{minipage}{.33\textwidth}
        \begin{prooftree}
          \AXC{$x_1 \mathbin{R} x_2$}
          \AXC{$x_2 \mathbin{R^t} x_3$}
          \RightLabel{;}
          \BIC{$x_1 \mathbin{R^t} x_3$}
        \end{prooftree}
      \end{minipage}
  \end{enumerate}
  where $x, x_1, x_2$ all denote elements of~$X$.
\end{definition}
\begin{proposition}
  The reflexive (resp.\ symmetric and transitive) closure is the least reflexive
  (resp.\ symmetric, and transitive) relation containing~$R$, i.e.\ the closure
  is symmetric and every symmetric relation~$S$ with $R \subseteq S$
  contains~$R^r$.
\end{proposition}
\begin{proof}
  The proof is left as an exercise to the reader.
\end{proof}
\subsubsection*{Exercise}
\begin{enumerate}
  \item Given a relation~$R$, define the least equivalence relation 
    containing~$R$. That is, define the \emph{equivalence closure} of~$R$.
  \item Show that every subset $S \subseteq X$ defines a Boolean-valued 
    function $\chi_S\colon X \to \{T, F\}$ and vice versa, and moreover the
    mapping $S \mapsto \chi_S$ is bijective. 
  \item Explain why the set-theoretic relation is essentially as the same as the
    functional relation. 
\end{enumerate}

\bibliographystyle{plain}
\bibliography{library} 
\end{document}
