%! TEX root = lecture0_slide.tex
\title{Lambda Calculus and Types}
\subtitle{Mathematical Preliminaries and Untyped Arithmetic Expression}
\begin{document}
\begin{frame}
\maketitle
\end{frame}

\section{Introduction}
\begin{frame}{Introduction}
  Why does theory of programming languages matter? 
  \begin{enumerate}
    \item A rigorous approach yields unambiguous and yet neat definition of a
      language. 
     \begin{quote}
       The Definition of Standard ML: 116 pp.\cite{Milner1997} \\
       \quad vs.\\
       The Standard of C++ Programming Language: 1605 pp.
     \end{quote} 
   \item A formally verified compiler is possible: 
     \begin{quote}
       CakeML: A verified implementation of Standard ML. \\
       \quad vs.\\
       CompCert: A formally verified compiler of ...\alert{C99}
     \end{quote} 
     (what is exactly C++, anyway?)
   \item The use of static type prevents mistakes in the early stage of
     developments. 
  \end{enumerate}
\end{frame}

\begin{frame}{Let's start with Calculus \dots}
  How to solve the following equation? 
\[
  \int \sin(x)\,\mathrm{d}x =\, ?
\]
\pause
\[
  \dots = \int \sin(x)\,\mathrm{d}x
\]
and it is mathematically correct! Aren't $x = x$ for any $x$? 
\pause
You are actually asked to \alert{evaluate} an expression to its simplest form
instead. 
\begin{align*}
  \int \sin(x)\,\mathrm{d}x \longrightarrow {\color{red} -\cos(x) + C}
\end{align*}
But, what should $\longrightarrow$ be? 
\end{frame}
\section{Mathematical Preliminaries}

\begin{frame}{Predicate, Relation}
  \begin{definition}
    A \emph{$n$-ary relation} $R$ over sets $X_1, X_2, \ldots,
    X_n$ is a set
    \[
      R \subseteq X_1 \times X_2 \dots \times X_n. 
    \]
  \end{definition}
  \begin{definition}

    \begin{enumerate}
      \item A $1$-ary relation $P \subseteq X$ is a \emph{predicate} on $X$. 
      \item A $2$-ary relation $S \subseteq X \times Y$ is a \emph{binary
        relation} on $X$ and $Y$.
    \end{enumerate}
  \end{definition}
    For readability, $P(x)$ stands for $x \in P$. Also we use `infix' or
    `mixfix' notations for relation.
      \[
        t \longrightarrow u\qquad\text{and}\qquad \Gamma \vdash x : \tau
      \]
      stands for relations $(t, u) \in {\longrightarrow}$ and $ (\Gamma, t, \tau) \in
      (\_\vdash\_:\_)$.
\end{frame}

\begin{frame}{A bit more about relations}

  A binary relation $R \subseteq X \times X$ is \ldots
  \begin{enumerate}
    \item \alert{reflexive} if $x\mathbin{R}x$ for any~$x \in X$;
    \item \alert{symmetric} if $x_1 \mathbin{R} x_2$ implies $x_2 \mathbin{R} x_1$
    \item \alert{transitive} if $x_1 \mathbin{R}
      x_2$ and $x_2 \mathbin{R} x_3$ implies $x_1 \mathbin{R} x_3$;
    \item an \alert{equivalence relation} if it satisfies all of above conditions.
  \end{enumerate}
  \mode<presentation>{\vfill}

  A binary relation $f \subseteq X \times Y$ is \ldots
  \begin{enumerate}
    \item \alert{functional} or a \alert{partial function} if 
      \[
        x \mathrel{f} y \quad\text{and}\quad
        x \mathrel{f} y' \quad\text{implies}\quad y = y'
      \]
    \item a \alert{(total) function} if it is functional and
      \[
        \forall x \in X.\, \exists y \in Y.\, x \mathrel{f} y
      \]
  \end{enumerate}
\end{frame}

\section{Induction and Recursion}
\begin{frame}{Formal language}
  \begin{definition}[Arithmetic expressions, by grammar]
    The set of arithmetic expressions is defined by 
    \[
      t \defeq \mathtt{0} \mid \mathtt{succ}\,t \mid \mathtt{add}\,t\,t
    \]
    
  \end{definition}
  $\mathtt{t}$ is a \emph{metavariable} to be replaced by an 
  expression. E.g., $\mathtt{add}\,\mathtt{0}\,(\mathtt{succ}\, \mathtt{0})$
  is an expression but $\mathtt{succ}$ alone is not. 
  \begin{definition}[Arithetic expressions, inductively]
    The set $\mathcal{T}$ for arithmetic expression is the
    \alert{least} set satisfying
    \begin{enumerate}
      \item $\mathtt{0} \in \mathcal{T}$,
      \item $\mathtt{succ}\,t \in \mathcal{T}$ if $t \in \mathcal{T}$, 
      \item $\mathtt{add}\,t\,u \in \mathcal{T}$ if $t, u \in \mathcal{T}$.
    \end{enumerate}
  \end{definition}
%  {\small The existence of such a set does not concern us here.}
\end{frame}

\begin{frame}{Judgement and Inference Rules}
  A \alert{judgement} is just a predicate and a \alert{rule of inference} is an
  implication in a specific form, possibly with a name.

  \mode<presentation>{\vfill}

  \begin{description}
    \item[Judgement] $0 \in \mathcal{T}$, $t$ is of type $\tau$, \dots
    \item[Inference rules]   \end{description}
      \begin{multicols}{3}
    \begin{prooftree}
      \AXC{\phantom{X}}
      \UIC{$\mathtt{0} \in \mathcal{T}$}
    \end{prooftree}
      \begin{prooftree}
        \AXC{$t \in \mathcal{T}$}
        \UIC{$\mathtt{succ}\,t \in \mathcal{T}$}
      \end{prooftree}
      \begin{prooftree}
        \AXC{$t \in \mathcal{T}$}
        \AXC{$u \in \mathcal{T}$}
        \BIC{$\mathtt{add}\,t\,u \in \mathcal{T}$}
      \end{prooftree}
      \end{multicols}
  \mode<presentation>{\vfill}
    The set of arithmetic expressions is specified by the rules.
    This form of definition is widely used and we will use this form too.
  \mode<presentation>{\vfill}
\end{frame}

\begin{frame}{Structural Induction}
  The Induction Principle holds not only for $\mathbb{N}$ but also for any
  structure defined inductively. 
  \begin{theorem}[Induction Principle on arithmetic expressions]
    Given a predicate $P$, if we can prove that
    \begin{multicols}{3}
    \begin{prooftree}
      \AXC{\phantom{X}}
      \UIC{$P(\mathtt{0})$}
    \end{prooftree}
      \begin{prooftree}
        \AXC{$P(t)$}
        \UIC{$P(\mathtt{succ}\,t)$}
      \end{prooftree}
      \begin{prooftree}
        \AXC{$P(t)$}
        \AXC{$P(u)$}
        \BIC{$P(\mathtt{add}\,t\,u)$}
      \end{prooftree} 
    \end{multicols}
    then $P(t)$ for any $t \in \mathcal{T}$. In other words, $\mathcal{T}
    \subseteq P$. 
  \end{theorem}
  Recall that $\mathcal{T}$ is by definition the smallest set containing
  $\mathtt{0}$ and
  closed under $\mathtt{succ}$ and $\mathtt{add}$. 
\end{frame}

\begin{frame}{Structural Recursion}
  \begin{theorem}[Recursion on arithmetic expressions]
    \label{thm:recursion}
    Given functions 
    \begin{enumerate}
      \item $f_{\mathtt{0}} \in S$, 
      \item $f_{\mathtt{succ}}\colon S \to S$, 
      \item $f_{\mathtt{add}}\colon S \times S \to S$, 
    \end{enumerate}
    there exists a unique function $f\colon \mathcal{T} \to S$ such that 
    \begin{enumerate}
      \item $f(\mathtt{0}) = f_{0}$
      \item $f(\mathtt{succ}\,n) = f_{\mathtt{succ}}(f(n))$
      \item $f(\mathtt{add}\,n\,m) = 
        f_{\mathtt{add}}(f(n), f(m))$.
    \end{enumerate}
  \end{theorem}
  \begin{proof}[Proof of Uniqueness]
    By structural induction on $\mathcal{T}$. 
  \end{proof}
  \end{frame}

  \begin{frame}{A glimpse of denotational semantics}
  \begin{example}
    A recursion from our arithmetic expressions to natural numbers can be given
    by 
    \begin{align*}
      \sem{-} & \colon \mathcal{T} \to \mathbb{N} \\
      \sem{\mathtt{0}} & = 0 \\
      \sem{\mathtt{succ}\,t} & = 1 + \sem{t} \\
      \sem{\mathtt{add}\,t\,u} & = \sem{t} + \sem{u}
    \end{align*}
    which stipulates the \emph{denotational semantics} of our arithmetic expressions. 
  \end{example}
  The subject of denotational semantics is left for self-study. For interested
  readers, see \cite{Scott1976,Streicher2006a}
\end{frame}

\section{Operational Semantics}
\begin{frame}{Reduction Relation}
  Instead of giving the \emph{interpretation} of expressions in other
  structures, expressions can be \alert{evaluated} within the same language. 

  Define a binary relation $\longrightarrow$ between arithmetic expressions by
  \begin{prooftree}
    \AXC{$t_1 \longrightarrow t_2$}
    \RightLabel{($\to$-$\mathtt{succ}$)}
    \UIC{$\mathtt{succ}\,t_1\longrightarrow\mathtt{succ}\,t_2$}
  \end{prooftree}
  \begin{prooftree}
    \AXC{$t_1 \longrightarrow t_2$}
    \AXC{$t_1 \neq \mathtt{succ}\,t$}
    \RightLabel{($\to$-$\mathtt{add}$)}
    \BIC{$\mathtt{add}\,t_1\,u \longrightarrow\mathtt{add}\,t_2\,u$}
  \end{prooftree}
  \begin{prooftree}
    \AXC{}
    \RightLabel{($\to$-$\mathtt{add0}$)}
    \UIC{$\mathtt{add}\,0\,u \longrightarrow u$}
  \end{prooftree}
  \begin{prooftree}
    \AXC{$\phantom{X}$}
    \RightLabel{($\to$-$\mathtt{add}\mathsf{succ}$)}
    \UIC{$\mathtt{add}\,(\mathtt{succ}\,t)\,u \longrightarrow
    \mathtt{succ}\,(\mathtt{add}\,t\,u)$}
  \end{prooftree}
\end{frame}

\begin{frame}{Some Reductions}
      $\mathtt{add}\; (\mathtt{succ}\; \mathtt{0})\;
      (\mathtt{succ}\;(\mathtt{succ}\;\mathtt{0}))$
      \vfill
      $\mathtt{succ}\; (\mathtt{add}\;(\mathtt{succ}\;\mathtt{0})\;
      (\mathtt{succ}\;\mathtt{0}))$
\end{frame}

\begin{frame}{Values, Normal forms}
  \begin{definition}
    The set of \alert{values} for arithmetic expression is defined by
      \begin{multicols}{2}
    \begin{prooftree}
      \AXC{\phantom{X}}
      \UIC{$\mathtt{0} \in \mathsf{Val}$}
    \end{prooftree}
      \begin{prooftree}
        \AXC{$t \in \mathsf{Val}$}
        \UIC{$\mathtt{succ}\,t \in \mathsf{Val}$}
      \end{prooftree}
      \end{multicols}
    In this case, a value is simply a numeral.
  \end{definition}
  \begin{definition}
    An expression is in \alert{normal form} if it cannot be
    reduced further, i.e.
    \[
      \neg (\exists t' \in \mathcal{T}.\, t \longrightarrow t')
    \]
    
  \end{definition}
  \begin{theorem}
    Every value is in normal form.
  \end{theorem}
\end{frame}
\begin{frame}{Determinacy}
The reduction relation is deterministic:
\begin{theorem}\label{thm:determinacy}
  Suppose that $t \in \mathcal{T}$ is an expression.
  If $t \longrightarrow u$ and $t \longrightarrow u'$, then 
  $u = u'$. That is, $\longrightarrow$ is functional. 
\end{theorem}
\begin{proof}
  By structural induction on the reduction relation $\longrightarrow$ (not
  $\mathcal{T}$). 
  
\end{proof}
  
\end{frame}

\begin{frame}{Multi-Step Reduction, Transitive Closure}
  $t \longrightarrow u$ prescribes how $t$ reduces to $u$
  in one step, so it is a \alert{one-step reduction}.
  \begin{definition}
    The \emph{transitive and reflexive closure} $R^*$ of a binary relation $R$
    is
    \begin{multicols}{2}
      \begin{prooftree}
        \AXC{$\phantom{t \longrightarrow u}$}
        \UIC{$t \mathrel{R^*} t$}
      \end{prooftree}
      \begin{prooftree}
        \AXC{$t \mathrel{R} u$}
        \AXC{$u \mathrel{R}^* v$}
        \BIC{$t \mathrel{R}^* v$}
      \end{prooftree}
    \end{multicols}
    In particular, $\longrightarrow^*$ is the \alert{multi-step} reduction. 
  \end{definition}
  \begin{theorem}[Uniqueness of normal forms]
    If $t \longrightarrow^* u$ and $t \longrightarrow^* u'$ where $u$ and $u'$
    are in normal form, then $u = u'$. 
  \end{theorem}
  \begin{proof}
    By induction. 
  \end{proof}
\end{frame}

\begin{frame}{Homework}
  \begin{enumerate}
    \item Finish the uniqueness proof of Theorem~\ref{thm:recursion}.
    \item Finish the proof of Theorem~\ref{thm:determinacy}.
    \item Show that the reflexive and transitive closure of any relation $R$ is
      reflexive and transitive. 
    \item Define Boolean expressions as follows:\\
      \begin{multicols}{2}
    \begin{prooftree}
      \AXC{\phantom{X}}
      \UIC{$\mathtt{true} \in \mathcal{T}_\Bool$}
    \end{prooftree}
      \begin{prooftree}
        \AXC{\phantom{X}}
        \UIC{$\mathtt{false} \in\mathcal{T}_\Bool$}
      \end{prooftree}
      \begin{prooftree}
        \AXC{$t_1 \in \mathcal{T}_\Bool$}
        \AXC{$t_2 \in \mathcal{T}_\Bool$}
        \AXC{$t_3 \in \mathcal{T}_\Bool$}
        \TIC{$\mathtt{if}\,t_1\,\mathtt{then}\,t_2\,\mathtt{else}\,t_3 
        \in \mathcal{T}_\Bool$}
      \end{prooftree}
      \end{multicols}
  \end{enumerate}
  where values are $\mathtt{true}$ and $\mathtt{false}$. Please give a
  reduction relation satisfying determinacy and that expressions are in normal
  forms if and only if they are values.
\end{frame}

\begin{frame}{Acknowledgement}
  I am grateful to \CJKtext{游書泓} (Shu-Hung You), \CJKtext{余家安} (Chia-An
  Yu), Yu-Hsi Chiang,
  Hsu Karinsu for their suggestions and corrections.
\end{frame}

\begin{frame}[allowframebreaks]{References}

\bibliographystyle{amsalpha}
\bibliography{library} 

\end{frame}


\end{document}
