%! TEX program = xelatex
%! TEX root = lecture2_slide.tex

\title{Programming Language Theory}
\subtitle{Higher-Order Functions}
\begin{document}

{\usebackgroundtemplate{\includegraphics[width=\paperwidth]{image.png}}
\begin{frame}\maketitle\end{frame}}

\section{Simply Typed $\lambda$-Calculus: Statics}

\begin{frame}{Typing judgement}
  A \alert{typing judgement} is of the form
  \[
    \Gamma \vdash M : \sigma
  \]
  saying the \emph{term $M$ is of type $\sigma$ under the context $\Gamma$}
  where 
  \begin{description}
    \item[context $\Gamma$] free variables $x : \tau$ available in $M$
    \item[term $M$] possibly with free variables in $\Gamma$,
    \item[type $\sigma$] for $M$
  \end{description}

  \[
    x_1:\tau_1, x_2: \tau_2 \vdash x_1 : \tau_1
  \]
  `\emph{Under the context consisting of variables $x_1:\tau_1, x_2:\tau_2$, the term $x_1$ is of type $\tau_1$.}'
\end{frame}

\begin{frame}{Context}
  
\begin{definition}
    A \emph{typing context}~$\Gamma$
    is a sequence
    \[
      \Gamma \equiv x_1 : \sigma_1,\; x_2 : \sigma_2,\; \ldots,\; x_n : \sigma_n
    \]
    of \alert{\emph{distinct variables}} $x_i$ of type $\sigma_i$.
\end{definition}

\begin{definition}
  The membership judgement $\Gamma \ni (x : \sigma)$ is defined inductively as follows.
  
  \begin{multicols}{2}
    \begin{prooftree}
      \AXC{$\vphantom{\Gamma}$}
      \RightLabel{(here)}
      \UIC{$\Gamma, x : \sigma \ni (x : \sigma)$}
    \end{prooftree}
    \begin{prooftree}
      \AXC{$\Gamma \ni (x : \sigma)$}
      \RightLabel{(there)}
      \UIC{$\Gamma, y : \tau \ni (x : \sigma)$}
    \end{prooftree}
  \end{multicols}
\end{definition}
\end{frame}


\begin{frame}{Higher-order function type}

  
\begin{definition}
  Define the judgement $\tau : \type$ by
  \begin{multicols}{2}
    \begin{prooftree}
      \AXC{$\sigma$ is a type variable}
      \RightLabel{(tvar)}
      \UIC{$\sigma : \type$}
    \end{prooftree}
    \begin{prooftree}
      \AXC{$\sigma : \type$}
      \AXC{$\tau   : \type$}
      \RightLabel{(fun)}
      \BIC{$\sigma \to \tau : \type$}
    \end{prooftree}
  \end{multicols}
  where $\sigma\to\tau$ represents a function type from $\sigma$ to $\tau$.

  Also $\sigma_1 \to \tau_1 = \sigma_2 \to \tau_2$ if and only if $\sigma_1 = \sigma_2 \text{ and } \tau_1 = \tau_2$.
\end{definition}
\mode<presentation>{\vfill}
\begin{block}{Convention}
\[
  \sigma_1 \to \sigma_2 \to \dots \sigma_n \quad  \defeq\quad \sigma_1 \to
  (\sigma_2 \to ( \dots \to (\sigma_{n-1} \to \sigma_n)\dots))
\]
\end{block}

\end{frame}

\begin{frame}
  The function type is \alert{higher-order}, because
  \begin{enumerate}
    \item functions can be arguments of another function;
    \item functions can be the result of a computation.
  \end{enumerate}
  \begin{example}
    \begin{description}
      \item[$\;(\sigma_1 \to \sigma_2) \to \tau$] a function type whose argument is of type $\sigma_1 \to \sigma_2$; 
      \item[$\;\sigma_1 \to (\sigma_2 \to \tau)$] a function whose return type is $\sigma_2 \to \tau$. 
    \end{description}
  \end{example}
    
\mode<presentation>{\vfill}
For a term $M$, how to construct a \emph{typing judgement}
\[
  \Gamma \vdash M : \sigma \to \tau
\]
\end{frame}

%\begin{frame}{Hypothetical judgement}
%  A \emph{hypothetical judgement} (or, an inference rule) is of the form
%  \begin{prooftree}
%    \AXC{$J_1$}
%    \AXC{$J_2$}
%    \AXC{$\dots$}
%    \AXC{$J_n$}
%    \RightLabel{(rule)}
%    \QuaternaryInfC{$J$}
%  \end{prooftree}
%  consisting of 
%  \begin{description}
%    \item[hypothesis] a set of judgements $J_i$, for $1 \leq i \leq n$
%    \item[name] the name for the reason why $J_i$'s imply $J$
%    \item[conclusion] a single judgement $J$
%  \end{description}
%  Examples include
%  \begin{enumerate}
%    \item The formation of $\lambda$-terms (var), (app), and (abs).
%    \item The formation of $\alpha$-equivalence $=_\alpha$.
%    \item The formation of simple types $\mathbb{T}$.
%  \end{enumerate}
%\end{frame}

\begin{frame}{Typing rule -- Curry-style typing system}
  A \emph{typing rule} is an inference rule with its conclusion a
  typing judgement.
  
  \begin{prooftree}
    \AXC{$\Gamma \ni (x : \sigma)$}
    \RightLabel{(var)}
    \UIC{$\Gamma \vdash_i x : \sigma$}
  \end{prooftree}
  \begin{prooftree}
    \AXC{$\Gamma, x : \sigma \vdash_i M : \tau$}
    \RightLabel{(abs)}
    \UIC{$\Gamma \vdash_i \lambda x.\; M : \sigma \to \tau$}
  \end{prooftree}
  \begin{prooftree}
    \AXC{$\Gamma \vdash_i M : \sigma \to \tau$}
    \AXC{$\Gamma \vdash_i N : \sigma$}
    \RightLabel{(app)}
    \BIC{$\Gamma \vdash_i M\;N : \tau$}
  \end{prooftree}

\mode<presentation>{\vfill}
It is known as the \alert{implicit typing} system since the typing information is an add-on to the term.
\end{frame}


%\begin{frame}{Derivation}
%  Given a set $\mathcal{R}$ of hypothetical judgements, a \emph{derivation} of
%  \begin{prooftree}
%    \AXC{$J_1$}
%    \AXC{$J_2$}
%    \AXC{$\dots$}
%    \AXC{$J_n$}
%    \QuaternaryInfC{$J$}
%  \end{prooftree}
%  is a tree of instances of rules in $\mathcal{R}$ composed with $J_i$'s as the top-level hypothesises.
%
%  A judgement $J$ is \emph{derivable} (with assumptions $J_i$'s) if there is derivation whose root is $J$ (with assumptions $J_i$'s).
%
%  \begin{example}
%    The fact that
%    \[
%      \lambda x\,y.\, y
%    \]
%    is a $\lambda$-term means the judgement $\lambda x\,y.\, y$ is derivable
%    from the formation rules of $\lambda$-terms without any assumption.
%  \end{example}
%\end{frame}

\begin{frame}{Typing derivation}
  The judgement $\vdash \lambda x.\, x : \sigma \to \sigma$, for all $\sigma
  \in \mathbb{T}$ has a derivation
  \begin{prooftree}
    \AXC{}
    \RightLabel{(var)}
    \UIC{$x : \sigma \vdash_i x : \sigma$}
    \RightLabel{(abs)}
    \UIC{$\vdash_i \lambda x.\; x : (\sigma \to \sigma)$}
  \end{prooftree}

  The judgement $ \vdash \lambda x\,y.\, x : \sigma \to \tau \to \sigma$
  has a derivation
\begin{prooftree}
  \AXC{}
  \RightLabel{(var)}
  \UIC{$x : \sigma, y: \tau \vdash_i x : \sigma$}
  \RightLabel{(abs)}
  \UIC{$x : \sigma \vdash_i \lambda y.\, x : \tau \to \sigma$}
  \RightLabel{(abs)}
  \UIC{$\vdash_i \lambda x\,y.\, x : \sigma \to \tau \to \sigma$}
\end{prooftree}

  Not every $\lambda$-term has a type: 
  \[
    \lambda x.\, x\,x
  \]
  there is no $\tau$ satisfying $\vdash \lambda x.\, x\;x : \tau$.
\end{frame}

\begin{frame}{Syntax-directedness}
  A typing system is \emph{syntax-directed} if it has \emph{exactly} one typing rule
  for each term construct. Therefore, 
  
  \begin{lemma}[Typing inversion]
    Suppose 
    \[
      \Gamma \vdash_i M : \tau
    \]
    is derivable. If 
    \begin{description}
      \item[$M \equiv x$] then $x : \tau$ occurs in $\Gamma$.
      \item[$M \equiv \lambda x.\, M'$] then $\tau = \sigma \to \tau'$ for some $\sigma$ and $\Gamma, x:\sigma \vdash_i M' : \tau'$.
      \item[$M \equiv L\;N$] there is some $\sigma$ such that $\Gamma \vdash_i L : \sigma \to \tau$ and $\Gamma \vdash_i N : \sigma$.
    \end{description}
  \end{lemma}
\end{frame}



\begin{frame}{Explicit typing: Typed terms}
\begin{definition}[Typed terms]
  The formation $M : \term_{\lambda_\to}$ of typed terms is defined by
    \begin{prooftree}
      \AXC{$x \in V$}
      \UIC{$x : \term_{\lambda_\to}$}
    \end{prooftree}
    \begin{prooftree}
      \AXC{$M : \term_{\lambda_\to}$}
      \AXC{$N : \term_{\lambda_\to}$}
      \BIC{$M\, N : \term_{\lambda_\to}$}
    \end{prooftree}
    \begin{prooftree}
      \AXC{$M : \term_{\lambda_\to}$}
      \AXC{$x \in V$}
      \AXC{$\tau : \type$}
      \TIC{$\lambda {\color{red}x:\tau}.\; M : \term_{\lambda_\to}$}
    \end{prooftree}
\end{definition}
\end{frame}

\begin{frame}{Explicit typing: Typing rules}
\begin{definition}[Typing Rules]
  Typing derivations on \emph{typed terms} are defined by 
  \begin{prooftree}
    \AXC{$\Gamma \ni (x : \sigma)$}
    \RightLabel{(var)}
    \UIC{$\Gamma \vdash_e x : \sigma$}
  \end{prooftree}
  \begin{prooftree}
    \AXC{$\Gamma \vdash_e M : \sigma \to \tau$}
    \AXC{$\Gamma \vdash_e N : \sigma$}
    \RightLabel{(app)}
    \BIC{$\Gamma \vdash_e M\;N : \tau$}
  \end{prooftree}
  \begin{prooftree}
    \AXC{$\Gamma, {\color{red}x : \sigma} \vdash_e M : \tau$}
    \RightLabel{(abs)}
    \UIC{$\Gamma \vdash_e \lambda {\color{red}x : \sigma} .\; M : \sigma \to \tau$}
  \end{prooftree}
\end{definition}

\end{frame}
\begin{frame}{Explicit typing: Unicity}
\begin{proposition}
  For every typed term $M$, context~$\Gamma$, and types $\sigma_i$, 
  \[
    \Gamma \vdash_e M : \sigma_1
    \quad\text{and}\quad
    \Gamma \vdash_e M : \sigma_2
    \implies
    \sigma_1 = \sigma_2
  \]
\end{proposition}
\begin{proof}[Proof sketch]
  Use the inversion lemma and the structural induction on $M$.

  E.g., suppose that $M$ is of the form
  \[
    L\;M'
  \]
  By inversion there are $\tau_i$ such that $\Gamma \vdash_e L: \tau_i \to
  \sigma_i$ and $\Gamma \vdash_e M': \tau_i$. By induction hypothesis, $\tau_1 \to
  \sigma_1 = \tau_2 \to \sigma_2$, so $\sigma_1 = \sigma_2$.
\end{proof}
\end{frame}

\begin{frame}{Exercise}
  \begin{enumerate}
    \item Derive the judgement
      \[
        \vdash \lambda f\,g\,x.\, f\,x\, (g\,x) : (\sigma \to \tau \to \rho) \to
        (\sigma\to\tau) \to \sigma\to\rho 
      \]
      for every $\sigma, \tau, \rho \in \mathbb{T}$.

    \item Describe all possible types for Church numeral $\bc_{n}$.

    
  \end{enumerate}

\end{frame}


\begin{frame}{Type erasure}
  An \emph{erasing map} $|-|\colon \term_{\lambda_\to} \to \term_{\lambda}$ is defined by
  \begin{align*}
    |x| & = x \\
    |M\; N| & = |M|\;|N| \\
    |\lambda x:\sigma.\, M| & = \lambda x.\, |M|
  \end{align*}
  \begin{example}
    \begin{enumerate}
      \item $|\lambda (f: \sigma \to \tau)\,(x: \sigma).\, f\;x| = \lambda f\, x.\, f\;x$
      \item $|(\lambda (x: \sigma)\,(y: \tau). y)\;z| = (\lambda x\,y.\, y)\; z$
    \end{enumerate}
  \end{example}

  $|-|$ is an translation from $\term_{\lambda_\to}$ to $\term_{\lambda}$.
  Does $|-|$ respect the behaviour of $\term_{\lambda_\to}$?
\end{frame}
\begin{frame}{From typed terms to untyped and back}
\begin{proposition}
  Let $M$ and $N$ be typed $\lambda$-terms in~$\term_{\lambda_\to}$. Then, 
  \begin{align*}
    \Gamma  \vdash_e M : \sigma & \text{ implies } \Gamma \vdash_i |M| :
    \sigma \\ 
    M \reduce N & \text{ implies } |M| \reduce |N|
  \end{align*}
\end{proposition}

\begin{proposition}
  Let $M$ and $N$ be $\lambda$-terms in~$\term_{\lambda}$. Then, 
  \begin{enumerate}
  \item If $\Gamma \vdash_i M : \sigma$, then there is $M' : \term_{\lambda_\to}$ with 
        $|M'| = M
        \quad\text{and}\quad
        \Gamma \vdash_e M' : \sigma$
      \item If $M \reduce N$ and $M = |M'|$ for some $M' : \term_{\lambda_\to}$,
      then there exists $N'$ with $|N'| = N$ and $M' \reduce N'$.
    \end{enumerate}
\end{proposition}
\end{frame}

\begin{frame}{Type inference}
  Can we answer the following questions
  \begin{description}
    \item[Typability] Given a closed term $M$, is there a type $\sigma$
      such that $\vdash M : \sigma$? 
    \item[Type checking] Given $\Gamma$ and $\sigma$, is $\Gamma \vdash M : \sigma$ derivable?
  \end{description}
  algorithmically?
  
  Typability is reducible to type checking problem of
  \[
    x_0: \tau \vdash \textbf{K}_1\;x_0\;M : \tau
  \]

  \begin{theorem}
    Type checking is \emph{decidable}
    in simply typed $\lambda$-calculus.
  \end{theorem}

\mode<presentation>{\vfill}
Check \emph{bidirectional type inference}.
\subsection*{Exercise}
  
\end{frame}

\section{Programming in Simply Typed $\lambda$-Calculus}

\begin{frame}[allowframebreaks]{Church encodings of natural numbers}
The type of natural numbers is of the form
\[
  \nat_\tau \defeq (\tau \to \tau) \to \tau \to \tau
\]
for every type $\tau \in \T$.
  
  \begin{description}
    \item[Church numerals]
      \begin{align*}
        & \bc_n \defeq \lambda f\,x.\,
        f^n x \\
        \vdash{} & \bc_n : \nat_\tau
      \end{align*}
    \item[Successor]
      \begin{align*}
        & \suc \defeq \lambda n \,f\,x\,.\;f\;(n\;f\;x) \\
        \vdash{} & \suc : \nat_\tau \to \nat_\tau
      \end{align*}
    \item[Addition]
      \begin{align*}
        & \add \defeq \lambda n\,m\,f\,x.\; (m\;f)\;(n\;f\;x) \\
        \vdash{} & \add : \nat_\tau \to \nat_\tau \to \nat_\tau
      \end{align*}
    \item[Muliplication] 
      \begin{align*}
        & \mul \defeq \lambda n\,m\,f\,x.\, (m\;(n\;f))\;x\\
      \vdash{} & \mul : \nat_\tau \to \nat_\tau \to \nat_\tau
      \end{align*}
    \item[Conditional]
      \begin{align*}
        & \ifz \defeq \lambda n\,x\,y.\, n\;(\lambda z.\, x)\;y\\
        \vdash {} & \ifz : ?
      \end{align*}
  \end{description}
The type of~$\ifz$ may not be as obvious as you may expect.
Try to find one as general as possible and justify your guess.

\end{frame}
\begin{frame}{Church encodings of boolean values}
We can also define the type of Boolean values 
for each type variable as
\[
  \bool_\tau \defeq \tau \to \tau \to \tau
\]
\begin{description}
  \item[Boolean values]
      \[
        \true \defeq \lambda x\,y.\,x 
        \quad\text{and}\quad
        \false \defeq \lambda x\,y.\,y 
      \]
  \item[Conditional]
    \begin{align*}
      & \cond \defeq \lambda b\,x\,y.\, b\,x\,y\\
      \vdash {} & \cond : \bool_{\tau} \to \tau \to \tau \to \tau
    \end{align*}
\end{description}
\end{frame}

\begin{frame}{Exercise}
  \begin{enumerate}
    \item Define conjunction $\mathtt{and}$, disjunction $\mathtt{or}$, and
      negation $\mathtt{not}$ in simply typed lambda calculus.

    \item Prove that $\mathtt{and}$, $\mathtt{or}$, and $\mathtt{not}$ are well-typed.
  \end{enumerate}
  
\end{frame}


\section{Properties of Simply Typed $\lambda$-Calculus}


\begin{frame}[c]{Type safety = Preservation + Progress}

\begin{quote}
  ``Well-typed programs cannot `go wrong'.''\\
  \hfill ---(Milner, 1978)
\end{quote}

\begin{description}
  \item[Preservation] If $\Gamma \vdash M : \sigma$ is derivable and $M \onereduce N$, then $\Gamma \vdash N : \sigma$.
  \item[Progress] If $\Gamma \vdash M : \sigma$ is derivable, then either $M$ is in \emph{normal form} or there is $N$ with $M \onereduce N$.
\end{description}

\end{frame}

\begin{frame}[allowframebreaks]{Converse of Preservation}
\begin{example}
  Recall that 
  \begin{enumerate}
    \item $\mathbf{I} = \lambda x.\, x$
    \item $\mathbf{K}_1 = \lambda x\,y.\, x$
    \item $\Omega = (\lambda x.\, x\,x)\,(\lambda x.\, x\,x)$
  \end{enumerate}
  and $\mathbf{K}_1\,\mathbf{I}\,\Omega \reduce \mathbf{I}$. However, 
  \[
    \vdash \mathbf{I} : \sigma \to \sigma
    \notimplies
    \vdash \mathbf{K}_1\;\mathbf{I}\;\Omega : \sigma \to \sigma.
  \]
\end{example}
How to prove it?
\framebreak
\begin{lemma}[Typability of subterms]
  Let $M$ be a term with $\Gamma \vdash M : \tau$ derivable. Then, for every
  subterm $M'$ of~$M$ there exists $\Gamma'$ such that
  \[
    \Gamma' \vdash M' : \sigma'.
  \]
\end{lemma}
\begin{proof}
  By induction on $\Gamma \vdash M : \sigma$.
\end{proof}
  $\Omega$ is not typable, so $\mathbf{K}_1\,\mathbf{I}\,\Omega$ is not typable.
\end{frame}

\begin{frame}{A prelude to the preservation proof}
    \begin{description}
      \item[Weakening] If $\Gamma \vdash M : \tau$ and $x \not\in \Gamma$, then
        $\Gamma, x : \sigma \vdash M : \tau$. 
      \item[Substitution] If $\Gamma, x : \tau \vdash M : \sigma$ and $\Gamma
        \vdash N : \tau$ then $\Gamma \vdash M\subst{N}{x} : \sigma$.
    \end{description}

  \begin{corollary}[Variable renaming]
    If $\Gamma, x : \tau \vdash M :\sigma$ and $y \not\in \mathrm{dom}(\Gamma)$, then $ \Gamma, y : \tau \vdash M\subst{y}{x} :
    \sigma$
    where $\mathrm{dom}(\Gamma)$
    denotes the set of variables which occur in $\Gamma$.
  \end{corollary}
  \begin{proof}
    $y$ is not in $\Gamma$, so 
      $\Gamma, y : \tau, x : \tau \vdash M$
    by weakening and by definition $\Gamma, y : \tau \vdash y : \tau$.
    Thus, by substitution, we have
    \[
      \Gamma, y : \tau \vdash M\subst{x}{y} : \sigma
    \]
  \end{proof}
\end{frame}

\begin{frame}{Preservation Theorem}
  \begin{theorem}
    If $\Gamma \vdash M : \sigma$ is derivable and $M \onereduce N$, then $\Gamma \vdash N : \sigma$. 
  \end{theorem}
  \begin{proof}[Proof sketch]
  By induction on both the derivation of $\Gamma \vdash M :
  \sigma$ and $M \onereduce N$.

  The only non-trivial case is
  \[
    \Gamma \vdash (\lambda  x_1 : \tau .\, M_1)\; N : \sigma%    \quad\text{and}\quad %    (\lambda (x_1 : \tau).\, M_1)\;N \onereduce M_2\subst{x_2}{N}
  \]
  with the substitution lemma applied to
  \[
    \Gamma, x_1 : \tau \vdash M_1 : \sigma
    \quad\text{and}\quad
    \Gamma \vdash N : \tau.
  \]
\end{proof}
  
\end{frame}

\begin{frame}{Normal form}
  The notion of normal form can be characterised syntactically:
  \begin{definition}
    Define judgements $\texttt{Neutral}\;M$ and $\texttt{Normal}\;M$ mutually by
    \begin{multicols}{2}
      \begin{prooftree}
        \AXC{$\vphantom{\Gamma}$}
        \UIC{$\texttt{Neutral}\;x$}
      \end{prooftree}
      \begin{prooftree}
        \AXC{$\neutral\;M$}
        \AXC{$\normal\;N$}
        \BIC{$\neutral\;M\;N$}
      \end{prooftree}
      \columnbreak
      \begin{prooftree}
        \AXC{$\neutral\;M$}
        \UIC{$\normal\;M$}
      \end{prooftree}
      \begin{prooftree}
        \AXC{$\normal\;M$}
        \UIC{$\normal\;\lambda x.\,M$}
      \end{prooftree}
    \end{multicols}
  \end{definition}
  \textbf{Idea.} $\neutral\;M$ (resp.\, $\normal\;M$) is derivable
  iff 
  \[
    M \equiv x\;N_1 \cdots N_k
    \quad\text{and}\quad
    M \equiv \lambda x_1 \cdots x_n .\, x\;N_1 \cdots N_k
  \]
  respectively where $N_i$'s are in normal form.
\end{frame}

\begin{frame}{Soundness and completeness of the inductive characterisation}
  \begin{lemma}
    Let $M$ be an untyped term.
    \begin{description}
      \item[Soundness] If $\normal\;M$ (resp.\ $\neutral\;M$) is derivable, then $M$ is in normal form.
      \item[Completeness]
        If $M$ is in normal form, then $\normal\;M$ is derivable.
    \end{description}
  \end{lemma}
  \begin{proof}[Proof sketch.]
    \begin{description}
      \item[Soundness] By mutual induction on the derivation of
    $\normal\;M$ and $\neutral\;M$.
      \item[Completeness] By induction on the formation of $M$.
    \end{description}
  \end{proof}

  
\end{frame}
\begin{frame}{Progress}
  \begin{theorem}
    If $\Gamma \vdash M : \sigma$ is derivable, then $\normal\;M$ or there is $N$ with $M \onereduce N$.
  \end{theorem}
  \begin{proof}[Proof sketch]
    By induction on the derivation of $\Gamma \vdash M : \sigma$. 
  \end{proof}
\end{frame}


\begin{frame}{Weak normalisation}
  \begin{definition}
    $M$ is \emph{weakly normalising} denoted by $M\downarrow$ if 
    \begin{multicols}{2}
      \begin{prooftree}
        \AXC{$\normal\; M$}
        \UIC{$M \downarrow$}
      \end{prooftree}
      \columnbreak
      \begin{prooftree}
        \AXC{$M \onereduce N$}
        \AXC{$N \downarrow$}
        \BIC{$M \downarrow$}
      \end{prooftree}
    \end{multicols}
  \end{definition}
  That is, $M$ is weakly normalising if there is a sequence
  \[
    M \onereduce M_1 \onereduce M_2 \onereduce \dots N \notonereduce 
  \]
  \begin{theorem}[Weak normalisation]
    Every term $M$ with $\Gamma \vdash M : \tau$ is weakly normalising.
  \end{theorem}

\end{frame}

\begin{frame}{Strong normalisation}
  \begin{definition}
    $M$ is \emph{strongly normalising} denoted by $M \Downarrow$ if 
    \begin{prooftree}
      \AXC{$\forall N.\, (M \onereduce N \implies N \Downarrow)$}
      \UIC{$M \Downarrow$}
    \end{prooftree}
    
  \end{definition}
  Intuitively, \emph{strong normalisation} says every sequence
    \[
      M \onereduce M_1 \onereduce M_2 \cdots
    \]
  terminates.
  \begin{theorem}
    Every term $M$ with $\Gamma \vdash M : \tau$ is strongly normalising.
  \end{theorem}
  
\end{frame}

\begin{frame}{Definability}
%Within simply typed lambda calculus, we wonder how many computable functions can
%be defined, as it excludes non-terminating $\lambda$-terms. 
  A function $f\colon \mathbb{N}^k \to \mathbb{N}$ is called
  \alert{\emph{$\lambda_\to$-definable}} if there is a $\lambda$-term $F$ of
  type $\nat \to \nat \to \dots \nat \to \nat$ such that
  \[
    F\,\bc_{n_1}\ldots\bc_{n_k} \reduce \bc_{f(n_1, \dots, n_k)}
  \]
  for every sequence $(n_1, n_2, \ldots, n_k) \in \mathbb{N}^k$.
  Diagrammatically, 

\[
  \xymatrix{
    (n_1, n_2, \ldots, n_k) \ar@{|->}[rr] \ar@{|->}[d]_{(\bc_{-})^k} & & f(n_1,
    n_2, \ldots, n_k) \ar@{|->}[d]^{\bc_{-}}\\
    (\bc_{n_1}, \bc_{n_2}, \ldots, \bc_{n_k}) \ar@{|->}[rr] & & 
    F\;\bc_{n_1}\; \bc_{n_2}\; \ldots\;\bc_{n_k}
    = \bc_{f(n_1, n_2, \ldots, n_k)}
  }
\]
\end{frame}
\begin{frame}{The limit of $\lambda_\to$}
\begin{theorem}
  The $\lambda_\to$-definable functions are the class of functions
  of the form $f\colon \mathbb{N}^k \to \mathbb{N}$ closed under 
  compositions
  which contains
  \begin{itemize}
    \item the constant functions,
    \item projections,
    \item additions,
    \item multiplications,
    \item and the conditional 
  \[
    \mathrm{ifz}(n_0, n_1, n_2) = 
    \begin{cases}
      n_1 & \text{if } n_0 = 0\\
      n_2 & \text{otherwise.}
    \end{cases}
  \]
  \end{itemize}
\end{theorem}
\end{frame}

%\subsection{System T}
%It is also convenient to add some primitive types to our typed lambda calculus (or
%so-called built-in types in programming languages used in practice) as well as
%\emph{primitive terms}. For example, for a system with natural numbers, we
%include numerals in the generation of syntax:
%  \begin{multicols}{2}
%    \begin{prooftree}
%      \AXC{$x \in V$}
%      \UIC{$x \in \Lambda_{\T, \nat}$}
%    \end{prooftree}
%    \begin{prooftree}
%      \AXC{\phantom{$n \in \mathbb{N}$}}
%      \UIC{$\zero\in \Lambda_{\T, \nat}$}
%    \end{prooftree}
%    \begin{prooftree}
%      \AXC{$M \in \Lambda_{\T, \nat}$}
%      \UIC{$\suc\;M\in \Lambda_{\T, \nat}$}
%    \end{prooftree}
%    \begin{prooftree}
%      \AXC{$M \in \Lambda_{\T,\nat}$}
%      \AXC{$N \in \Lambda_{\T,\nat}$}
%      \BIC{$(M\, N) \in \Lambda_{\T, \nat}$}
%    \end{prooftree}
%    \begin{prooftree}
%      \AXC{$M \in \Lambda_{\T, \nat}$}
%      \AXC{$x \in V$}
%      \AXC{$\tau \in \mathbb{T}$}
%      \TIC{$\lambda (x:\tau).\; M \in \Lambda_{\T, \nat}$}
%    \end{prooftree}
%  \end{multicols}
%\noindent and let $\mathbb{G} \defeq \{\nat\}$ with an addition set of typing
%rules:
%\begin{multicols}{2}
%\begin{prooftree}
%  \AXC{\phantom{$\Gamma$}}
%  \UIC{$\Gamma \vdash \zero : \nat$}
%\end{prooftree}
%\begin{prooftree}
%  \AXC{$\Gamma \vdash M : \nat$}
%  \UIC{$\Gamma \vdash \suc\:M : \nat$}
%\end{prooftree}
%\end{multicols}
%\noindent In this way, we can derive that every $\underline{n} \defeq
%\suc^n\;\zero$ has the type of $\nat$:
%\[
%  \Gamma \vdash \underline{n} : \nat
%\]
%
%However, it still remains to add primitive operations such as addition,
%multiplication, division, and so on with a proper set of reduction rules. As we
%have seen that recursions on natural numbers in untyped lambda calculus can be
%defined with the fixpoint operator. Something similar can be done:
%\begin{definition}
%  In addition to typing rules introduced so far, we add the following
%  \begin{prooftree}
%    \AXC{$\Gamma \vdash M : \nat$}
%    \AXC{$\Gamma \vdash M_0 : \sigma$}
%    \AXC{$\Gamma \vdash F : \nat \to (\sigma \to \sigma)$}
%    \TIC{$\Gamma \vdash \fold\;M_0\;F : \nat \to \sigma $}
%  \end{prooftree}
%  with additional reductions
%  \[
%    \fold\;M_0\;F\;\zero
%    \xrightarrow{\beta}
%    M_0
%    \quad\text{and}\quad
%    \fold\;M_0\;F\;(\suc\;M)
%    \xrightarrow{\beta}
%    F\;M\;(\fold\;F\;M).
%  \]
%\end{definition}
%\begin{definition}
%  \todo[inline]{define addition, multiplication using $\fold$}
%\end{definition}
%\begin{frame}{Remark}
%With the decidability of type checking, Preservation Theorem, Progress Theorem
%(a well-typed term is either a ``value'' or a reducible term), and the strong
%normalisation, we actually have exhibited a decidable evaluator of simply typed
%lambda calculus that always reduce a well-typed term of type~$\sigma$ to a value
%of type~$\sigma$.
%\end{frame}

\begin{frame}{Homework}
  \begin{enumerate}
    \item (2.5\%) Show the Preservation Theorem. \\
      \textbf{Hint.} Apply the Substitution Lemma if applicable. 
    \item (2.5\%) Show the Progress Theorem.
    \item (2.5\%) Show that if $M$ is in normal form then $\normal\;M$ is derivable.
  \end{enumerate}
  
\end{frame}

%\begin{frame}{References}
%
%\bibliographystyle{amsalpha}
%\bibliography{library} 
%
%\end{frame}

\appendix
\section{Appendix Takahashi's Proof of confluence}

\begin{frame}{Confluence: Parallel reduction}
  Consider untyped $\lambda$-calculus. 

  Let $M \parreduce N$ denote the \emph{parallel reduction} defined by
  \begin{multicols}{2}
    \begin{prooftree}
      \AXC{$\vphantom{M\reduce M}$}
      \UIC{$x \parreduce x$}
    \end{prooftree}
    \begin{prooftree}
      \AXC{$M \parreduce N$}
      \UIC{$\lambda x.\, M \parreduce \lambda x.\, N$}
    \end{prooftree}
    \columnbreak
    \begin{prooftree}
      \AXC{$M \parreduce M'$}
      \AXC{$N \parreduce N'$}
      \BIC{$M\;N \parreduce M'\;N'$}
    \end{prooftree}
    \begin{prooftree}
      \AXC{$M \parreduce M'$}
      \AXC{$N \parreduce N'$}
      \BIC{$(\lambda x.\, M)\;N \parreduce M'\subst{N'}{x}$}
    \end{prooftree}
  \end{multicols}
  For example, 
  \[
    \underline{(\lambda x.\, (\lambda y.\, y)\;x)}\;\underline{((\lambda x.\, x)\;\false)}
    \parreduce
    \false
  \]
  because $(\lambda y.\,y)\;x \parreduce x$ and $(\lambda x.\,x)\;\false \parreduce \false$.
\end{frame}

\begin{frame}{Confluence: Properties of parallel reduction}
  \begin{lemma}
    \begin{enumerate}
      \item $M \parreduce M$ holds for any term~$M$, 
      \item $M \onereduce N$ implies $M \parreduce N$, and
      \item $M \parreduce N$ implies $M \reduce N$.
    \end{enumerate}
  \end{lemma}
  Therefore, $M \parreduce^* N$ is equivalent to $M \reduce N$. 
  \begin{lemma}[Substitution respects parallel reduction]
      $M \parreduce M'$ and $N \parreduce N'$ imply $M\subst{N}{x} \parreduce
      M'\subst{N'}{x}$. 
  \end{lemma}
  \begin{proof}[Proof sketch]
    By induction on the derivation of $M \parreduce M'$.
    
  \end{proof}

\end{frame}

\begin{frame}{Complete development}
  The \emph{complete development} $M^*$ of a $\lambda$-term $M$ is defined by
  \begin{align*}
    x^*      & = x \\
    (\lambda x.\, M)^* & = \lambda x.\, M^* \\
    \left((\lambda x.\, M)\;N\right)^* & = M^*\subst{N^*}{x} \\
    (M\;N)^* & = M^*\;N^* && \text{ if $M \not\equiv \lambda x.\,M'$ } 
  \end{align*}
  \begin{theorem}[Triangle property]
    If $M \parreduce N$, then $N \parreduce M^*$.
  \end{theorem}
  \begin{proof}[Proof sketch]
    By induction on $M \parreduce N$.
    
  \end{proof}
\end{frame}

\begin{frame}{Strip Lemma}
  \begin{theorem}
    If $L \parreduce^* M_1$ and $L \parreduce M_2$, then there exists $N$
    satisfying that $M_1 \parreduce N$ and $M_2 \parreduce^* N$, i.e.\
    \[
      \xymatrix{
        & L \ar@{=>}[rd] \ar@{=>}[ld]^(.8){*}_(.8){\beta} \\
        M_1 \ar@{=>}[rd] & & M_2 \ar@{=>}[ld]^(.8){*}_(.8){\beta} \\
            & N
      }
    \]
  \end{theorem}
  \begin{proof}[Proof sketch]
    By induction on $L \parreduce^* M_1$. 
  \end{proof}
\end{frame}

\begin{frame}{Confluence}
  \begin{theorem}
    If $L \parreduce^* M_1$ and $L \parreduce^* M_2$, then there exists $N$ such that $M_1 \parreduce^* N$ and $M_2 \parreduce^* N$.
    \[
      \xymatrix{
        & L \ar@{=>}[rd]^(.8){*}_(.8){\beta} \ar@{=>}[ld]^(.8){*}_(.8){\beta} \\
        M_1 \ar@{=>}[rd]^(.8){*}_(.8){\beta} & & M_2 \ar@{=>}[ld]^(.8){*}_(.8){\beta} \\
            & N
      }
    \]
  \end{theorem}
  \begin{corollary}
    The confluence of $\reduce$ holds. 
    
  \end{corollary}
  
\end{frame}
\end{document}
