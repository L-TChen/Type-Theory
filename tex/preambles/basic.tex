\usetheme[block=fill,titleformat=smallcaps,progressbar=foot]{metropolis}

\author[L.-T. Chen]{陳亮廷 Chen, Liang-Ting }
\institute{Institute of Information Science\\Academia Sinica}
\date[FLOLAC'24]{Formosan Summer School on Logic, Language, and Computation 2024}

%\setbeamercolor{frame}{fg=black}

\newcounter{saveenumi}
\newcommand{\seti}{\setcounter{saveenumi}{\value{enumi}}}
\newcommand{\conti}{\setcounter{enumi}{\value{saveenumi}}}
\mode<presentation>{\resetcounteronoverlays{saveenumi}}

\usepackage[british]{babel}
\mode<article>{\usepackage[a4paper]{geometry}}
\usepackage{bussproofs}
\EnableBpAbbreviations
\usepackage{mathpartir}
\usepackage{multicol}
\usepackage{centernot}
\usepackage{mathtools}
\usepackage{xspace}

\usepackage{graphicx}
\usepackage[all,cmtip]{xy}
\mode<article>{\usepackage{hyperref}}
\usepackage{tikz}
\usetikzlibrary{trees,decorations.pathreplacing,fit}
\usepackage{microtype}

\setbeamertemplate{theorems}[numbered]
\theoremstyle{plain}
\newtheorem*{theorem*}{Theorem}
\newtheorem{proposition}[theorem]{Proposition}
\newtheorem{convention}[theorem]{Convention}
\newtheorem*{convention*}{Convention}
\newtheorem*{quiz*}{Quiz}
\newtheorem{remark}{Remark}[section]
\theoremstyle{definition}
\newtheorem*{exercise*}{Exercise}

\newenvironment<>{idea}[1]{\begin{block}{Idea}#1}{\end{block}}

\usepackage{fontspec}
\usepackage{xeCJK}

%\setmainfont{Fira Sans}
\setsansfont[BoldFont={Fira Sans Light}]{Fira Sans}
\setmonofont{Fira Code}
\setCJKmainfont{Noto Serif TC}
\setCJKsansfont{Noto Sans TC}
\setCJKmonofont{Fira Code}

\newcommand{\FV}{\mathbf{FV}}
\newcommand{\Var}{\mathbf{Var}}
\newcommand{\BV}{\mathbf{BV}}
\newcommand{\term}{\mathtt{Term}}
\newcommand{\type}{\mathtt{Type}}
\newcommand{\reason}[1]{\{\text{\,#1\,}\}}
\renewcommand{\o}{\circ}
\newcommand{\mnd}[1]{{\overleftarrow{#1}}}
\newcommand{\cmnd}[1]{{\overrightarrow{#1}}}
\makeatletter
\newcommand*{\defeq}{\mathrel{\rlap{%
                     \raisebox{0.3ex}{$\m@th\cdot$}}%
                     \raisebox{-0.3ex}{$\m@th\cdot$}}%
                     =}
\makeatother

\newcommand{\op}{\mathrm{op}}
\newcommand{\subst}[2]{[{#1}/{#2}]}

\newcommand{\Hask}{\textsc{Haskell}\xspace}
\newcommand{\Python}{\textsc{Python}}
\newcommand{\OCaml}{\textsc{OCaml}}

\newcommand{\leftexp}[2]{{\vphantom{#2}}^{#1}{#2}}
\newcommand{\set}[2]{\{\,#1\mid#2\,\}}
\newcommand{\sem}[1]{[\![#1]\!]} 
\newcommand{\id}{\mathit{id}}

\newcommand{\M}{\mathsf{M}}
\newcommand{\N}{\mathsf{N}}
\newcommand{\V}{\mathsf{V}}
\newcommand{\T}{\mathbb{T}}

% Macros for lambda calculus
\DeclareRobustCommand\longtwoheadrightarrow
     {\relbar\joinrel\twoheadrightarrow}

\newcommand{\reduce}{\longtwoheadrightarrow_{\beta}}
\newcommand{\parreduce}{\Longrightarrow_{\beta}}
\newcommand{\onereduce}{\longrightarrow_{\beta}}
\newcommand{\nat}{\mathtt{nat}}
\newcommand{\bool}{\mathtt{bool}}
\newcommand{\List}{\mathtt{list}}
\newcommand{\Bool}{\mathbb{B}}

\newcommand{\bc}{\mathbf{c}}
\newcommand{\add}{\mathtt{add}}
\newcommand{\mul}{\mathtt{mul}}
\newcommand{\fix}{\mathtt{fix}}
\newcommand{\natrec}{\mathtt{natrec}}
\newcommand{\fold}{\mathtt{fold}}
\newcommand{\suc}{\mathtt{suc}}
\newcommand{\pred}{\mathtt{pred}}
\newcommand{\ifz}{\mathtt{ifz}}
\newcommand{\cond}{\mathtt{cond}}
\newcommand{\zero}{\mathtt{zero}}
\newcommand{\true}{\mathtt{true}}
\newcommand{\false}{\mathtt{false}}
\newcommand{\neutral}{\mathtt{Neutral}}
\newcommand{\normal}{\mathtt{Normal}}

\newcommand{\PCF}{{\textbf{PCF}}}

\newcommand{\notimplies}{\centernot\implies}
\newcommand{\notonereduce}{\centernot\onereduce}

\newcommand {\textplus}{\nolinebreak\hspace{-.05em}\raisebox{.4ex}{\tiny\bf +}}

\makeatletter
%%% Lambda abstractions.
% Each variable being abstracted over is a separate argument.  If
% there is more than one such argument, they *must* be enclosed in
% braces.  Arguments can be untyped, as in \lam{x}{y}, or typed with a
% colon, as in \lam{x:A}{y:B}. In the latter case, the colons are
% automatically noticed and (with current implementation) the space
% around the colon is reduced.  You can even give more than one variable
% the same type, as in \lam{x,y:A}.
\def\lam#1{{\lambda}\@lamarg#1:\@endlamarg\@ifnextchar\bgroup{.\,\lam}{.\,}}
\def\@lamarg#1:#2\@endlamarg{\if\relax\detokenize{#2}\relax #1\else\@lamvar{\@lameatcolon#2},#1\@endlamvar\fi}
\def\@lamvar#1,#2\@endlamvar{(#2\,{:}\,#1)}
% \def\@lamvar#1,#2{{#2}^{#1}\@ifnextchar,{.\,{\lambda}\@lamvar{#1}}{\let\@endlamvar\relax}}
\def\@lameatcolon#1:{#1}
\let\lamt\lam
% This version silently eats any typing annotation.
\def\lamu#1{{\lambda}\@lamuarg#1:\@endlamuarg\@ifnextchar\bgroup{.\,\lamu}{.\,}}
\def\@lamuarg#1:#2\@endlamuarg{#1}
\makeatother
