%! TEX root = lecture3_slide.tex
\usepackage{listings}
\title{Lambda Calculus and Types}
\subtitle{Parametric Polymorphism}
\begin{document}
\begin{frame}
\maketitle
\end{frame}

\begin{frame}[fragile]{Polymorphism}
  \emph{Abstraction principle~\cite{Pierce2002}}:
  \begin{quotation}
    Each significant piece of functionality in a program should be implemented
    in just one place in the source code. Where similar functions are carried
    out by distinct pieces of code, it is generally beneficial to combine them
    into one by abstracting out the varying parts.
  \end{quotation}

  In Haskell
  \begin{semiverbatim}
    head :: [a] -> a
    head (x:_) = x
  \end{semiverbatim}
\end{frame}

\section{System F---Polymorphic Lambda Calculus}
\begin{frame}{Types}
  Given type variables~$\mathbb{V}$, the set $\type$ of \emph{types} is
  defined by
  \begin{multicols}{2}
    \begin{prooftree}
      \AXC{$t \in \mathbb{V}$}
      \UIC{$t \in \type$}
    \end{prooftree}
    \begin{prooftree}
      \AXC{$\sigma \in \type$}
      \AXC{$\tau   \in \type$}
      \BIC{$\sigma \to \tau \in \type$}
    \end{prooftree}
    \color{red}
    \begin{prooftree}
      \AXC{$\sigma \in \type$}
      \AXC{$t   \in \mathbb{V}$}
      \RightLabel{(poly)}
      \BIC{$\forall t.\, \sigma \in \type$}
    \end{prooftree}
  \end{multicols}
For example, the type $\forall t.\, t \to t$ corresponds to a type variable
universally quantified in GHC with extension \texttt{RankNType}
\begin{semiverbatim}
  forall a. a -> a
\end{semiverbatim}
or simply 
\begin{semiverbatim}
  a -> a
\end{semiverbatim}
\end{frame}


\begin{frame}{Free and Bound Variables, again}
\begin{definition}
  The \emph{free variable} $\FV\colon \type \to \mathcal{P}\mathbb{V}$ of a type
  is defined by
  \begin{align*}
    \FV(t) & {} = t \\
    \FV(\sigma \to \tau) & {} = \FV(\sigma) \cup \FV(\tau) \\
    \FV(\forall t.\, \sigma) & {} = \FV(\sigma) - \{t\}
  \end{align*}
  For convenience, the function extends to contexts:
  \[
    \FV(\Gamma) = \set{ t \in \mathbb{V}}{ \exists (x : \sigma) \in \Gamma
      \wedge t \in \FV(\sigma) }.
  \]
\end{definition}
  \begin{enumerate}
    \item $\FV(t_1) = \{t_1\}$.
    \item $\FV(\forall t.\, (t \to t) \to t \to t) = \emptyset$.
    \item $\FV(x : t_1, y : t_2, z : \forall t.\,t)
      = \{t_1, t_2\}$.
  \end{enumerate}
\end{frame}
\begin{frame}{Capture-Avoiding Substitution for Type}
  \begin{definition}
  The \emph{(capture-avoidance) substitution} of a type $\rho$ for the free
  occurrence of a type variable~$t$ is defined by 
  \begin{align*}
    t\subst{t}{\rho} & = \rho \\
    u\subst{t}{\rho} & = u && \text{if $u \neq t$}\\
    (\sigma\to\tau)\subst{t}{\rho} & =
    \sigma\subst{t}{\rho}\to
    \tau\subst{t}{\rho} \\
    (\forall t.\sigma)\subst{t}{\rho} & = \forall t.\sigma \\
    (\forall u.\sigma)\subst{t}{\rho} & = \forall u.\sigma\subst{t}{\rho}
                                      &&
    \text{if $u \neq t, u \not\in \FV(\rho)$} 
  \end{align*}
  \end{definition}
  Recall that $u \not\in \FV(\rho)$ means that $u$ is \emph{fresh} for $\rho$. 
\end{frame}

\begin{frame}[allowframebreaks]{Typed $\lambda$-terms}
\begin{definition}
  Given a set~$V$ of variables, a set $\mathbb{V}$ of type variables,
  the set terms for System $F$ is defined by
  \begin{multicols}{2}
    \begin{prooftree}
      \AXC{$x \in V$}
      \UIC{$x \in \term$}
    \end{prooftree}
    \begin{prooftree}
      \AXC{$M \in \term$}
      \AXC{$N \in \term$}
      \BIC{$(M\;N) \in \term$}
    \end{prooftree}
    \begin{prooftree}
      \AXC{$M \in \term$}
      \AXC{$x \in V$}
      \AXC{$\tau \in \type$}
      \TIC{$\lambda (x : \tau).\, M \in \term$}
    \end{prooftree}
    \color{red}
    \begin{prooftree}
      \AXC{$M \in \term$}
      \AXC{$t \in \mathbb{V}$}
      \BIC{$\Lambda t.\; M \in \term$}
    \end{prooftree}
    \begin{prooftree}
      \AXC{$M \in \term$}
      \AXC{$\tau \in \type$}
      \BIC{$(M\;\tau) \in \term$}
    \end{prooftree}
  \end{multicols}
\end{definition}
\framebreak
GHC with the extension \texttt{ScopedTypeVariables} allows you to use type
variables. 
\begin{semiverbatim}
 f :: forall a. [a] -> [a]

 f = \\(xs :: [a]) -> reverse xs
\end{semiverbatim}
GHC with the extension \texttt{TypeApplications} allows you to apply a type
argument.
\begin{semiverbatim}
  Prelude> :t id

  id :: a -> a

  Prelude> :t id @Int

  id @Int :: Int -> Int
\end{semiverbatim}

\framebreak
  Can you see the corresponding untyped/simply typed $\lambda$-terms of the
  following $\lambda$-terms in System F? 

  \begin{enumerate}
    \item $\Lambda t.\,\lambda (x : t).\, x$
    \item $\Lambda t.\, \lambda (x : t)(y : t).\, x$
    \item $(\Lambda t.\, \lambda (x : t)(y : t).\, x)\;\sigma\;M\;N$
     where  $M, N$ are terms and $\sigma$ a type.
    \item $\Lambda t.\, \lambda (f : t \to t)(x : t).\, f\;(f\;x)$
  \end{enumerate}
System $F$ is more expressive than simply typed lambda calculus.
\end{frame}

\begin{frame}{Typing Rules}
  
The typing rule for simply typed lambda calculus are extended to
accommodate polymorphism: 
  \begin{multicols}{2} 
  \begin{prooftree}
    \AXC{$\phantom{\Gamma}$}
    \UIC{$\Gamma, x : \sigma \vdash x : \sigma$}
  \end{prooftree}
  \begin{prooftree}
    \AXC{$\Gamma \vdash M : (\sigma \to \tau)$}
    \AXC{$\Gamma \vdash N : \sigma$}
    \BIC{$\Gamma \vdash (M\;N) : \tau$}
  \end{prooftree}
  \begin{prooftree}
    \AXC{$\Gamma, x : \sigma \vdash M : \tau$}
    \UIC{$\Gamma \vdash \lambda (x : \sigma).\; M : (\sigma \to \tau)$}
  \end{prooftree}
  \color{red}
  \begin{prooftree}
    \AXC{$\Gamma, t \vdash M : \sigma$}
    \RightLabel{($\forall$-intro)}
    \UIC{$\Gamma \vdash \Lambda t.\;M : \forall t.\, \sigma$}
  \end{prooftree}
  \begin{prooftree}
    \AXC{$\Gamma \vdash M : \forall v.\, \sigma$}
    \RightLabel{($\forall$-elim)}
    \UIC{$\Gamma \vdash (M\;\tau) : \sigma\subst{v}{\tau}$}
  \end{prooftree}
  \end{multicols}
\end{frame}
\begin{frame}{Typing Derivation}

The typing judgement ${}\vdash\Lambda t.\, \lambda (x : t)(y : t).\, x : \forall
t.\;t \to t \to t$ is derivable from the following derivation:
\begin{prooftree}
  \AXC{}
  \RightLabel{(var)}
  \UIC{$x : t, y : t \vdash x : t$}
  \UIC{$x : t \vdash \lambda (y : t).\; x : t \to t$}
  \UIC{$\vdash \lambda (x : t)(y : t).\; x : t \to t \to t$}
  \UIC{$\vdash \Lambda t.\,\lambda (x : t)(y : t).\; x :\forall t.\,t \to t \to
    t$}
\end{prooftree}
  
\begin{block}{Exercise}
  Derive the following judgements:
  \begin{enumerate}
    \item ${}\vdash\Lambda t.\,\lambda (x : t).\, x : \forall t.\;t\to t$
    \item $a : \sigma, b : \sigma
      \vdash (\Lambda t.\, \lambda (x : t)(y : t).\, x)\;\sigma\;a\;b
      : \sigma$
    \item ${}\vdash\Lambda t.\, \lambda (f : t \to t)(x : t).\, f\;(f\;x) :
      \forall t.\;(t\to t) \to t\to t$
    \end{enumerate}
\end{block}
\end{frame}

\begin{frame}{Substitution for Term}
  
In line with the new syntax, we need to extend \emph{substitution over
  terms}. Pretty obvious and routine. 

\mode<presentation>{\vfill}
\begin{block}{Homework}
  Define \emph{capture-avoiding substitution} for terms of System F. 
\end{block}

\end{frame}


\begin{frame}{Evaluation}
  
The $\beta$-conversion has two rules
\[
  (\lambda (x : \tau).\, M)\,N \longrightarrow_{\beta}
  M\subst{x}{N}
  \quad\text{and}\quad
  \color{red} (\Lambda t.\, M)\;\tau \longrightarrow_{\beta} M \subst{t}{\tau}
\]
The full $\beta$-reduction is a relation on $\lambda$-terms defined by
\begin{multicols}{2}
    \begin{prooftree}
      \AXC{$M_1 \longrightarrow_{\beta} M_2$}
      \UIC{$M_1 \onereduce M_2$}
    \end{prooftree}
    \begin{prooftree}
      \AXC{$M_1 \onereduce M_2$}
      \UIC{$\lambda (x:\tau).\, M_1 \onereduce \lambda (x:\tau).\, M_2$}
    \end{prooftree}
    \begin{prooftree}
      \AXC{$M_1 \onereduce M_2$}
      \UIC{$M_1\,N \onereduce M_2\,N$}
    \end{prooftree}
    \begin{prooftree}
      \AXC{$N_1\onereduce  N_2$}
      \UIC{$M\,N_1 \onereduce M\,N_2$}
    \end{prooftree}
\end{multicols}
If $M \onereduce N$, then $M$ and $N$ \alert{denotes} the same value. So, 
\[
  M =_\beta N
\]
where $=_\beta$ is the congruence and equivalence closure of
$\longrightarrow_\beta$.
\end{frame}

% \begin{frame}{Extension: General Recursion}
% \emph{General recursion} can be supported by adding a fixpoint construct:
%   \begin{prooftree}
%     \AXC{$x \in V$}
%     \AXC{$M \in \term$}
%     \BIC{$\fix x.\; M$}
%   \end{prooftree}
%   with additional $\delta$-reduction rule:
%   \begin{multicols*}{2}
%   \begin{prooftree}
%     \AXC{$\phantom{\fix}$}
%     \UIC{$\fix x.\; M \longrightarrow_{\delta} M\subst{x}{\fix x.\; M}$}
%   \end{prooftree}
%   \columnbreak
%   \begin{prooftree}
%     \AXC{$M \longrightarrow_{\delta1} M'$}
%     \UIC{$\fix x.\, M \longrightarrow_{\delta1} \fix x.\, M$}
%   \end{prooftree}
%   \end{multicols*}
%   and adopted definitions for free variable and substitution. And, there is one
%   more typing rule which is \alert{\emph{logically inconsistent}}:
%   \begin{prooftree}
%     \AXC{$\Gamma, x : \sigma \vdash M : \sigma$}
%     \UIC{$\Gamma \vdash \fix (x : \sigma).\; M : \sigma$}
%   \end{prooftree}
% \end{frame}

\section{Programming in System F}
\begin{frame}{Self Application}
Self-application is not typable in simply typed lambda calculus. 
  \[
    \lambda (x : t).\, x\; x
  \]
  However, self-application is possible in System F. 
  \[
    \lambda (x : \forall t. t\to t).\, x\;(\forall t. t\to t)\;x
  \]
  \vfill
  \begin{block}{Exercise}
    Instantiate the first $t$ with the type $\forall t.\, t \to t$.  
  \end{block}
\end{frame}
\begin{frame}[allowframebreaks]{Church Encoding of Bottom Type}
\begin{definition}
The bottom type is a type defined by
\[
  \bot \defeq \forall t.\,t
\]
\end{definition}
\begin{block}{Exercise}
  Does the bottom type contain any inhabitant? Try to define 
  one in System F and one in Haskell
  \begin{semiverbatim}
    contradiction :: forall a. a

    contradiction = ?
  \end{semiverbatim}
\end{block}
\end{frame}
\begin{frame}[allowframebreaks]{Church Encoding of Sum Type}

\begin{definition}
  The \emph{sum type} (i.e.\ \texttt{Either} in Haskell) is defined by
  \[
    \sigma + \tau \defeq \forall t. (\sigma \to t) \to (\tau \to t) \to t
  \]
\end{definition}
It has two injection functions: the first injection is defined by
\begin{align*}
  \mathtt{left} & \defeq \lambda (x : \sigma).\;\Lambda t.\,\lambda (f : \sigma\to
  t)(g : \tau\to t).\, f\;x \\
  \mathtt{right} & \defeq \;?
\end{align*}
\framebreak

With the sum type, we can implement the usual construct \texttt{either}:
\[
  \mathtt{either} : \forall\,t.\;(\sigma \to t) \to (\tau \to t)
  \to (\sigma + \tau) \to t
\]
which corresponds to the Haskell function
{\small
\begin{semiverbatim}
  either :: (a -> c) -> (b -> c) -> Either a b -> c

  either f g (Left  a) = f a 

  either f g (Right b) = g b
\end{semiverbatim}}

\mode<presentation>{\vfill}
\begin{block}{Exercise}
  Define the other injection function $\mathtt{right}$.  Define
  $\mathtt{either}$ in System F.
\end{block}
\end{frame}

\begin{frame}{Church Encoding of Product Type}
\begin{definition}[Product Type]
  The product type is defined by
  \[
    \sigma \times \tau \defeq \forall t. (\sigma \to \tau \to t) \to t
  \]
\end{definition}
The pairing function is defined by
\begin{align*}
  \left< \_, \_\right> \defeq \lambda (x : \sigma)(y : \tau).\,\Lambda t.\,\lambda (f : \sigma \to \tau \to t).\,
  f\;x\;y
\end{align*}
\begin{block}{Exercise}
Define projections 
\[
  \mathtt{proj}_1 : \sigma \times \tau \to \sigma
  \quad\text{and}\quad
  \mathtt{proj}_2 : \sigma\times \tau \to \tau
\]
\end{block}
\end{frame}
\begin{frame}[allowframebreaks]{Church Encoding of Natural Numbers}
The type of Church numerals is defined by 
\[
  \nat \defeq \forall t.\, (t\to t)\to t\to t
\]
  \begin{description}
    \item[Church numerals]
      \begin{align*}
        \bc_n & : \nat \\
        \bc_n & \defeq \Lambda t.\,\lambda (f:t \to t)\,(x:t).\,
        f^n\;x
      \end{align*}
    \item[Successor]
      \begin{align*}
        \suc & : \nat \to \nat \\
        \suc & \defeq \,\lambda (n : \nat).\,\Lambda t.\,\lambda
        (f : t \to t)\,(x:t)\,.\;f\;(n\;t\;f\;x) 
      \end{align*}
    \item[Addition]
      \begin{align*}
        \add & : \nat \to \nat \to \nat \\
        \add & \defeq \lambda (n : \nat)\,(m:\nat)\,&& \Lambda t.\, \lambda
        (f:t\to t)\,(x: t).\\
        &&& (m\;t\;f)\;(n\;t\;f\;x) 
      \end{align*}
    \item[Multiplication] 
      \begin{align*}
       \mul & : \nat\to \nat \to\nat\\
       \mul & \defeq\,?
      \end{align*}
    \item[Conditional]
      \begin{align*}
       \ifz & : \forall t.\,\nat \to t \to t \to t \\
       \ifz & \defeq\,?
      \end{align*}
  \end{description}
However, polymorphic lambda calculus allows us to define
\emph{iterator} over natural numbers or the \texttt{fold} function in Haskell.
\begin{align*}
  \mathtt{fold} & : \forall t.\, (t \to t) \to t \to \nat \to t  \\
  \mathtt{fold} & \defeq \Lambda t.\, \lambda (f : t \to t)(e_0 : t)(n : \nat). 
  n\;t\;f\;e_0
\end{align*}

In Haskell, the above $\lambda$-term can be given as
\begin{semiverbatim}
  fold :: (a -> a) -> a -> [()] -> a

  fold f e []            = e

  fold f e (():xs) = f () (fold f e xs)
\end{semiverbatim}

\begin{block}{Exercise}
  Define $\add$ and $\mul$ using $\mathtt{fold}$ and justify your
  answer.
  
  \begin{enumerate}
    \item $\add'\defeq \;? : \nat\to\nat\to\nat$
    \item $\mul'\defeq \;? : \nat\to\nat\to\nat$
  \end{enumerate}
\end{block}

\mode<presentation>{\vfill}
\begin{block}{Homework}
  Show that
  \begin{align*}
    & \mathtt{fold}\;\tau\;f\;e_0\;\bc_0 && =_\beta e_0 \\
    & \mathtt{fold}\;\tau\;f\;e_0\;\bc_{n + 1} && =_\beta
    f\;(\mathtt{fold}\;\tau\;f\;e_0\;\bc_n)
  \end{align*}
\end{block}
\framebreak
\end{frame}

\begin{frame}[allowframebreaks]{Church Encoding of Lists}
\begin{definition}
  For any type $\sigma$, the type of lists over $\sigma$ is 
\[
  \List\,\sigma\defeq \forall t.\, t \to (\sigma \to t \to t) \to t 
\]
\end{definition}
with ``list constructors'':
\[
  \mathtt{nil}_\sigma \defeq \Lambda t.\lambda (h : t)(f : \sigma\to t \to t).\,
  h
\]
and  
\[
  \mathtt{cons}_\sigma \defeq \lambda (x : \sigma)(xs : \List\,
  \sigma).\,\Lambda t.\lambda(h : t)(f : \sigma\to t \to t).f\,x\,(xs\; t\;
  h\; f)
\]

\framebreak
\begin{block}{Homework}
\begin{enumerate}
  \item Define $\mathtt{fold}$ over the list type over~$\sigma$.
  \item Define $\mathtt{length}_{\sigma} : \List\,\sigma \to \nat$ calculating
    the length of a given list so that
    \[
      \mathtt{length}\;\mathtt{nil}_\sigma =_\beta \bc_0
    \]
    and
    \[
      \mathtt{length}\;(\mathtt{cons}_\sigma\;x\;xs)
      =_\beta \suc\;(\mathtt{length}\;xs) 
    \]
  \item The above type of lists depends on a specific
    type~$\sigma$. Generalise it further. 
  \end{enumerate}
  
\end{block}
\framebreak

\end{frame}

\begin{frame}[allowframebreaks]{Abstract Data Type  and Existential Type}
  Type does not only prevent simple type-mismatch but also help program
  abstraction, e.g., abstract data types where the implementation detail is
  invisible to outside. 
  
  \begin{example}
    Roughly speaking, a \emph{stack} is a data structure with two operations:
    \begin{itemize}
      \item \texttt{push}
      \item \texttt{pop}
    \end{itemize}
    Any implementation satisfying $\mathtt{pop} \circ (\mathtt{push}\, n\, st) =
    (n, st)$ can be seen as a stack. 
  \end{example}
  \framebreak
  Such abstraction can be described by an \emph{existential type}.
  \[
    \exists t.\, \tau(t)
    \quad\text{or simply}\quad
    \exists t.\, \tau
  \]
  where $t$ is a type variable and $\tau$ is a type (potentially 
  containing $t$ as a free variable).
\mode<presentation>{\vfill}
  A \emph{witness} of type $\exists t.\, \tau$ is a pair $(\sigma, M)$ of a
  \emph{type} $\sigma$ and an element $M$ of type $\tau\subst{t}{\sigma}$. 

\mode<presentation>{\vfill}
  (The typing rule and $\beta$-conversion are omitted here.)
  \framebreak
  \begin{example}
    The type of stacks over natural numbers can be defined by
    \[
      \exists t. (t \times \nat \to t) \times (t \to 1 + \nat \times t) 
    \]
    \begin{enumerate}
      \item The first component is the \text{push} function.
      \item The second component is the \text{pop} function.
    \end{enumerate}
\mode<presentation>{\vfill}
    An instance of a stack is an element $(\sigma, M)$ of the above type
    \begin{enumerate}
      \item $\sigma$ is a type
      \item $\mathtt{proj}_1\, M \colon \sigma \times \nat \to \sigma$
      \item $\mathtt{proj}_2\, M \colon \sigma \to 1 + \nat \times
          \sigma$. 
    \end{enumerate}
  \end{example}
\end{frame}

\begin{frame}[fragile]{Encoding of Existential Type in Haskell}
  A limited form of existential type can be defined by 
  \begin{semiverbatim}
    data Exists t where
      Ex :: t a -> Exists t
  \end{semiverbatim}
  and another version for typeclass
  \begin{semiverbatim}
    data Exists' c where
      Ex' :: c a => a -> Exists' c
  \end{semiverbatim}
  in GHC with extensions \texttt{GADTs} and \texttt{ConstraintKinds}. 

\mode<presentation>{\vfill}
  \begin{block}{Exercise}
    Define the top type $\top$ which contains every typable term.
  \end{block}
\end{frame}

\begin{frame}{Encoding of Existential Type in System F}
  The existential type can also be encoded in System F as 
  \[
    \exists t.\, \tau \defeq \forall u.\; (\forall t.\; \tau \to u) \to u
  \]
  using continuation passing style.\footnote{Recall that $\exists x.\,\varphi
  = (\forall x.\, (\varphi \to \bot)) \to \bot$ classically.  }
\mode<presentation>{\vfill}
  A witeness $(\sigma, M)$ of $\exists t.\, \tau$ is encoded as
  \[
    \Lambda u.\,\lambda (f : \forall t.\, \tau \to u).\, f\;\sigma\;M
  \]

  \begin{block}{Exercise}
    Check that the above terms do make sense. How to use a term of some existential
    type? 
  \end{block}
  
\end{frame}
\section{Properties of System F and Beyond}
\begin{frame}{Type Safety and Normalisation}
  \begin{theorem}[Preservation]
    If $\Gamma \vdash M : \sigma$ 
    and $M \onereduce N$, then
    $\Gamma \vdash N : \sigma$. 
  \end{theorem}

  \begin{theorem}[Progress]
    If ${}\vdash M : \sigma$, then either $M$ is a value or
    there is $N$ such that $M \onereduce N$. 
  \end{theorem}
  The above two properties are proved by induction.

  \begin{theorem}[Strong Normalisation]
    If $\vdash M : \sigma$, then $M$ terminates. 
  \end{theorem}
\end{frame}

\begin{frame}{Type Erasure}
\begin{definition}
  The \emph{erasing map} is a function defined by
  \begin{align*}
    |x| & = x \\
    |\lambda (x : \tau).\,M| & = \lambda x.\, |M| \\
    |M\;N| & = (|M|\;|N|) \\
    |\Lambda t.\, M| & = |M| \\
    |M\;\tau| & = |M|
  \end{align*}
\end{definition}

\begin{proposition}
  Within System F, if ${}\vdash M : \sigma$ and $|M|
  \onereduce N'$, then there exists a well-typed term~$N$ with
  ${}\vdash N : \sigma$ and $|N| = N'$.
\end{proposition}
\end{frame}

\begin{frame}{Undecidability of Type Inference}
  \begin{theorem}
     It is undecidable whether, given a closed term $M$ of the untyped
     lambda-calculus, there is a well-typed term $M'$ in System F such that
     $|M'| = M$.  
  \end{theorem}

  \begin{description}
    \item[Arbitrary Rank Polymorphism] $\forall$ can appear
      anywhere {\small (GHC with \texttt{-XRankNType})}. 
    \item[Rank-1 Polymorphism]
      $\forall$ only appear in the outermost position.
  \end{description}
  \emph{Hindley-Milner type system} (extended by Haskell 98, Standard ML, etc.)
  only supports rank-1 polymorphism and type inference becomes
  decidable.\footnote{It is decidable up to rank-$2$ polymorphism.}
\end{frame}

\begin{frame}{Definability}
  \emph{Second-order Peano Arithmetic} is a set of axioms for natural numbers in
  second-order classical logic where the \emph{induction scheme} in PA---for
  every predicate $P$, 
  \[
    P(0) \to (\forall x.\, Px \to P(\mathtt{succ}\,x)) \to \forall a.\,P
  \]
  is replaced by the \emph{induction axiom}:
  \[
    {\color{red} \forall P}.\,(P(0) \to
    \forall x.\,Px \to P(\mathbf{succ}\,x)) \to 
    \forall a.\, P
  \]
  \begin{theorem}[Girard]
    The class of functions definable in System F is exactly 
    the class of provable recursive functions of second-order Peano Arithmetic.
  \end{theorem}
  
  What does it mean via Curry-Howard correspondence? 
\end{frame}

\begin{frame}{Parametricity}
  What functions can you write for the following type?
  \[
    \Lambda t. t \to t 
  \]

\mode<presentation>{\vfill}
  \begin{block}{Homework}
    Read \cite{Wadler1989} if interested. 
  \end{block}
\end{frame}

\begin{frame}{System $F_\omega$}
  Recall that for $\sigma, \tau \in \type$ the sum type of $\sigma$ and $\tau$
  is 
  \[
    \sigma + \tau \defeq \forall t. (\sigma \to t) \to (\tau \to t) \to t
  \]
  Can we internalise this \alert{type construction}? 
  \vfill
  \alert{\emph{Kinds}} are like \emph{classes} in set theory:
  \begin{enumerate}
    \item $*$ : the kind of types 
    \item $* \Rightarrow *$ : the kind of type operators, e.g., 
      $a \mapsto [a]$ 
    \item \ldots 
  \end{enumerate}
  System $F_\omega$ is an extension of System $F$ with type-level functions and
  kinds. See \cite{Pierce2002} for further detail. 
\end{frame}

\begin{frame}[allowframebreaks]{Impredicativity}
  A definition is \alert{\emph{impredicative}} if it has a quantifier whose
  domain includes itself (which is being defined). 
  \begin{block}{Russell's Paradox}
    \[
      R \defeq \set{x}{ x \not\in x }
    \]
  \end{block}
  Recall the self-application ...
  \[
    (\Lambda t.\,\lambda (x: t). x)\;{\color{red}(\forall t.\,t \to t)}\;
    (\Lambda t.\, \lambda (x: t). x) 
  \]
  which is actually impredicative! 
  This form of polymorphism is called \alert{\emph{impredicative polymorphism}}. 
  \begin{block}{Girard's Paradox}
    An encoding of Russell's paradox in extensions of
    System F. 
  \end{block}

  Martin-L\"of's type theory (on which Agda is based) was inconsistent, as it
  included an axiom\footnote{It is only a simplified story...}:
  \[
    \mathsf{Set} : \mathsf{Set}
  \]

  \begin{block}{Inconsistency of Haskell}
    Impredicativity $+$ Injectivity $+$ Type Case Analysis
    $=$ Inconsistency (Russell paradox)
    \footnote{\url{http://okmij.org/ftp/Haskell/impredicativity-bites.html}}
    
  \end{block}
\end{frame}
%\section{Parametricity}
%Polymorphism in System F is \emph{parametric} in the sense that every
%instance is uniformly defined. To put it differently, it cannot depend on any
%specific type. It is different from the \emph{ad hoc} polymorphism adopted in
%programming languages like C++ where programmers are allowed to give a
%specific implementation for some type.
%
%Why is a uniform definition important? Not only we can assure ourselves that a
%polymorphic function has a uniform result for each instance, but also it
%automatically satisfies certain properties which only depends on its type (not
%its instance).
%
%In the end of this lecture, we state a powerful result informally, originally
%called \emph{Abstraction Theorem} by Reynolds~\cite{Reynolds1983} and nowadays
%known as \emph{parametricity}. After that, we apply this theorem to characterise
%some data types introduced so far.
%
%\begin{proposition}
%  Let $\sigma$ be a type without any free variable and $\sigma^+$ the relation
%  on~$\sigma$ lifted from~$\sigma$ (defined below). Then, 
%  \[
%    \vdash F : \sigma
%    \implies (F, F) \in \sigma^+.
%  \]
%\end{proposition}
%
%To decode this proposition, we introduce a new notation: a relation $R_\sigma
%\subset \sigma_1 \times \sigma_2$ is denoted by $R_\sigma\colon \sigma_1
%\not\to\sigma_2$. For each type in System F, we define its lifted relation
%$\sigma^+$
%\begin{definition}
%  For each type formation (construct), we define its corresponding formation of
%  relations 
%  \begin{enumerate}
%    \item For $f_i:\sigma_i \to \tau_i$, $(f_1, f_2) \in R_\sigma \to S_\tau$ if
%      and only if for every $(M_1, M_2) \in R_\Sigma$ we have $(f_1\;M_1,
%      f_2\;M_2) \in S_\tau$.
%  \end{enumerate}
%\end{definition}
%
%
%
%\subsection*{Exercise}


\begin{frame}{References}
\bibliographystyle{amsalpha}
\bibliography{library} 
\end{frame}

\end{document}
