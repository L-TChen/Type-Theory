%! TEX root = lecture2_slide.tex
\title{Lambda Calculus and Types}
\subtitle{Higher-Order Functions}
\begin{document}
\begin{frame}
\maketitle
\end{frame}

\begin{frame}{Type-driven Development}

  \begin{enumerate}
    \item Untyped lambda calculus is as powerful as Turing machine---every
      computable function on natural numbers can be expressed.
    \item It is difficult to see the intention of $\lambda$-terms. 
    \item Let \alert{types} guide you how to design programs. 
  \end{enumerate}

  \begin{description}
    \item[Implicit typing]
      types are reconstructed from $\lambda$-terms 
    \item[Explicit typing] (preferable) 
    $\lambda$-terms are annotated with types.
  \end{description}
\end{frame}

\section{Simply Typed Lambda Calculus}
\subsection{Implicit typing: system with type assignment}
\begin{frame}[allowframebreaks]{Implicit Typing}

For an implicit typing system, types are introduced separately.
\begin{definition}
  Given type variables~$\mathbb{V}$, the set $\mathbb{T}$ of
  \emph{types} is defined by
  \begin{multicols}{2}
    \begin{prooftree}
      \AXC{$B \in \mathbb{V}$}
      \UIC{$B \in \mathbb{T}$}
    \end{prooftree}
    \begin{prooftree}
      \AXC{$\sigma \in \mathbb{T}$}
      \AXC{$\tau   \in \mathbb{T}$}
      \RightLabel{(fun)}
      \BIC{$\sigma \to \tau \in \mathbb{T}$}
    \end{prooftree}
  \end{multicols}
  where $\sigma\to\tau$ is the type of functions from $\sigma$ to $\tau$.
\end{definition}
N.B.\ If $\mathbb{V}$ is empty, then $\mathbb{T}$ is also empty. Why?

In line with the convention for application, 
\[
    \sigma_1 \to \sigma_2 \to \dots \sigma_n \quad  \defeq\quad \sigma_1 \to
    (\sigma_2 \to ( \dots \to (\sigma_{n-1} \to \sigma_n)\dots))
  \]
\begin{definition}
  That \alert{\emph{$M$ has type $\sigma$}} is denoted by
      \[
        M : \sigma
      \]
      Given distinct variables $x_i$'s, a \alert{\emph{typing context}}~$\Gamma$
      is a set 
      \[
        \Gamma = \{x_1 : \sigma_1, x_2 : \sigma_2, \ldots, x_n : \sigma_n\}
      \]
  \end{definition}
  \begin{block}{Convention}
      \[
        \Gamma, x : \sigma\quad\text{stands for}\quad\Gamma\cup \{x : \sigma\}.
      \]
  \end{block}

\begin{definition}[Typing judgement]
  A \alert{\emph{typing judgement}} is a context~$\Gamma$ with $M:\sigma$
  denoted by
  \[
        \Gamma \vdash M : \sigma
  \]
\end{definition}
\begin{definition}[Typing rules]
  \begin{multicols}{2} 
  \begin{prooftree}
    \AXC{\phantom{$\Gamma$}}
    \RightLabel{(var)}
    \UIC{$\Gamma, x : \sigma \vdash x : \sigma$}
  \end{prooftree}
  \begin{prooftree}
    \AXC{$\Gamma \vdash M : (\sigma \to \tau)$}
    \AXC{$\Gamma \vdash N : \sigma$}
    \RightLabel{(app)}
    \BIC{$\Gamma \vdash (M\;N) : \tau$}
  \end{prooftree}
  \begin{prooftree}
    \AXC{$\Gamma, x : \sigma \vdash M : \tau$}
    \RightLabel{(abs)}
    \UIC{$\Gamma \vdash \lambda x.\; M : (\sigma \to \tau)$}
  \end{prooftree}
  \end{multicols}
\end{definition}
\end{frame}

\begin{frame}[allowframebreaks]{Typing Derivation}
  \begin{itemize}
    \item A \alert{\emph{typing derivation}} of a typing judgement $\Gamma
      \vdash M : \sigma$ is a tree of instances of typing rules whose root is
      $\Gamma \vdash M : \sigma$.
    \item A typing judgement is \alert{\emph{derivable}} if there is a typing
      derivation of that judgement. 
  \end{itemize}
\begin{example}
    The judgement $\vdash \lambda x.\, x : \sigma \to \sigma$, for all $\sigma \in
    \mathbb{T}$, can be derived by
  \begin{prooftree}
    \AXC{}
    \RightLabel{(var)}
    \UIC{$x : \sigma \vdash x : \sigma$}
    \RightLabel{(abs)}
    \UIC{$\vdash \lambda x.\; x : (\sigma \to \sigma)$}
  \end{prooftree}
\end{example}
\framebreak
\begin{example}
    The judgement $ \vdash \lambda x\,y.\, x : \sigma \to \tau \to \sigma$
      for all $\sigma, \tau\in\mathbb{T}$, can be derived by
\begin{prooftree}
  \AXC{}
  \RightLabel{(var)}
  \UIC{$x : \sigma, y: \tau \vdash x : \sigma$}
  \RightLabel{(abs)}
  \UIC{$x : \sigma \vdash \lambda y.\, x : \tau \to \sigma$}
  \RightLabel{(app)}
  \UIC{$\vdash \lambda x\,y.\, x : \sigma \to \tau \to \sigma$}
\end{prooftree}
\end{example}


\begin{block}{Exercise}
    Derive $\vdash \lambda f\,g\,x.\, f\,x\, (g\,x) : (\sigma \to \tau \to \rho) \to
    (\sigma\to\tau) \to \sigma\to\rho$ for all $\sigma, \tau, \rho \in
    \mathbb{T}$.
\end{block}

\framebreak

\begin{example}
  Not every $\lambda$-term has a type. For example,
  \[
    \lambda x.\, x\,x
  \]
  does not have a type, since $\sigma \to \sigma$ is not equal to $\sigma$.
\end{example}
\begin{block}{Exercise}
  Describe all possible types for Church numeral $\bc_{n}$.  Hint. You may find
  $\bc_{0}$ and $\bc_{1}$ confusing.
\end{block}
\end{frame}


\subsection{Explicit Typing: System of Typed Terms}
\begin{frame}[allowframebreaks]{Explicit typing}
\begin{definition}[Typed terms]
  The set $\Lambda_\mathbb{T}$ of typed $\lambda$-terms is defined by
  \begin{multicols}{2}
    \begin{prooftree}
      \AXC{$x \in V$}
      \UIC{$x \in \Lambda_\T$}
    \end{prooftree}
    \begin{prooftree}
      \AXC{$M \in \Lambda_\T$}
      \AXC{$N \in \Lambda_\T$}
      \BIC{$(M\, N) \in \Lambda_\T$}
    \end{prooftree}
    \begin{prooftree}
      \AXC{$M \in \Lambda_\T$}
      \AXC{$x \in V$}
      \AXC{$\tau \in \mathbb{T}$}
      \TIC{$\lambda \alert{x:\tau}.\; M \in \Lambda_\T$}
    \end{prooftree}
  \end{multicols}
\end{definition}
Substitution and $\alpha\beta\eta$-conversions are defined in a similar way. 
\framebreak

\begin{definition}[Typing Rules]
  A typing derivation on \emph{typed terms} is defined by 
  \begin{multicols}{2} 
  \begin{prooftree}
    \AXC{}
    \RightLabel{(var)}
    \UIC{$\Gamma, x : \sigma \vdash x : \sigma$}
  \end{prooftree}
  \begin{prooftree}
    \AXC{$\Gamma \vdash M : \sigma \to \tau$}
    \AXC{$\Gamma \vdash N : \sigma$}
    \RightLabel{(app)}
    \BIC{$\Gamma \vdash (M\;N) : \tau$}
  \end{prooftree}
  \begin{prooftree}
    \AXC{$\Gamma, x : \sigma \vdash M : \tau$}
    \RightLabel{(abs)}
    \UIC{$\Gamma \vdash \lambda x:\sigma.\; M : \sigma \to \tau$}
  \end{prooftree}
  \end{multicols}
\end{definition}
A typed $\lambda$-terms~$M$ with a derivable judgement
$\Gamma \vdash M : \sigma$ is \alert{well-typed}. 
\begin{proposition}
  For every typed term $M$, context~$\Gamma$, and types $\sigma, \tau$, 
  \[
    \Gamma \vdash M : \sigma
    \quad\text{and}\quad
    \Gamma \vdash M : \tau
    \implies
    \sigma = \tau
  \]
\end{proposition}
\begin{block}{Exercise}
  Show that $\lambda (x : \sigma).\, x\; x$ is a typed $\lambda$-term but
    there exists no type $\tau$ with 
    \[
      \vdash \lambda (x : \sigma).\, x\; x : \tau
    \]
  \end{block}
\end{frame}

\subsection{Implicit typing versus explicit typing}

\begin{frame}{From Typed Terms to Untyped Terms}
  An \emph{erasing map} $|-|\colon \Lambda_\T \to \Lambda$ is defined by
  \begin{align*}
    |x| & = x \\
    |(M\; N)| & = (|M|\;|N|) \\
    |\lambda (x:\sigma).\, M| & = \lambda x.\, |M|
  \end{align*}
\begin{proposition}\label{prop:type-erasure}
  Let $M$ and $N$ be typed $\lambda$-terms in~$\Lambda_\T$. Then, 
  \begin{align*}
    \Gamma  \vdash M : \sigma & \text{ implies } \Gamma \vdash |M| :
    \sigma \\ 
    M \reduce N & \text{ implies } |M| \reduce |N|
  \end{align*}
\end{proposition}
\end{frame}

\begin{frame}{From Untyped Terms to Typed Terms}
\begin{proposition}\label{prop:type-reconstruction}
  Let $M$ and $N$ be $\lambda$-terms in~$\Lambda$. 
  \begin{enumerate}
    \item If $\Gamma \vdash M : \sigma$, then there is a typed term $M'$ with 
      \[
        |M'| = M
        \quad\text{and}\quad
        \Gamma \vdash M' : \sigma
      \]
    \item If $M \reduce N$ and $M = |M'|$ for some typed $\lambda$-term $M'$,
      then there exists $N'$ with $|N'| = N$ and $M' \reduce N'$.
    \end{enumerate}
\end{proposition}
\begin{block}{Homework}
  Prove Propositions~\ref{prop:type-erasure} and~\ref{prop:type-reconstruction}
  by induction. 
\end{block}
\end{frame}
\begin{frame}{Type Inference}
\begin{description}
  \item[Typability:] Is there a type $\sigma$ with $\vdash M : \sigma$? 
  \item[Type checking:] Does $\vdash M : \sigma$ hold for a given type $\sigma$?
  \end{description}
Both questions are positive.

\begin{theorem}
  Typability and type checking are both \emph{decidable}
  in simply typed $\lambda$-calculus.
\end{theorem}

\subsection*{Exercise}
  
\end{frame}

\section{Properties of Simply Typed Lambda Calculus}
\subsection{Preservation}
\begin{frame}{Type Safety}
``Well-typed programs do not go wrong!''
\begin{theorem}[Preservation Theorem]
  \label{thm:preservation}
  Let $M$ and $N$ be typed $\lambda$-terms. 
  If $M \onereduce N$, then
  \[
    \Gamma \vdash M : \sigma \text{ implies } \Gamma \vdash N : \sigma
  \]
\end{theorem}

``Well-typed programs never get stuck!''
\begin{theorem}[Progress Theorem]
  If ${}\vdash M : \sigma$ for $M \in \Lambda_\T$, then either $M$ is a value or
  there is $N$ with $M \onereduce N$.
\end{theorem}
The evaluation of well-typed programs always produces a value of the same type
if terminates. 
\end{frame}

%\begin{proof}
%  We prove it by induction on both the derivation of $\Gamma \vdash M :
%  \sigma$ and $M \onereduce N$. However, the only non-trivial case is
%  \[
%    \Gamma \vdash (\lambda (x_1 : \tau).\, M_1)\; N : \sigma
%    \quad\text{and}\quad
%    (\lambda (x_1 : \tau).\, M_1)\;N \onereduce M_2\subst{x_2}{N}
%  \]
%  with the induction hypothesis valid for
%  \[
%    \Gamma, x_1 : \tau \vdash M_1 : \sigma
%    \quad\text{and}\quad
%    \Gamma \vdash N : \tau
%  \]
%  where $\lambda x_1:\tau.\,M_1$ and $\lambda x_2:\tau.\,M_2$ are
%  $\alpha$-equivalent. The rest is left as an exercise.
%\end{proof}
\begin{frame}[allowframebreaks]{Converse of Preservation}
The converse of Preservation Theorem does not hold.
\begin{example}
  Recall that 
  \begin{enumerate}
    \item $\mathbf{I} = \lambda x.\, x$
    \item $\mathbf{K}_1 = \lambda x\,y.\, x$
    \item $\Omega = (\lambda x.\, x\,x)\,(\lambda x.\, x\,x)$
  \end{enumerate}
  and $\mathbf{K}_1\,\mathbf{I}\,\Omega \reduce \mathbf{I}$. However, 
  \[
    \vdash \mathbf{I} : \sigma \to \sigma
    \not\implies
    \vdash \mathbf{K}_1\;\mathbf{I}\;\Omega : \sigma \to \sigma.
  \]
\end{example}
\framebreak
To see why the latter judgement fails, we need the following:
\begin{lemma}[Typability of subterms]
  Let $M$ be a well-typed $\lambda$-term and $\sigma$ a type under the
  context~$\Gamma$. Then, for every subterm $M'$ of~$M$ there exists $\Gamma'$
  such that
  \[
    \Gamma' \vdash M' : \sigma'.
  \]
\end{lemma}
\begin{proof}
  By induction on $\Gamma \vdash M : \sigma$.
\end{proof}
  $\Omega$ is not typable, so $\mathbf{K}_1\,\mathbf{I}\,\Omega$ is not typable.
\end{frame}

\begin{frame}[allowframebreaks]{Proving Preservation Theorem}
  To prove type preservation, we need a few lemmas. 

  Let $\mathrm{dom}(\Gamma)$ denote the set of variables which occur in
  $\Gamma$. 
  \begin{lemma}[Weakening]
    If $\Gamma \vdash M : \tau$ and $x \not\in \Gamma$, then 
    $\Gamma, x : \sigma \vdash M : \tau$. 
  \end{lemma}
  \begin{lemma}[Substitution]
    If $\Gamma, x : \tau \vdash M : \sigma$ and $\Gamma \vdash N : \tau$ then
    $\Gamma \vdash M\subst{x}{N} : \sigma$.
    
  \end{lemma}

  \begin{corollary}
    If $\Gamma, x : \tau \vdash M :\sigma$ and $x'$ is not in
    $\mathrm{dom}(\Gamma)$, then $ \Gamma, x' : \tau \vdash M\subst{x}{x'} :
    \sigma$. 
  \end{corollary}
  \framebreak
  \begin{proof}
    Since $x'$ is not in $\Gamma$, it follows that 
    \[
      \Gamma, x' : \tau, x : \tau \vdash M 
    \]
    by Weakening Lemma and also by definition $\Gamma, x' : \tau \vdash x' :
    \tau$.
    Therefore, by Substitution Lemma, it follows that
    \[
      \Gamma, x' : \tau \vdash M\subst{x}{x'} : \sigma
    \]
  \end{proof}
  \begin{block}{Homework}
    Prove Preservation Theorem by induction.  Use Substitution Lemma if
    applicable. 
  \end{block}
\end{frame}

\begin{frame}{Progress Theorem}
  A \emph{value} is a $\lambda$-abstraction. 
\begin{theorem}[Progress Theorem]
  If ${}\vdash M : \sigma$ for $M \in \Lambda_\T$, then either $M$ is a value or
  there is $N$ with $M \onereduce N$.
\end{theorem}
\begin{proof}
  By induction on typing derivations. 
\end{proof}
\end{frame}

\subsection{Strong normalisation}
\begin{frame}{Strong Normalisation}
  
Another property of type system is called \emph{strong normalisation}. It means
that every well-typed terms terminates eventually. 
\begin{theorem}
  Suppose that ${}\vdash M : \sigma$. Then,
  there is no infinite sequence of reductions
  \[
    M \onereduce M_1 \onereduce \dots \onereduce \dots.
  \]
\end{theorem}
\begin{corollary}
  Every well-typed $\lambda$-term ${}\vdash M:\sigma$ has a normal form.
\end{corollary}
\end{frame}

\section{Programming in Simply Typed Lambda Calculus}
\begin{frame}[allowframebreaks]{Church Encodings of Natural Numbers}
The type of natural numbers is of the form
\[
  \nat_\tau \defeq (\tau \to \tau) \to \tau \to \tau
\]
for every type $\tau \in \T$.
  
  \begin{description}
    \item[Church numerals]
      \begin{align*}
        & \bc_n \defeq \lambda f\,x.\,
        f^n x \\
        \vdash{} & \bc_n : \nat_\tau
      \end{align*}
    \item[Successor]
      \begin{align*}
        & \suc \defeq \lambda n \,f\,x\,.\;f\;(n\;f\;x) \\
        \vdash{} & \suc : \nat_\tau \to \nat_\tau
      \end{align*}
    \item[Addition]
      \begin{align*}
        & \add \defeq \lambda n\,m\,f\,x.\; (m\;f)\;(n\;f\;x) \\
        \vdash{} & \add : \nat_\tau \to \nat_\tau \to \nat_\tau
      \end{align*}
    \item[Muliplication] 
      \begin{align*}
        & \mul \defeq \lambda n\,m\,f\,x.\, (m\;(n\;f))\;x\\
      \vdash{} & \mul : \nat_\tau \to \nat_\tau \to \nat_\tau
      \end{align*}
    \item[Conditional]
      \begin{align*}
        & \ifz \defeq \lambda n\,x\,y.\, n\;(\lambda z.\, x)\;y\\
        \vdash {} & \ifz : ?
      \end{align*}
  \end{description}
The type of~$\ifz$ may not be as obvious as you may expect.
Try to find one as general as possible and justify your guess.

\end{frame}
\begin{frame}{Church Encodings of Boolean values}
We can also define the type of Boolean values 
for each type variable as
\[
  \bool_\tau \defeq \tau \to \tau \to \tau
\]
\begin{description}
  \item[Boolean values]
      \[
        \true \defeq \lambda x\,y.\,x 
        \quad\text{and}\quad
        \false \defeq \lambda x\,y.\,y 
      \]
  \item[Conditional]
    \begin{align*}
      & \cond \defeq \lambda b\,x\,y.\, b\,x\,y\\
      \vdash {} & \cond : \bool_{\tau} \to \tau \to \tau \to \tau
    \end{align*}
\end{description}
\begin{block}{Exercise}
  Define conjunction $\mathtt{and}$, disjunction $\mathtt{or}$, and negation
  $\mathtt{not}$ in simply typed lambda calculus.
\end{block}
\end{frame}

\begin{frame}[allowframebreaks]{Definability}
%Within simply typed lambda calculus, we wonder how many computable functions can
%be defined, as it excludes non-terminating $\lambda$-terms. 
  A function $f\colon \mathbb{N}^k \to \mathbb{N}$ is called
  \alert{\emph{$\lambda_\to$-definable}} if there is a $\lambda$-term $F$ of
  type $\nat \to \nat \to \dots \nat \to \nat$ such that
  \[
    F\,\bc_{n_1}\ldots\bc_{n_k} \reduce \bc_{f(n_1, \dots, n_k)}
  \]
  for every sequence $(n_1, n_2, \ldots, n_k) \in \mathbb{N}^k$.
  Diagrammatically, 

\[
  \xymatrix{
    (n_1, n_2, \ldots, n_k) \ar@{|->}[rr] \ar@{|->}[d]_{(\bc_{-})^k} & & f(n_1,
    n_2, \ldots, n_k) \ar@{|->}[d]^{\bc_{-}}\\
    (\bc_{n_1}, \bc_{n_2}, \ldots, \bc_{n_k}) \ar@{|->}[rr] & & 
    F\;\bc_{n_1}\; \bc_{n_2}\; \ldots\;\bc_{n_k}
    = \bc_{f(n_1, n_2, \ldots, n_k)}
  }
\]
\begin{theorem}[Schwichtenberg~\cite{Schwichtenberg1976}]
  The $\lambda_\to$-definable functions are the class of functions
  of the form $f\colon \mathbb{N}^k \to \mathbb{N}$ closed under compositions
  and containing the constant functions, projections, additions, multiplications,
  and the conditional 
  \[
    \mathrm{ifz}(n_0, n_1, n_2) = 
    \begin{cases}
      n_1 & \text{if } n_0 = 0\\
      n_2 & \text{otherwise.}
    \end{cases}
  \]
\end{theorem}
\end{frame}

%\subsection{System T}
%It is also convenient to add some primitive types to our typed lambda calculus (or
%so-called built-in types in programming languages used in practice) as well as
%\emph{primitive terms}. For example, for a system with natural numbers, we
%include numerals in the generation of syntax:
%  \begin{multicols}{2}
%    \begin{prooftree}
%      \AXC{$x \in V$}
%      \UIC{$x \in \Lambda_{\T, \nat}$}
%    \end{prooftree}
%    \begin{prooftree}
%      \AXC{\phantom{$n \in \mathbb{N}$}}
%      \UIC{$\zero\in \Lambda_{\T, \nat}$}
%    \end{prooftree}
%    \begin{prooftree}
%      \AXC{$M \in \Lambda_{\T, \nat}$}
%      \UIC{$\suc\;M\in \Lambda_{\T, \nat}$}
%    \end{prooftree}
%    \begin{prooftree}
%      \AXC{$M \in \Lambda_{\T,\nat}$}
%      \AXC{$N \in \Lambda_{\T,\nat}$}
%      \BIC{$(M\, N) \in \Lambda_{\T, \nat}$}
%    \end{prooftree}
%    \begin{prooftree}
%      \AXC{$M \in \Lambda_{\T, \nat}$}
%      \AXC{$x \in V$}
%      \AXC{$\tau \in \mathbb{T}$}
%      \TIC{$\lambda (x:\tau).\; M \in \Lambda_{\T, \nat}$}
%    \end{prooftree}
%  \end{multicols}
%\noindent and let $\mathbb{G} \defeq \{\nat\}$ with an addition set of typing
%rules:
%\begin{multicols}{2}
%\begin{prooftree}
%  \AXC{\phantom{$\Gamma$}}
%  \UIC{$\Gamma \vdash \zero : \nat$}
%\end{prooftree}
%\begin{prooftree}
%  \AXC{$\Gamma \vdash M : \nat$}
%  \UIC{$\Gamma \vdash \suc\:M : \nat$}
%\end{prooftree}
%\end{multicols}
%\noindent In this way, we can derive that every $\underline{n} \defeq
%\suc^n\;\zero$ has the type of $\nat$:
%\[
%  \Gamma \vdash \underline{n} : \nat
%\]
%
%However, it still remains to add primitive operations such as addition,
%multiplication, division, and so on with a proper set of reduction rules. As we
%have seen that recursions on natural numbers in untyped lambda calculus can be
%defined with the fixpoint operator. Something similar can be done:
%\begin{definition}
%  In addition to typing rules introduced so far, we add the following
%  \begin{prooftree}
%    \AXC{$\Gamma \vdash M : \nat$}
%    \AXC{$\Gamma \vdash M_0 : \sigma$}
%    \AXC{$\Gamma \vdash F : \nat \to (\sigma \to \sigma)$}
%    \TIC{$\Gamma \vdash \fold\;M_0\;F : \nat \to \sigma $}
%  \end{prooftree}
%  with additional reductions
%  \[
%    \fold\;M_0\;F\;\zero
%    \xrightarrow{\beta}
%    M_0
%    \quad\text{and}\quad
%    \fold\;M_0\;F\;(\suc\;M)
%    \xrightarrow{\beta}
%    F\;M\;(\fold\;F\;M).
%  \]
%\end{definition}
%\begin{definition}
%  \todo[inline]{define addition, multiplication using $\fold$}
%\end{definition}
\begin{frame}{Remark}
With the decidability of type checking, Preservation Theorem, Progress Theorem
(a well-typed term is either a ``value'' or a reducible term), and the strong
normalisation, we actually have exhibited a decidable evaluator of simply typed
lambda calculus that always reduce a well-typed term of type~$\sigma$ to a value
of type~$\sigma$.
\end{frame}

\begin{frame}[allowframebreaks]{References}

\bibliographystyle{amsalpha}
\bibliography{library} 

\end{frame}

\end{document}
