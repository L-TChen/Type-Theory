%! TEX root = lecture2_note.tex
\title{Lambda Calculus and Types, 2 \\<all>
  Simply Typed Lambda Calculus}
\begin{document}
\begin{frame}
\maketitle
\end{frame}

\section{Simply typed lambda calculus}
As we have seen in the first lecture, $\lambda$-terms can ``call'' themselves,
projections $\mathbf{K}_1$ and $\mathbf{K}_2$ can also be viewed as Boolean
values, and so on. Despite its powerfulness, the expressiveness can be a
disaster if treated as a programming language, because it is difficult to see the
intention of such terms. 

This motivates us to introduce \emph{types} over $\lambda$-terms. There are two
approaches given by Curry and Church corresponding to
two paradigms in programming language called \emph{implicit} and \emph{explicit
  typing} respectively.  In the typed lambda calculus \textit{\`a la} Curry, a
$\lambda$-term is assigned to all possible types; while in the calculus
\textit{\`a la} Church, a $\lambda$-term is annotated with types. 

\subsection{Implicit typing: system with type assignment}
For an implicit typing system, types are defined in addition to $\lambda$-terms.
We begin a minimalistic typing system consisting of function types with primitive
types (such as types of Boolean values and natural numbers). We call it
\emph{simply typed lambda calculus \textit{\`a la} Curry}.
\begin{definition}
  Given a set of type variables~$\mathbb{V}$ and a set $\mathbb{G}$ of ground
  types (a.k.a.\ primitive types), the set $\mathbb{T} = \mathbb{T}_\mathbb{G}$
  of \emph{types} is inductively defined as
  \begin{multicols}{3}
    \begin{prooftree}
      \AXC{$B \in \mathbb{V}$}
      \UIC{$B \in \mathbb{T}$}
    \end{prooftree}
    \begin{prooftree}
      \AXC{$B \in \mathbb{G}$}
      \UIC{$B \in \mathbb{T}$}
    \end{prooftree}
    \begin{prooftree}
      \AXC{$\sigma \in \mathbb{T}$}
      \AXC{$\tau   \in \mathbb{T}$}
      \RightLabel{(fun)}
      \BIC{$\sigma \to \tau \in \mathbb{T}$}
    \end{prooftree}
  \end{multicols}
\end{definition}
\begin{definition}
  Given a set of variables and a set of ground types, we define the following
  notions.
  \begin{enumerate}
    \item A \emph{statement} consists of a $\lambda$-term~$M$
      and a type~$\sigma \in \mathbb{T}$ denoted by 
      \[
        M : \sigma
      \]
      and we say that $M$ has the type $\sigma$.
    \item A \emph{context}~$\Gamma$ is a (possibly infinite) set of statements 
      \[
        \Gamma = \{x_1 : \sigma_1, x_2 : \sigma_2, \ldots, x_n : \sigma_n\}
      \]
      where $x_i$ are all distinct variables. 
  \end{enumerate}
\end{definition}
A judgement in the system of assignment is constructed inductively as follows:
\begin{definition}[Typing rules]
  A \emph{judgement} consists of a context~$\Gamma$ and a statement~$M:\sigma$ denoted by
      \[
        \Gamma \vdash M : \sigma
      \]
  indicating that $M : \sigma$ is derivable from $\Gamma$. Judgements in the
  system of type assignment are constructed inductively as follows:
  \begin{multicols}{2} 
  \begin{prooftree}
    \AXC{$(x : \sigma) \in \Gamma$}
    \RightLabel{(var)}
    \UIC{$\Gamma \vdash x : \sigma$}
  \end{prooftree}
  \begin{prooftree}
    \AXC{$\Gamma \vdash M : (\sigma \to \tau)$}
    \AXC{$\Gamma \vdash N : \sigma$}
    \RightLabel{(app)}
    \BIC{$\Gamma \vdash (M\;N) : \tau$}
  \end{prooftree}
  \begin{prooftree}
    \AXC{$\Gamma, x : \sigma \vdash M : \tau$}
    \RightLabel{(abs)}
    \UIC{$\Gamma \vdash \lambda x.\; M : (\sigma \to \tau)$}
  \end{prooftree}
  \end{multicols}
\end{definition}

\begin{example}
  The following judgements are derivable:
  \begin{enumerate}
    \item $\vdash \lambda x.\, x : \sigma \to \sigma$, for all $\sigma \in
      \mathbb{T}$;
    \item $ \vdash \lambda x.\, (\lambda y.\, x) : \sigma \to (\tau \to \sigma)$
      for all $\sigma, \tau\in\mathbb{T}$;
    \item $\vdash \lambda f.\,\lambda g.\, \lambda x.\, (f\,x)\, (g\,y) 
      : (\sigma \to \tau \to \rho) \to (\sigma\to\tau) \to \sigma\to\rho$
      for all $\sigma, \tau, \rho \in \mathbb{T}$.
  \end{enumerate}
\end{example}
To see why the first judgement is derivable, we apply typing rules
step by step:
\begin{prooftree}
  \AXC{$x : \sigma \in \Gamma, x : \sigma$}
  \RightLabel{(var)}
  \UIC{$\Gamma, x : \sigma \vdash x : \sigma$}
  \RightLabel{(abs)}
  \UIC{$\Gamma \vdash \lambda x.\; x : (\sigma \to \sigma)$}
\end{prooftree}
And, also the second one:
\begin{prooftree}
  \AXC{$x : \sigma \in x : \sigma, y:\tau$}
  \RightLabel{(var)}
  \UIC{$x : \sigma, y: \tau \vdash x : \sigma$}
  \RightLabel{(abs)}
  \UIC{$x : \sigma \vdash \lambda y.\, x : \tau \to \sigma$}
  \RightLabel{(app)}
  \UIC{$\vdash \lambda x.\, (\lambda y.\, x) : \sigma \to (\tau \to \sigma)$}
\end{prooftree}
The third one is left as an exercise to the reader.

\begin{example}
  Not every $\lambda$-term has a type. For example,
  \[
    \lambda x.\, x\,x
  \]
  does not have a type, since $\sigma \to \sigma$ is not equal to $\sigma$.
\end{example}


\subsection{Explicit typing: system of typed terms}
Instead of having an external typing system, it is also possible to internalise
the typing information as part of $\lambda$-terms. This system is called
\emph{simply typed lambda calculus \textit{\`a la} Church}.
\begin{definition}
  The set $\Lambda_\mathbb{T}$ of typed $\lambda$-terms is inductively defined
  \begin{multicols}{2}
    \begin{prooftree}
      \AXC{$x \in V$}
      \UIC{$x \in \Lambda_\T$}
    \end{prooftree}
    \begin{prooftree}
      \AXC{$M \in \Lambda_\T$}
      \AXC{$N \in \Lambda_\T$}
      \BIC{$(M\, N) \in \Lambda_\T$}
    \end{prooftree}
    \begin{prooftree}
      \AXC{$M \in \Lambda_\T$}
      \AXC{$x \in V$}
      \AXC{$\tau \in \mathbb{T}$}
      \TIC{$\lambda (x:\tau).\; M \in \Lambda_\T$}
    \end{prooftree}
  \end{multicols}
\end{definition}
Likewise, the \emph{substitution} and the \emph{one-step $\beta$-reduction} are
defined on typed $\lambda$-terms in the same way.
\begin{definition}[Typing rules]
  A judgement $\Gamma \vdash M : \sigma$ on typed terms is constructed
  inductively by the following rules.
  \begin{multicols}{2} 
  \begin{prooftree}
    \AXC{$(x : \sigma) \in \Gamma$}
    \RightLabel{(var)}
    \UIC{$\Gamma \vdash x : \sigma$}
  \end{prooftree}
  \begin{prooftree}
    \AXC{$\Gamma \vdash M : (\sigma \to \tau)$}
    \AXC{$\Gamma \vdash N : \sigma$}
    \RightLabel{(app)}
    \BIC{$\Gamma \vdash (M\;N) : \tau$}
  \end{prooftree}
  \begin{prooftree}
    \AXC{$\Gamma, x : \sigma \vdash M : \tau$}
    \RightLabel{(abs)}
    \UIC{$\Gamma \vdash \lambda (x:\sigma).\; M : (\sigma \to \tau)$}
  \end{prooftree}
  \end{multicols}
\end{definition}
Unlike implicit typing, every well-typed term with explicit typing has 
a unique type:
\begin{proposition}
  For every typed term $M$, context~$\Gamma$, and types $\sigma, \tau$, 
  \[
    \Gamma \vdash M : \sigma
    \quad\text{and}\quad
    \Gamma \vdash M : \tau
    \implies
    \sigma = \tau
  \]
\end{proposition}

\subsection{Implicit typing versus explicit typing}
The typing information can be erased from typed terms and translated
from typed $\lambda$-terms to $\lambda$-terms with implicit type.
\begin{definition}
  An \emph{erasing map} $|-|\colon \Lambda_\T \to \Lambda$ is defined
  inductively:
  \begin{align*}
    |x| & \defeq x \\
    |(M\; N)| & \defeq (|M|\;|N|) \\
    |\lambda (x:\sigma).\, M| & \defeq \lambda x.\, |M|
  \end{align*}
\end{definition}
\begin{proposition}
  Let $M$ and $N$ be typed $\lambda$-terms in~$\Lambda_\T$. Then,
  \begin{enumerate}
    \item 
      \[
        \Gamma \vdash M : \sigma \implies
        \Gamma \vdash |M| : \sigma
      \]
    \item 
      \[
        M \reduce N 
        \implies
        |M| \reduce |N|
      \]
  \end{enumerate}
\end{proposition}
\begin{proposition}
  Let $M$ and $N$ be $\lambda$-terms in~$\Lambda$. 
  \begin{enumerate}
    \item If $\Gamma \vdash M : \sigma$, then there is a typed term $M'$ with 
      \[
        |M'| = M
        \quad\text{and}\quad
        \Gamma \vdash M' : \sigma
      \]
    \item If $M \reduce N$ and $M = |M'|$ for some typed $\lambda$-term $M'$,
      then there exists $N'$ with $|N'| = N$ and $M' \reduce N'$.
    \end{enumerate}
\end{proposition}

In practice, the type inference is left to the evaluator of a programming
language, so it is natural to ask if this system of assignment has the following
properties:
\begin{description}
  \item[Typability:] Is there a type $\sigma$ with $\vdash M : \sigma$? 
  \item[Type checking:] Does $\vdash M : \sigma$ hold for a given type $\sigma$?
  \end{description}
Both questions are answered positively:

\begin{theorem}
  Typability and type checking are both \emph{decidable}.
\end{theorem}
%See \cite[Section 3.2]{Sorensen2006} for further detail.

\subsection*{Exercise}
\begin{enumerate}
  \item Show that $\Omega$ does not have a type.
  \item Show that $\vdash \mathbf{K}_1\;\mathbf{I}\;\Omega : \sigma \to \sigma$
    does not have a type.
  \item Show that $\lambda (x : \sigma).\, x\; x$ is a typed $\lambda$-term but
    there exists no type $\tau$ with 
    \[
      \vdash \lambda (x : \sigma).\, x\; x : \tau.
    \]
  \end{enumerate}
\section{Type safety}
\subsection{Preservation}
One of the main features of typed programs is that that every well-typed terms
always reduces to a term of the same type---``well-typed programs do not go
wrong!''.
\begin{theorem}[Preservation Theorem]
  \label{thm:preservation}
  Let $M$ and $N$ be typed $\lambda$-terms. 
  If $M \onereduce N$, then
  \[
    \Gamma \vdash M : \sigma \implies \Gamma \vdash N : \sigma
  \]
\end{theorem}
\begin{proof}
  We prove it by induction on both the derivation of $\Gamma \vdash M :
  \sigma$ and $M \onereduce N$. However, the only non-trivial case is
  \[
    \Gamma \vdash (\lambda (x : \tau).\, M_1)\; N : \sigma
    \quad\text{and}\quad
    (\lambda (x_1 : \tau).\, M_1)\;N \onereduce M_2\subst{x_2}{N}
  \]
  with the hypothesis assumption
  \[
    \Gamma, x : \tau \vdash M_1 : \sigma
    \quad\text{and}\quad
    \Gamma \vdash N : \tau
  \]
  where $\lambda x_1:\tau.\,M_1$ and $\lambda x_2:\tau.\,M_2$ are
  $\alpha$-equivalent. The rest is left as an exercise.
\end{proof}
However, the converse may not hold. Observe the following examples
\begin{example}
  Recall that 
  \begin{enumerate}
    \item $\mathbf{I} = \lambda x.\, x$
    \item $\mathbf{K}_1 = \lambda x.\, \lambda y.\, x$
    \item $\Omega = (\lambda x.\, x\,x)\,(\lambda x.\, x\,x)$
  \end{enumerate}
  and $\mathbf{K}_1\,\mathbf{I}\,\Omega \reduce \mathbf{I}$. However, 
  \[
    \vdash \mathbf{I} : \sigma \to \sigma
    \not\implies
    \vdash \mathbf{K}_1\;\mathbf{I}\;\Omega : \sigma \to \sigma.
  \]
\end{example}
To see why the latter judgement fails, we need the following lemma:
\begin{lemma}[Typability of subterms]
  Let $M$ be a  $\lambda$-term and $\sigma$ a type. Then,  
  \[
    \Gamma \vdash M : \sigma 
    \implies \Gamma' \vdash M' : \sigma'
  \]
  for every subterm $M'$ of~$M$.
\end{lemma}
\begin{proof}
  By induction on $\Gamma \vdash M : \sigma$.
\end{proof}
\subsection{Progress}
Another feature of well-typed terms is that a well-typed \emph{closed} term is
either a \emph{value} or reducible---``well-typed programs do not get stuck!''

\begin{theorem}[Progress Theorem]
  If ${}\vdash M : \sigma$ for a typed $\lambda$-term~$M$, then either $M$ is in
  a value or there is $M'$ with $M \onereduce M'$.
\end{theorem}
\todo[inline]{define value?}

\subsection{Strong normalisation}
The last property of type system in this lecture is called \emph{strong
  normalisation}. It means that every well-typed terms terminates eventually. 
\begin{theorem}
  Suppose that ${}\vdash M : \sigma$. Then,
  there is no infinite sequence of reductions
  \[
    M \onereduce M_1 \onereduce \dots \onereduce \dots.
  \]
\end{theorem}
\begin{corollary}
  Every well-typed $\lambda$-term ${}\vdash M:\sigma$ has a normal form.
\end{corollary}
With the decidability of type checking, Preservation Theorem, Progress Theorem,
and the strong normalisation, we actually have shown that there exists a
decidable evaluator of simply typed lambda calculus that always reduce a
well-typed term of type~$\sigma$ to a value of type~$\sigma$. In fact, since
statements are proved by induction, we can implement an evaluator naively from
the proofs. In fact, we will see later how to extract a program from a
\emph{formal} proof.
\subsection*{Exercise}
\begin{enumerate}
  \item (*)Show that if $\Gamma, x : \tau \vdash M : \sigma$
    and $\Gamma \vdash N : \tau$ then $\Gamma \vdash M\subst{x}{N} : \sigma$.
    Hint: Recall the definition of substitution and prove by induction
    on $\Gamma, x:\tau\vdash M : \sigma$.
  \item Show that $\Gamma, x : \tau \vdash M :\sigma
    \iff \Gamma', x' : \tau \vdash M\subst{x}{x'} : \sigma$. That is,
    $\alpha$-equivalent (typed) terms have the same type.
  \item Finish the proof of Theorem~\ref{thm:preservation}.
  \item Prove that Preservation Theorem and Progress Theorem hold for
    simply lambda calculus \textit{\`a la} Curry using type erasing.
\end{enumerate}
\section{Programming in simply typed lambda calculus}
It is also convenient to add some primitive types our typed lambda calculus (or
so-called built-in types in programming languages used in practice) as well as
\emph{primitive terms}. For example, for a system with natural numbers, we
include numerals in the generation of syntax:
  \begin{multicols}{2}
    \begin{prooftree}
      \AXC{$x \in V$}
      \UIC{$x \in \Lambda_{\T, \nat}$}
    \end{prooftree}
    \begin{prooftree}
      \AXC{\phantom{$n \in \mathbb{N}$}}
      \UIC{$\zero\in \Lambda_{\T, \nat}$}
    \end{prooftree}
    \begin{prooftree}
      \AXC{$M \in \Lambda_{\T, \nat}$}
      \UIC{$\suc\;M\in \Lambda_{\T, \nat}$}
    \end{prooftree}
    \begin{prooftree}
      \AXC{$M \in \Lambda_{\T,\nat}$}
      \AXC{$N \in \Lambda_{\T,\nat}$}
      \BIC{$(M\, N) \in \Lambda_{\T, \nat}$}
    \end{prooftree}
    \begin{prooftree}
      \AXC{$M \in \Lambda_{\T, \nat}$}
      \AXC{$x \in V$}
      \AXC{$\tau \in \mathbb{T}$}
      \TIC{$\lambda (x:\tau).\; M \in \Lambda_{\T, \nat}$}
    \end{prooftree}
  \end{multicols}
\noindent and let $\mathbb{G} \defeq \{\nat\}$ with an addition set of typing
rules:
\begin{multicols}{2}
\begin{prooftree}
  \AXC{\phantom{$\Gamma$}}
  \UIC{$\Gamma \vdash \zero : \nat$}
\end{prooftree}
\begin{prooftree}
  \AXC{$\Gamma \vdash M : \nat$}
  \UIC{$\Gamma \vdash \suc\:M : \nat$}
\end{prooftree}
\end{multicols}
\noindent In this way, we can derive that every $\underline{n} \defeq
\suc^n\;\zero$ has the type of $\nat$:
\[
  \Gamma \vdash \underline{n} : \nat
\]

However, it still remains to add primitive operations such as addition,
multiplication, division, and so on with a proper set of reduction rules. As we
have seen that Church numerals and arithmetic operations in untyped lambda
calculus can be defined with recursion. Something similar can be done:
\begin{definition}
  In addition to typing rules introduced so far, we add the following
  \begin{prooftree}
    \AXC{$\Gamma \vdash M : \nat$}
    \AXC{$\Gamma \vdash M_0 : \sigma$}
    \AXC{$\Gamma \vdash F : \nat \to (\sigma \to \sigma)$}
    \TIC{$\Gamma \vdash \fold\;M_0\;F : \nat \to \sigma $}
  \end{prooftree}
  with additional reductions
  \[
    \fold\;M_0\;F\;\zero
    \xrightarrow{\beta}
    M_0
    \quad\text{and}\quad
    \fold\;M_0\;F\;(\suc\;M)
    \xrightarrow{\beta}
    F\;M\;(\fold\;F\;M).
  \]
\end{definition}
\begin{definition}
  \todo[inline]{define addition, multiplication using $\fold$}
\end{definition}
\subsection*{Exercise}
\begin{enumerate}
  \item Define the conditional construct with $\fold$, i.e.\
    \[
      \Gamma \vdash \mathtt{if\_then\_else} : \nat \to (\tau \to \tau)
    \]
    for every $\tau \in \mathbb{T}$ and show that it reduces to the desired
    term.
  \item Define the factorial of natural number with $\fold$.
\end{enumerate}

\bibliographystyle{plain}
\bibliography{../../library} 
\end{document}
