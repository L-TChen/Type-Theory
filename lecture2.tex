%! TEX root = lecture2_note.tex
\title{Lambda Calculus and Types, 2}
\subtitle{Simply Typed Lambda Calculus}
\begin{document}
\begin{frame}
\maketitle
\end{frame}

\section{Simply typed lambda calculus}
As we have mentioned in the first lecture, untyped lambda calculus is as powerful as
a computer such that every computable function on natural numbers can be
expressed. However, its powerfulness can be a disaster if treated as a
programming language, because it is difficult to see the intention of such
terms. 

This motivates us to introduce certain descriptions, or called \emph{types}, over
$\lambda$-terms. There are two approaches given by Curry and Church
corresponding to two paradigms in programming language called \emph{implicit}
and \emph{explicit typing} respectively.  In the typed lambda calculus
\textit{\`a la} Curry, a $\lambda$-term is assigned to all possible types; in
the calculus \textit{\`a la} Church, a $\lambda$-term is annotated with types. 

\subsection{Implicit typing: system with type assignment}
For an implicit typing system, types are defined in addition to $\lambda$-terms.
We begin with a minimalistic typing system consisting of function types. We call it
\emph{simply typed lambda calculus \textit{\`a la} Curry}.
\begin{definition}
  Given type variables~$\mathbb{V}$, the set $\mathbb{T}$ of
  \emph{types} is inductively defined by
  \begin{multicols}{2}
    \begin{prooftree}
      \AXC{$B \in \mathbb{V}$}
      \UIC{$B \in \mathbb{T}$}
    \end{prooftree}
    \begin{prooftree}
      \AXC{$\sigma \in \mathbb{T}$}
      \AXC{$\tau   \in \mathbb{T}$}
      \RightLabel{(fun)}
      \BIC{$\sigma \to \tau \in \mathbb{T}$}
    \end{prooftree}
  \end{multicols}
  where $\sigma\to\tau$ is understood as the function type.
\end{definition}
\begin{convention}
  In line with the convention on applications
  \begin{align*}
    M_1\;M_2 \dots M_n \quad &\defeq\quad
    (\dots ((M_1\;M_2)\;M_3) \dots)\; M_n \\
    \text{we adopt the } & \text{following abbreviation} \\
    \sigma_1 \to \sigma_2 \to \dots \sigma_n
    \quad & \defeq\quad \sigma_1 \to (\sigma_2 \to (
    \dots \to (\sigma_{n-1} \to \sigma_n)\dots))
  \end{align*}
\end{convention}
\begin{definition}
    A \emph{statement} consists of a $\lambda$-term~$M$ and a type~$\sigma \in
    \mathbb{T}$ denoted by 
      \[
        M : \sigma
      \]
    and we say that $M$ has the type $\sigma$.  A \emph{context}~$\Gamma$ is a
    set of statements 
      \[
        \Gamma = \{x_1 : \sigma_1, x_2 : \sigma_2, \ldots, x_n : \sigma_n\}
      \]
      where $x_i$ are all distinct variables. It is convenient to write 
      \[
        \Gamma, x : \sigma\quad\text{for}\quad\Gamma\cup \{x : \sigma\}.
      \]
\end{definition}
\begin{definition}[Typing rules]
  A \emph{judgement} consists of a context~$\Gamma$ and a statement~$M:\sigma$
  denoted by
  \[
        \Gamma \vdash M : \sigma
      \]
  which indicates that $M : \sigma$ is derivable from $\Gamma$. In the case that
  $\Gamma = \emptyset$, we write $\vdash M : \sigma$ only. Judgements in the
  system of type assignment are constructed inductively:
  \begin{multicols}{2} 
  \begin{prooftree}
    \AXC{}
    \RightLabel{(var)}
    \UIC{$\Gamma, x : \sigma \vdash x : \sigma$}
  \end{prooftree}
  \begin{prooftree}
    \AXC{$\Gamma \vdash M : (\sigma \to \tau)$}
    \AXC{$\Gamma \vdash N : \sigma$}
    \RightLabel{(app)}
    \BIC{$\Gamma \vdash (M\;N) : \tau$}
  \end{prooftree}
  \begin{prooftree}
    \AXC{$\Gamma, x : \sigma \vdash M : \tau$}
    \RightLabel{(abs)}
    \UIC{$\Gamma \vdash \lambda x.\; M : (\sigma \to \tau)$}
  \end{prooftree}
  \end{multicols}
\end{definition}

A term~$M$ of type $\sigma \to \tau$ is understood as a \emph{function} 
from the domain $\sigma$ to the codomain $\tau$.
\begin{example}
  The following judgements are derivable:
  \begin{enumerate}
    \item $\vdash \lambda x.\, x : \sigma \to \sigma$, for all $\sigma \in
      \mathbb{T}$;
    \item $ \vdash \lambda x\,y.\, x : \sigma \to \tau \to \sigma$
      for all $\sigma, \tau\in\mathbb{T}$;
    \item $\vdash \lambda f\,g\,x.\, f\,x\, (g\,x)
      : (\sigma \to \tau \to \rho) \to (\sigma\to\tau) \to \sigma\to\rho$
      for all $\sigma, \tau, \rho \in \mathbb{T}$.
  \end{enumerate}
\end{example}
To see why the first judgement is derivable, we apply typing rules
step by step:
\begin{prooftree}
  \AXC{}
  \RightLabel{(var)}
  \UIC{$x : \sigma \vdash x : \sigma$}
  \RightLabel{(abs)}
  \UIC{$\vdash \lambda x.\; x : (\sigma \to \sigma)$}
\end{prooftree}
And, also the second one:
\begin{prooftree}
  \AXC{}
  \RightLabel{(var)}
  \UIC{$x : \sigma, y: \tau \vdash x : \sigma$}
  \RightLabel{(abs)}
  \UIC{$x : \sigma \vdash \lambda y.\, x : \tau \to \sigma$}
  \RightLabel{(app)}
  \UIC{$\vdash \lambda x\,y.\, x : \sigma \to \tau \to \sigma$}
\end{prooftree}
The third one is left as an exercise to the reader.

\begin{example}
  Not every $\lambda$-term has a type. For example,
  \[
    \lambda x.\, x\,x
  \]
  does not have a type, since $\sigma \to \sigma$ is not equal to $\sigma$.
\end{example}


\subsection{Explicit typing: system of typed terms}
Instead of having an external typing system, it is also possible to internalise
the typing information as part of $\lambda$-terms. This system is called
\emph{simply typed lambda calculus \textit{\`a la} Church}, denoted by
$\lambda_\to$.
\begin{definition}
  The set $\Lambda_\mathbb{T}$ of typed $\lambda$-terms is inductively defined
  \begin{multicols}{2}
    \begin{prooftree}
      \AXC{$x \in V$}
      \UIC{$x \in \Lambda_\T$}
    \end{prooftree}
    \begin{prooftree}
      \AXC{$M \in \Lambda_\T$}
      \AXC{$N \in \Lambda_\T$}
      \BIC{$(M\, N) \in \Lambda_\T$}
    \end{prooftree}
    \begin{prooftree}
      \AXC{$M \in \Lambda_\T$}
      \AXC{$x \in V$}
      \AXC{$\tau \in \mathbb{T}$}
      \TIC{$\lambda (x:\tau).\; M \in \Lambda_\T$}
    \end{prooftree}
  \end{multicols}
\end{definition}
Likewise, the \emph{substitution} and the \emph{one-step $\beta$-reduction} are
defined on typed $\lambda$-terms in the same way. 
\begin{definition}[Typing rules]
  A judgement $\Gamma \vdash M : \sigma$ on typed terms is constructed
  inductively by the following rules.
  \begin{multicols}{2} 
  \begin{prooftree}
    \AXC{}
    \RightLabel{(var)}
    \UIC{$\Gamma, x : \sigma \vdash x : \sigma$}
  \end{prooftree}
  \begin{prooftree}
    \AXC{$\Gamma \vdash M : (\sigma \to \tau)$}
    \AXC{$\Gamma \vdash N : \sigma$}
    \RightLabel{(app)}
    \BIC{$\Gamma \vdash (M\;N) : \tau$}
  \end{prooftree}
  \begin{prooftree}
    \AXC{$\Gamma, x : \sigma \vdash M : \tau$}
    \RightLabel{(abs)}
    \UIC{$\Gamma \vdash \lambda (x:\sigma).\; M : (\sigma \to \tau)$}
  \end{prooftree}
  \end{multicols}
\end{definition}
A typed $\lambda$-terms~$M$ with a judgement $\Gamma \vdash M : \sigma$ is
called \emph{well-typed}.  Unlike implicit typing, every well-typed term with
explicit typing has a unique type:
\begin{proposition}
  For every typed term $M$, context~$\Gamma$, and types $\sigma, \tau$, 
  \[
    \Gamma \vdash M : \sigma
    \quad\text{and}\quad
    \Gamma \vdash M : \tau
    \implies
    \sigma = \tau
  \]
\end{proposition}

\subsection{Implicit typing versus explicit typing}
The typing information can be erased from typed terms and translated
from typed $\lambda$-terms to $\lambda$-terms with implicit type.
\begin{definition}
  An \emph{erasing map} $|-|\colon \Lambda_\T \to \Lambda$ is defined
  inductively:
  \begin{align*}
    |x| & \defeq x \\
    |(M\; N)| & \defeq (|M|\;|N|) \\
    |\lambda (x:\sigma).\, M| & \defeq \lambda x.\, |M|
  \end{align*}
\end{definition}
\begin{proposition}
  Let $M$ and $N$ be typed $\lambda$-terms in~$\Lambda_\T$. Then, 
  \[
    \left(\Gamma \vphantom{\reduce} \vdash M : \sigma \implies \Gamma \vdash |M| : \sigma\right)
    \quad\text{and}\quad \left(M \reduce N \implies |M| \reduce |N|\right).
  \]
\end{proposition}
\begin{proposition}
  Let $M$ and $N$ be $\lambda$-terms in~$\Lambda$. 
  \begin{enumerate}
    \item If $\Gamma \vdash M : \sigma$, then there is a typed term $M'$ with 
      \[
        |M'| = M
        \quad\text{and}\quad
        \Gamma \vdash M' : \sigma
      \]
    \item If $M \reduce N$ and $M = |M'|$ for some typed $\lambda$-term $M'$,
      then there exists $N'$ with $|N'| = N$ and $M' \reduce N'$.
    \end{enumerate}
\end{proposition}

In practice, the type inference is left to the evaluator of a programming
language, so it is natural to ask if the following properties can be done
in a mechanical way:
\begin{description}
  \item[Typability:] Is there a type $\sigma$ with $\vdash M : \sigma$? 
  \item[Type checking:] Does $\vdash M : \sigma$ hold for a given type $\sigma$?
  \end{description}
Both questions are answered positively:

\begin{theorem}
  Typability and type checking are both \emph{decidable}.
\end{theorem}
%See \cite[Section 3.2]{Sorensen2006} for further detail.

\subsection*{Exercise}
\begin{enumerate}
  \item Show that $\Omega$ does not have a type.
  \item Show that $\lambda (x : \sigma).\, x\; x$ is a typed $\lambda$-term but
    there exists no type $\tau$ with 
    \[
      \vdash \lambda (x : \sigma).\, x\; x : \tau.
    \]
  \item Describe all possible types for Church numerals $\bc_{n} = \lambda
    f\,x.\, f^n\;x$. You may find $\bc_{0}$ and $\bc_{1}$ confusing, though.
  \end{enumerate}
\section{Type safety}
\subsection{Preservation}
One of the main features of typed programs is that that every well-typed terms
always reduces to a term of the same type---``well-typed programs do not go
wrong!''
\begin{theorem}[Preservation Theorem]
  \label{thm:preservation}
  Let $M$ and $N$ be typed $\lambda$-terms. 
  If $M \onereduce N$, then
  \[
    \Gamma \vdash M : \sigma \implies \Gamma \vdash N : \sigma
  \]
\end{theorem}
\begin{proof}
  We prove it by induction on both the derivation of $\Gamma \vdash M :
  \sigma$ and $M \onereduce N$. However, the only non-trivial case is
  \[
    \Gamma \vdash (\lambda (x_1 : \tau).\, M_1)\; N : \sigma
    \quad\text{and}\quad
    (\lambda (x_1 : \tau).\, M_1)\;N \onereduce M_2\subst{x_2}{N}
  \]
  with the induction hypothesis valid for
  \[
    \Gamma, x_1 : \tau \vdash M_1 : \sigma
    \quad\text{and}\quad
    \Gamma \vdash N : \tau
  \]
  where $\lambda x_1:\tau.\,M_1$ and $\lambda x_2:\tau.\,M_2$ are
  $\alpha$-equivalent. The rest is left as an exercise.
\end{proof}
However, the converse may not hold.
\begin{example}
  Recall that 
  \begin{enumerate}
    \item $\mathbf{I} = \lambda x.\, x$
    \item $\mathbf{K}_1 = \lambda x\,y.\, x$
    \item $\Omega = (\lambda x.\, x\,x)\,(\lambda x.\, x\,x)$
  \end{enumerate}
  and $\mathbf{K}_1\,\mathbf{I}\,\Omega \reduce \mathbf{I}$. However, 
  \[
    \vdash \mathbf{I} : \sigma \to \sigma
    \not\implies
    \vdash \mathbf{K}_1\;\mathbf{I}\;\Omega : \sigma \to \sigma.
  \]
\end{example}
To see why the latter judgement fails, we need the following lemma:
\begin{lemma}[Typability of subterms]
  Let $M$ be a  $\lambda$-term and $\sigma$ a type. Then,  
  \[
    \Gamma \vdash M : \sigma 
    \implies \Gamma' \vdash M' : \sigma'
  \]
  for every subterm $M'$ of~$M$.
\end{lemma}
\begin{proof}
  By induction on $\Gamma \vdash M : \sigma$.
\end{proof}
%\subsection{Progress}
%Another feature of well-typed terms is that a well-typed \emph{closed} term is
%either a \emph{value} or reducible---``well-typed programs do not get stuck!''
%
%\begin{theorem}[Progress Theorem]
%  If ${}\vdash M : \sigma$ for $M \in \Lambda_\T$, then either $M$ is a value or
%  there is $M'$ with $M \onereduce M'$.
%\end{theorem}
%\todo[inline]{define value?}
%
\subsection{Strong normalisation}
Another property of type system is called \emph{strong normalisation}. It means
that every well-typed terms terminates eventually. 
\begin{theorem}
  Suppose that ${}\vdash M : \sigma$. Then,
  there is no infinite sequence of reductions
  \[
    M \onereduce M_1 \onereduce \dots \onereduce \dots.
  \]
\end{theorem}
\begin{corollary}
  Every well-typed $\lambda$-term ${}\vdash M:\sigma$ has a normal form.
\end{corollary}
With the decidability of type checking, Preservation Theorem, Progress Theorem,
and the strong normalisation, we actually have shown that there exists a
decidable evaluator of simply typed lambda calculus that always reduce a
well-typed term of type~$\sigma$ to a value of type~$\sigma$. In fact, since
statements are proved by induction, we can implement an evaluator naively from
the proofs. Moreover, we will see later how to extract a program from a
\emph{formal} proof.
\subsection*{Exercise}
\begin{enumerate}
  \item Show that $\vdash \mathbf{K}_1\;\mathbf{I}\;\Omega : \sigma \to \sigma$
    does not have a type.
  \item Why does the converse of Theorem~\ref{thm:preservation} fail?
  \item (*)Show that if $\Gamma, x : \tau \vdash M : \sigma$
    and $\Gamma \vdash N : \tau$ then $\Gamma \vdash M\subst{x}{N} : \sigma$.
    Hint: Recall the definition of substitution and prove by induction
    on the derivation of $\Gamma, x:\tau\vdash M : \sigma$.
  \item Show that $\Gamma, x : \tau \vdash M :\sigma
    \iff \Gamma, x' : \tau \vdash M\subst{x}{x'} : \sigma$. That is,
    $\alpha$-equivalent well-typed terms have the same type.
  \item Complete the proof of Theorem~\ref{thm:preservation}. Hint: Use the
    above two statements.
  \item Prove that Preservation Theorem holds for simply typed lambda calculus
    \textit{\`a la} Curry using type erasing.
\end{enumerate}
\section{Programming in simply typed lambda calculus}
In this section, we use $\lambda_\to$ in the style of Curry.  
Church numerals are also typable in simply typed lambda calculus.
The type of natural numbers is of the form
\[
  \nat_\tau \defeq (\tau \to \tau) \to \tau \to \tau
\]
for every type $\tau \in \T$, and Church
numerals are of type $\nat_\tau$ for every type~$\tau$:
  \begin{description}
    \item[Church numerals]
      \begin{align*}
        & \bc_n \defeq \lambda f\,x.\,
        f^n x \\
        \vdash{} & \bc_n : \nat_\tau
      \end{align*}
    \item[successor]
      \begin{align*}
        & \suc \defeq \lambda n \,f\,x\,.\;f\;(n\;f\;x) \\
        \vdash{} & \suc : \nat_\tau \to \nat_\tau
      \end{align*}
    \item[addition]
      \begin{align*}
        & \add \defeq \lambda n\,m\,f\,x.\; (m\;f)\;(n\;f\;x) \\
        \vdash{} & \add : \nat_\tau \to \nat_\tau \to \nat_\tau
      \end{align*}
    \item[muliplication] 
      \begin{align*}
        & \mul \defeq \lambda n\,m\,f\,x.\, (m\;(n\;f))\;x\\
      \vdash{} & \mul : \nat_\tau \to \nat_\tau
      \end{align*}
    \item[conditional]
      \begin{align*}
        & \ifz \defeq \lambda n\,x\,y.\, n\;(\lambda z.\, x)\;y\\
        \vdash {} & \ifz : ?
      \end{align*}
  \end{description}
The type of~$\ifz$ may not be as obvious as you may expect.
Try to find one as general as possible and justify your guess.

Within simply typed lambda calculus, we may naturally ask how many computable
functions can be defined, as it excludes non-terminating $\lambda$-terms. 
\begin{definition}
  A function $f\colon \mathbb{N}^k \to \mathbb{N}$ is called
  \emph{$\lambda_\to$-definable} if there exists a $\lambda$-term $F$
   of type $\nat \to \nat \to \dots \nat \to \nat$ (with $k+1$
  many $\nat$ in total) such that
  \[
    F\,\bc_{n_1}\ldots\bc_{n_k} \reduce \bc_{f(n_1, \dots, n_k)}
  \]
  for every sequence $(n_1, n_2, \ldots, n_k) \in \mathbb{N}^k$.
\end{definition}
\begin{theorem}[Schwichtenberg~\cite{Schwichtenberg1976}]
  The $\lambda_\to$-definable functions are the class of functions
  of the form $f\colon \mathbb{N}^k \to \mathbb{N}$ closed under compositions
  and contains the constant functions, projections, additions, multiplications,
  and the conditional function 
  \[
    \mathrm{ifz}(n_0, n_1, n_2) = 
    \begin{cases}
      n_1 & \text{if } n_0 = 0\\
      n_2 & \text{otherwise.}
    \end{cases}
  \]
\end{theorem}

By a similar technique, we can also define the type of Boolean values as
\[
  \bool_\tau \defeq \tau \to \tau \to \tau
\]
for every types $\tau$ with the subscript suppressed if there is no danger of
confusion.  Truth values and the conditional operator are defined by
\begin{description}
  \item[Boolean values]
      \[
        \true \defeq \lambda x\,y.\,x \\
        \quad\text{and}\quad
        \false \defeq \lambda x\,y.\,y \\
      \]
  \item[Conditional]
    \begin{align*}
      & \cond \defeq \lambda b\,x\,y.\, b\,x\,y\\
      \vdash {} & \cond : \bool_{\tau} \to \tau \to \tau \to \tau
    \end{align*}
\end{description}

%\subsection{System T}
%It is also convenient to add some primitive types to our typed lambda calculus (or
%so-called built-in types in programming languages used in practice) as well as
%\emph{primitive terms}. For example, for a system with natural numbers, we
%include numerals in the generation of syntax:
%  \begin{multicols}{2}
%    \begin{prooftree}
%      \AXC{$x \in V$}
%      \UIC{$x \in \Lambda_{\T, \nat}$}
%    \end{prooftree}
%    \begin{prooftree}
%      \AXC{\phantom{$n \in \mathbb{N}$}}
%      \UIC{$\zero\in \Lambda_{\T, \nat}$}
%    \end{prooftree}
%    \begin{prooftree}
%      \AXC{$M \in \Lambda_{\T, \nat}$}
%      \UIC{$\suc\;M\in \Lambda_{\T, \nat}$}
%    \end{prooftree}
%    \begin{prooftree}
%      \AXC{$M \in \Lambda_{\T,\nat}$}
%      \AXC{$N \in \Lambda_{\T,\nat}$}
%      \BIC{$(M\, N) \in \Lambda_{\T, \nat}$}
%    \end{prooftree}
%    \begin{prooftree}
%      \AXC{$M \in \Lambda_{\T, \nat}$}
%      \AXC{$x \in V$}
%      \AXC{$\tau \in \mathbb{T}$}
%      \TIC{$\lambda (x:\tau).\; M \in \Lambda_{\T, \nat}$}
%    \end{prooftree}
%  \end{multicols}
%\noindent and let $\mathbb{G} \defeq \{\nat\}$ with an addition set of typing
%rules:
%\begin{multicols}{2}
%\begin{prooftree}
%  \AXC{\phantom{$\Gamma$}}
%  \UIC{$\Gamma \vdash \zero : \nat$}
%\end{prooftree}
%\begin{prooftree}
%  \AXC{$\Gamma \vdash M : \nat$}
%  \UIC{$\Gamma \vdash \suc\:M : \nat$}
%\end{prooftree}
%\end{multicols}
%\noindent In this way, we can derive that every $\underline{n} \defeq
%\suc^n\;\zero$ has the type of $\nat$:
%\[
%  \Gamma \vdash \underline{n} : \nat
%\]
%
%However, it still remains to add primitive operations such as addition,
%multiplication, division, and so on with a proper set of reduction rules. As we
%have seen that recursions on natural numbers in untyped lambda calculus can be
%defined with the fixpoint operator. Something similar can be done:
%\begin{definition}
%  In addition to typing rules introduced so far, we add the following
%  \begin{prooftree}
%    \AXC{$\Gamma \vdash M : \nat$}
%    \AXC{$\Gamma \vdash M_0 : \sigma$}
%    \AXC{$\Gamma \vdash F : \nat \to (\sigma \to \sigma)$}
%    \TIC{$\Gamma \vdash \fold\;M_0\;F : \nat \to \sigma $}
%  \end{prooftree}
%  with additional reductions
%  \[
%    \fold\;M_0\;F\;\zero
%    \xrightarrow{\beta}
%    M_0
%    \quad\text{and}\quad
%    \fold\;M_0\;F\;(\suc\;M)
%    \xrightarrow{\beta}
%    F\;M\;(\fold\;F\;M).
%  \]
%\end{definition}
%\begin{definition}
%  \todo[inline]{define addition, multiplication using $\fold$}
%\end{definition}
\subsection*{Exercise}
\begin{enumerate}
  \item Define the conjunction operation $\mathtt{and} : \bool_\bool \to \bool
    \to \bool$ and similarly for $\mathtt{or}$, and $\mathtt{not}$
    in simply typed lambda calculus.
\end{enumerate}

\bibliographystyle{plain}
\bibliography{library} 
\end{document}
