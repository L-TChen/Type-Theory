%! TEX program = xelatex
%! TEX root = lecture1_slide.tex
\title{An Elementary Introduction to Programming Languages}
\subtitle{Untyped Lambda Calculus}
\begin{document}
\begin{frame}
\maketitle
\end{frame}

\begin{frame}[fragile]{Assessment guidelines}
  \begin{description}
    \item[Deadline] 17:00, 20 Aug
    \item[Assessment] 
      \begin{description}
        \item[Assignment] (10\% + 10\%)
        \item[Exam] (100\%)
      \end{description}
    \item[Email] \texttt{liang.ting.chen.tw(at)gmail(dot)com}
  \end{description}

  \textbf{Please follow the instructions for assignments below.}

  \begin{enumerate}
    \item Write down your name and your student number. 
    \item Use A4 paper only in physical form or in \emph{non-scanned} PDF. 
    \item Be brief but comprehensive.
    \item Submit assignments in person or by email with 
      \begin{description}
        \item[subject] \texttt{[FLOLAC] PL HW\%x\%}
        \item[attachment] \texttt{PL-HW\%x\% - \%STDNO\% - \%NAME\%.pdf}
        \item[body] (optional)
      \end{description}
  \end{enumerate}
\end{frame}
%\begin{frame}{Introduction}
%  $\lambda$-calculus \dots
%  \begin{enumerate}
%    \item was developed by Alonzo Church
%    \item is a model of computation
%    \item is a backbone of programming languages
%  \end{enumerate}
%  \mode<presentation>{\vfill}
%  As a programming language, it only supports $3$ constructs 
%  \begin{enumerate}
%    \item variable
%    \item function definition
%    \item function application
%  \end{enumerate}
%  \mode<presentation>{\vfill}
%\end{frame}

\section{Syntax and Semantics of Untyped $\lambda$-Calculus}

\begin{frame}{Terms of $\lambda$-calculus}

Let $V \defeq \{ x, y, z, \ldots \}$ be a countably infinite set of
\emph{variables}.
\begin{definition}[Syntax of $\lambda$-calculus]
   The set~$\Lambda$ of \alert{$\lambda$-terms} is defined
  by
  \begin{enumerate}
    \item Variable:
      \begin{prooftree}
        \AXC{$x \in V$}
        \RightLabel{(var)}
        \UIC{$x \in \Lambda$}
      \end{prooftree}

    \item Function application of $M$ to the argument $N$:
      \begin{prooftree}
        \AXC{$M \in \Lambda$}
        \AXC{$N \in \Lambda$}
        \RightLabel{(app)}
        \BIC{$M \; N \in \Lambda$}
      \end{prooftree}

    \item Function abstraction with an argument $x$ and a function body $M$:
      \begin{prooftree}
        \AXC{$M \in \Lambda$}
        \AXC{$x \in V$}
        \RightLabel{(abs)}
        \BIC{$\lambda x.\, M \in \Lambda$}
      \end{prooftree}
  \end{enumerate}
\end{definition}
\end{frame}

\begin{frame}{Examples and non-examples}
  \begin{enumerate}
    \item $(x\;y)\;z$
    \item $x\;(y\; z)$
    \item $\lambda x.\, y$
    \item $\lambda x.\, x$
    \item $\lambda s.\,(\lambda z.\, (s \;z))$
    \item $\lambda a.\,(\lambda b.\, (a\;(\lambda c.\, a\; b)))$
    \item $(\lambda x.\, x)\;(\lambda y.\, y)$
  \end{enumerate}
  The following are NOT examples
  \begin{enumerate}
    \item $\lambda (\lambda x.\, x).\, y$
    \item $\lambda x. $
    \item $\lambda.\, x$
    \item $\dots$
  \end{enumerate}
\end{frame}

\begin{frame}{Closed terms}
  A term $M$ is \alert{closed} if each variable which occurs in $M$ are captured by an abstraction.

  Put differently, $M$ is closed if every variable is in scope.

  Closed terms:
  \begin{enumerate}
    \item $\lambda x.\, x$
    \item $\lambda s.\,(\lambda z.\, (s \;z))$
    \item $\lambda a.\,(\lambda b.\, (a\;(\lambda c.\, a\; b)))$
    \item $(\lambda x.\, x)\;(\lambda y.\, y)$
  \end{enumerate}
  Open terms:
  \begin{enumerate}
    \item $(x\;y)\;z$
    \item $x\;(y\; z)$
    \item $\lambda x.\, y$
  \end{enumerate}
\end{frame}

\begin{frame}{Conventions}
  \begin{description}
    \item[Consecutive abstractions]
      \[
        \lambda x_1\,x_2\,\ldots x_n.\, M \defeq \lambda x_1.\,(\lambda x_2.\,(\ldots (\lambda x_n.\, M)\ldots))
      \]
    \item[Consecutive applications]
      \[
        M_1\;M_2\; M_3\; \dots\;M_n \defeq (\dots((M_1\;M_2)\;M_3) \dots )\; M_n
      \]
    \item[Function body extends as far right as possible]
      \[
        \lambda x.\, M\;N \defeq \lambda x.\,(M\; N)
      \]
      instead of $(\lambda x.\,M)\; N$.
  \end{description}
  For example, 
      $\lambda x_1.\,(\lambda x_2.\, x_1) \equiv \lambda x_1\,x_2.\,x_1$
      and
      $\alert{x\;y\;z}$ means $\alert{(x\; y)\;z}$. 
  \alert{Warning}. We apply these rules in our \emph{meta}-language.
\end{frame}

\begin{frame}{Meta-Language and Object-Language}
  \begin{itemize}
    \item \emph{Meta-language} is the language we use to describe the object of
      study. E.g. English, or naive set theory. 
    \item \emph{Object-language} is the object of study. E.g., arithmetic
      expressions, $\lambda$-terms, etc.
  \end{itemize}
  As \emph{naming} a function is not allowed in $\lambda$-calculus, we do so
  in the meta-language:
    \[
      \mathtt{id} \defeq \lambda x.\, x
    \]
  \begin{enumerate}
    \item $\mathtt{id}$ is a symbol different from `$\lambda x.\,x$' in our meta-language.
    \item $\mathtt{id}$ and $\lambda x.\, x$ are the \emph{same} $\lambda$-term, denoted by
      \[
        \mathtt{id} \equiv \lambda x.\, x
      \]
  \end{enumerate}
\end{frame}

\begin{frame}
  \begin{example}[Identity function]
    \[
      \mathtt{id} \defeq \lambda x.\, x
    \]
  \end{example}

  \begin{example}[Projections]
    \[
      \mathtt{fst} \defeq \lambda x.\, \lambda y.\, x
      \quad\text{and}\quad \mathtt{snd} \defeq \lambda x.\, \lambda y.\, y
    \]
  \end{example}
  \begin{example}[Church encoding of Natural numbers]
    \begin{align*}
      &\bc_{0}   &&\defeq \;&& \lambda f\,x.\, x \\
      &\bc_{1} &&\defeq \;&& \lambda f\,x.\, f\,x \\
      &\bc_{2} &&\defeq \;&& \lambda f\,x.\, f\,(f\,x) \\
      &\bc_{3} &&\defeq \;&& \lambda f\,x.\, f\,(f\,(f(\,x))) \\
      && \vdots 
    \end{align*}
  \end{example}
\end{frame}

\begin{frame}{$\alpha$-equivalence, informally}
  \begin{definition}
    Two $\lambda$-terms $M$ an $N$ are \alert{$\alpha$-equivalent} 
    \[
      M \equiv_\alpha N
    \]
    if variables \emph{bound} by abstractions can be renamed to derive the same term. 
  \end{definition}
  \begin{example}
    \begin{enumerate}
      \item $\lambda x.\, x$ and $\lambda y.\, y$ are distinct $\lambda$-terms but $\lambda x.\, x
        \equiv_\alpha \lambda y.\, y$. 
      \item $\lambda x.\, \lambda y.\, y \equiv_\alpha 
        \lambda z.\, \lambda y.\, y$. 
      \item $\lambda x.\, \lambda y.\, x \mathrel{\not\equiv}_\alpha
        \lambda x.\, \lambda y.\, y$. 
    \end{enumerate}
  \end{example}
  $\alpha$-equivalent terms are \emph{programs of the same structure}.
  
\end{frame}

\begin{frame}{Informal evaluation}
  The \alert{evaluation} of $\lambda$-calculus happens in this form only:
  \[
    \underbrace{(\lambda x.\, M)\,N}_{\beta\text{-redex}} \longrightarrow 
    \underbrace{M\;\subst{N}{x}}_{\text{substitution of $N$ for $x$ in $M$}}
  \]

  For example, $(\lambda x.\, x +1)\;3 \to 3 + 1$.
  \mode<presentation>{\vfill}
  How to evaluate the following terms? 
  \begin{enumerate}
    \item $(\lambda x.x)\,z$
    \item $(\lambda x\, y.\,x)\,y$
    \item $(\lambda y\, y.\,y)\,x$
  \end{enumerate}
  \mode<presentation>{\vfill}
  The goal of this part is to define evaluation precisely. 

\end{frame}

\begin{frame}{Equality, equality, equality!}

  Moreover, we will have different notions of \alert{equality}.

  \begin{itemize}
    \item $1 + 1 \neq_\alpha 2$
    \item $1 + 1 \reduce 2$ but $2 \not\reduce 1 + 1$
    \item $1 + 1 =_\beta 2$
  \end{itemize}
  Each of above statements says the following.
  \begin{itemize}
    \item $1 + 1$ is a different expression from $2$.
    \item $1 + 1$ reduces to $2$, but $2$ does not reduce to $1 + 1$.
    \item $1 + 1$ and $2$ have the same computational meaning.
  \end{itemize}
\end{frame}

\begin{frame}{Free and Bound Variables}
\begin{definition}
  The set $\FV$ of free variables of a term $M$ is defined by
  \begin{align*}
    \FV(x) & = \{x\} \\
    \FV(\lambda x.\, M) & = \FV(M) - \{x\} \\
    \FV(M\;N) & = \FV(M) \cup \FV(N)
  \end{align*}
\end{definition}
$\FV(x\;y\;z) = \FV(x\;y) \cup \FV(z) = \FV(x) \cup \FV(y) \cup \{z\} = \{x, y, z\}$
\begin{definition}
  \begin{enumerate}
    \item A variable $y$ in $M$ is \alert{free} if $y \in \FV(M)$.
    \item A $\lambda$-term~$M$ is \alert{\emph{closed}} if $\FV(M) = \emptyset$. 
  \end{enumerate}
\end{definition}

\end{frame}

\begin{frame}{Exercise: free variables}
  Calculate the set of free variables of following terms:
  \begin{enumerate}
    \item $x\;(y\; z) $
    \item $\lambda x.\, y$
    \item $\lambda x.\, x$
    \item $\lambda s\,z.\, (s \;z))$
    \item $(\lambda x.\, x)\;\lambda y.\, y$
  \end{enumerate}
\end{frame}

\begin{frame}{Exercise: bound variables}
  A variable $x$ in a term $M$ is called \emph{bound} if it is not free.

  Define
  \begin{itemize}
    \item $\Var(M)$ the set of variables of a term $M$ by structural recursion on $\Lambda$.
    \item $\BV(M)$ the set of bound variables.
  \end{itemize}
\end{frame}

\begin{frame}{Substitution}
   A \alert{substitution} is a process of replacing \emph{free} variables by
   another terms. 

   The name of a variable does not matter but the location does. So, ...
   \begin{enumerate}
     \item bound variables should remain bound after substitution.
     \item other free variables should remain free after substitution.
   \end{enumerate}

   Concretely, we want to avoid ...
  \begin{enumerate}
    \item $(\lambda y.\,y)[x/y] \equiv (\lambda y.\, x)$
    \item $(\lambda y.\, x)[y/x] \equiv (\lambda y.\, y)$ 
  \end{enumerate}
   
  
\end{frame}
  
\begin{frame}{Naive substitution I}
  \begin{definition}
    For $x \in V$ and $L \in \Lambda$, the substitution of $L$ for $x$ is
    defined by
    \begin{align*}
      x\subst{L}{x} & = L \\
      y\subst{L}{x} & = y & \text{if $x \neq y$} \\
      (M\;N)\subst{L}{x}& = M\subst{L}{x} \; N\subst{L}{x} \\
      (\lambda y.\, M)\subst{L}{x} & = \lambda y.\, M \subst{L}{x}
    \end{align*}
  \end{definition}
    A bound variable may become free. 
    \[
      (\lambda x.\, x)\subst{y}{x} = \lambda x.\, y
    \]
\end{frame}
\begin{frame}{Naive substitution II}
  \begin{definition}
    For $x \in V$ and $L \in \Lambda$, the substitution
    of $L$ for $x$ is defined by
    \begin{align*}
      x\subst{L}{x} & = L \\
      y\subst{L}{x} & = y && \text{if $x \neq y$} \\
      (M\,N)\subst{L}{x} & = M\subst{L}{x}\; N\subst{L}{x} \\
      (\lambda y.\, M)\subst{L}{x} & = \lambda y.\, M\subst{L}{x} && \text{if $x \neq y$} \\
      (\lambda y.\, M)\subst{L}{x} & = \lambda y.\, M && \text{if $x = y$} 
    \end{align*}
  \end{definition}
    A variable may be captured by an abstraction. 
    \[
      (\lambda x. y)\subst{x}{y} = \lambda x.\, x
    \]
\end{frame}


\begin{frame}{Capture-avoiding substitution}
  Capture-avoiding substitution of $L$ for the \alert{free occurrences} of $x$ is a
  \emph{partial} function $(\cdot)\subst{L}{x}\colon \Lambda \to \Lambda$ defined by
  \begin{align*}
    x\subst{L}{x} & = L \\
    y\subst{L}{x} & = y && \text{if $x \neq y$} \\
    (M\, N)\subst{L}{x} & = (M\subst{L}{x}\; N\subst{L}{x}) \\
    (\lambda x.\, M)\subst{L}{x} & = \lambda x.\, M \\
    (\lambda y.\, M)\subst{L}{x} & = \lambda y.\, M\subst{L}{x}                                 &&\text{if $x \neq y$ and $y \not\in \FV(L)$}
  \end{align*}
  \begin{definition}[Freshness]
    A variable $y$ is \alert{fresh} for $L$ if $y \notin \FV(L)$.  
  \end{definition}
\end{frame}

\begin{frame}{Renaming of Bound Variables}
If a variable $y$ is \emph{fresh} for $M$, the bound variable $x$ of~$\lambda
x.\, M$ can be renamed to~$y$ without changing the meaning. 
  \mode<presentation>{\vfill}
\begin{definition}[$\alpha$-conversion]
  $\alpha$-conversion is a relation $\to_\alpha$ defined by
  \[
    (\lambda x.\, M) \longrightarrow_\alpha \lambda y.\, M\subst{y}{x}
    \quad\text{if $y$ is fresh for $M$}.
  \]
\end{definition}

Yet, $M\;(\lambda x.\, x) \longrightarrow_\alpha M\;(\lambda y.\, y)$ does not hold.
\end{frame}

\begin{frame}
  \begin{definition}[$\alpha$-equivalence]
  \begin{multicols}{2}
    \begin{prooftree}
      \AXC{$M_1 \mathrel{\to_\alpha} M_2$}
      \UIC{$M_1 \mathrel{=_\alpha} M_2$}
    \end{prooftree}
    \begin{prooftree}
      \AXC{$M_1 \mathrel{=_\alpha} M_2$}
      \UIC{$M_2 \mathrel{=_\alpha} M_1$}
    \end{prooftree}
    \begin{prooftree}
      \AXC{$M_1 \mathrel{=_\alpha} M_2$}
      \AXC{$M_2 \mathrel{=_\alpha} M_3$}
      \BIC{$M_1 \mathrel{=_\alpha} M_3$}
    \end{prooftree}
    \begin{prooftree}
      \AXC{$M_1 \mathrel{=_\alpha} M_2$}
      \UIC{$(M_1\,N) \mathrel{=_\alpha} (M_2\,N)$}
    \end{prooftree}
    \begin{prooftree}
      %\AXC{$M_1 \mathrel{R} M_2$}
      \AXC{$N_1 \mathrel{=_\alpha} N_2$}
      \UIC{$(M\,N_1) \mathrel{=_\alpha} (M\,N_2)$}
    \end{prooftree}
    \begin{prooftree}
      \AXC{$M_1 \mathrel{=_\alpha}M_2$}
      \UIC{$(\lambda x.\, M_1) \mathrel{=_\alpha}(\lambda x.\, M_2)$}
    \end{prooftree}
  \end{multicols}
\end{definition}

  $(\lambda y.\, y)\;(\lambda x.\, x) =_\alpha (\lambda x.\,x)\;(\lambda y.\,y)$. Why? Apply our rules! 
\end{frame}

\begin{frame}

  \begin{proof}
  \begin{prooftree}
    \AXC{$\lambda x.\, x \to_\alpha \lambda y.\, x\subst{y}{x}$}
    \UIC{$\lambda x.\, x =_\alpha \lambda y.\, y$}
    \UIC{$(\lambda y.\, y)\; (\lambda x.\, x) =_\alpha (\lambda y.\, y)\; (\lambda y.\, y)$}
    \AXC{$\lambda y.\, y \to_\alpha \lambda x.\, y\subst{x}{y}$}
    \UIC{$\lambda y.\, y =_\alpha \lambda x.\, x$}
    \UIC{$(\lambda y.\, y)\; (\lambda y.\, y) =_\alpha (\lambda x.\, x)\; (\lambda y.\, y)$}
    \BIC{$(\lambda y.\, y)\; (\lambda x.\, x) =_\alpha (\lambda x.\, x)\; (\lambda y.\, y)$}
  \end{prooftree}
  \end{proof}
  
\end{frame}
\begin{frame}{Exercise}

Which of the following pairs are $\alpha$-equivalent? Why?
\begin{enumerate}
  \item $x$ and $y$
  \item $\lambda x\,y.\, y$ and $\lambda z\,y.\, y$
  \item $\lambda x\,y.\, x$ and $\lambda y\,x.\, y$
  \item $\lambda x\,y.\, x$ and $\lambda x\,y.\, y$
\end{enumerate}
  
  $(\lambda y.\,y\;y)\;(\lambda z.\,\lambda x.\, z\;x)$
\begin{block}{Convention}
  Terms are equal up to $\alpha$-equivalence of bound variables. 
\end{block}

That is, feel free to rename any bound variable whenever convenient.
  
\end{frame}
%
%
%\begin{frame}{$\eta$-conversion}
%  Pointy style is \emph{assumed} to be equivalent to point-free style. 
%  \begin{description}
%    \item[$\eta$-reduction]
%      $\lambda x.\,(M\,x) \longrightarrow_\eta M$
%    \item[$\eta$-expansion]
%      $M \longrightarrow_\eta \lambda x.\,(M\,x) $
%      where $x$ is fresh for $M$.
%  \end{description}
%  \mode<presentation>{\vfill}
%  $\eta$-equivalence is a form of \emph{extensionality} limited to
%  $\lambda$-terms.
%  \[
%    f = g \iff \forall x.\, f(x) = g(x)
%  \]
%\end{frame}

\begin{frame}{$\beta$-conversion}
\begin{definition}[$\beta$-conversion]
  $\beta$-conversion is defined by
  \[
    \underbrace{(\lambda x.\, M)\;N}_{\beta\text{-redex}} \longrightarrow_{\beta} M\subst{N}{x}
  \]
\end{definition}
  Good: 
    \[
      ((\lambda x.\,\lambda y.\,x)\;M)
      \longrightarrow_{\beta}
      (\lambda y.\, x)\subst{M}{x}
      = \lambda y.\, x\subst{M}{x}
      = \lambda y.\, M
    \]
  Bad: 
  \[
    ((\lambda x\, y.\, x)\; M)\;N
    \longrightarrow_{\beta} \, ?
  \]
\end{frame}

\begin{frame}{Full $\beta$-reduction}
\begin{definition}
  The \alert{full $\beta$-reduction} is defined by
  \begin{multicols}{2}
    \begin{prooftree}
      \AXC{$M_1 \longrightarrow_{\beta} M_2$}
      \UIC{$M_1 \onereduce M_2$}
    \end{prooftree}
    \begin{prooftree}
      \AXC{$M_1 \onereduce M_2$}
      \UIC{$\lambda x.\, M_1 \onereduce \lambda x.\, M_2$}
    \end{prooftree}
    \begin{prooftree}
      \AXC{$M_1 \onereduce M_2$}
      \UIC{$M_1\;N \onereduce M_2\;N$}
    \end{prooftree}
    \begin{prooftree}
      \AXC{$N_1\onereduce  N_2$}
      \UIC{$M\;N_1 \onereduce M\;N_2$}
    \end{prooftree}
  \end{multicols}
\end{definition}
  $M \reduce M'$ denotes there are finitely many $M_i$ such that 
  \[
    M_0 \onereduce M_1 \onereduce \dots M_i \dots \onereduce N 
  \]
  with $M = M_0$ and $M_n = N$ (where $n$ can be $0$). 
\end{frame}

\begin{frame}{$\alpha$-conversion during $\beta$-conversion}

  Renaming of bound variables may need to happen during reduction:
  \begin{align*}
    (\lambda y.\, y\; y)\; (\lambda z\,x.\, z\;x) 
  & \onereduce && (\lambda z\,x.\, z\;x)\; (\lambda z\,x.\, z\;x) \\
  & \onereduce && \lambda x.\, (\lambda z\,x.\, z\;x) \;x  \\
  & =_\alpha   && \lambda x.\, (\lambda z\,y.\, z\;y) \;x  \\
  & \onereduce && \lambda x.\, (\lambda y.\, x\;y)
  \end{align*}
  
  Even worse, we actually need infinitely many variables:
  \[
    (\lambda y.\,y\; s\; y)\; (\lambda t\,z\,x.\, z\; (t\; x)\; z)
  \]
  \begin{exercise*}
    Evaluate the above term.
  \end{exercise*}
\end{frame}

\begin{frame}{Computational meaning}
  \begin{definition}
    $M$ and $N$ have the same \alert{\emph{computational meaning}} if $M =_\beta N$
    where $=_\beta$ is defined inductively by
    \begin{multicols}{2}
      \begin{prooftree}
        \AXC{$M \onereduce N$}
        \UIC{$M =_\beta N$}
      \end{prooftree}
      \begin{prooftree}
        \AXC{}
        \UIC{$M =_\beta M$}
      \end{prooftree}
      \columnbreak
      \begin{prooftree}
        \AXC{$M =_\beta N$}
        \UIC{$N =_\beta M$}
      \end{prooftree}
      \begin{prooftree}
        \AXC{$L =_\beta M$}
        \AXC{$M =_\beta N$}
        \BIC{$L =_\beta N$}
      \end{prooftree}
    \end{multicols}
  \end{definition}
  \begin{itemize}
    \item $\bc_2 =_\beta (\lambda x\, y. y)\;\bc_1\;\bc_2$
    \item $\lambda z.\, (\lambda x\, y. x)\; z =_\beta \lambda z\, y.\, z$
  \end{itemize}
\end{frame}


\section{Programming in Lambda Calculus}
\begin{frame}{Church Encoding of Boolean values}
  
Boolean and conditional can be encoded as combinators.
  
\begin{description}
  \item[Boolean]
    \begin{align*}
      & \mathtt{True}  && \defeq && \lambda x\,y.\, x \\
      & \mathtt{False} && \defeq && \lambda x\,y.\, y
    \end{align*}

  \item[Conditional]
    \begin{align*}
      \mathtt{if} & \defeq \lambda b\;x\;y.\;b\,x\,y  \\
      \mathtt{if} \;\mathtt{True} \;M\;N &\reduce M \\
      \mathtt{if} \;\mathtt{False}\;M\;N &\reduce N
    \end{align*}
    for any two $\lambda$-terms $M$ and~$N$.
\end{description}

\end{frame}
\begin{frame}[allowframebreaks]{Church Encoding of Natural Numbers}

  Natural numbers can be encoded as $\lambda$-terms, so can arithmetic operations. 
  \begin{description}
    \item[Church numerals] 
      \begin{align*}
        &\bc_0 &&\defeq\;&& \lambda f\,x.\, x \\
        &\bc_{1} &&\defeq\;&& \lambda f\,x.\, f\,x \\
        &\bc_{2} &&\defeq\;&& \lambda f\,x.\, f\,(f\,x) \\
        &\bc_{n+1} &&\defeq\;&& \lambda f\,x.\, f^{n+1}\,(x)
      \end{align*}
      where $f^1(x) \defeq f\;x$ and $f^{n+1}(x)  \defeq f\;(f^n(x))$.
    \item[Successor]
      \begin{align*}
        & \mathtt{succ} && \defeq\; && \lambda n.\,\lambda f\,x.\, f\,(n\,f\,x) \\
        & \mathtt{succ}\;\bc_n && \reduce\; && \bc_{n+1}
      \end{align*}
      for any natural number~$n \in \mathbb{N}$.
    \item[Addition]
      \begin{align*}
        & \mathtt{add} && \defeq\; && \lambda n\,m.\,\lambda f\,x.\;
        m\;f\;(n\;f\;x)  \\ & \mathtt{add}\;\bc_{n}\;\bc_{m}\;
                            && \reduce\; && \bc_{n + m}
      \end{align*}
  \end{description}


  \begin{description}
    \item[Conditional]
      \begin{align*}
        & \mathtt{ifz} && \defeq \; \lambda n\;x\;y.\;n\;(\lambda z.\;y)\;x 
        \\
        & \mathtt{ifz}\;\bc_0\;M\;N && \reduce \; M \\
        & \mathtt{ifz}\;\bc_{n+1}\;M\;N && \reduce \; N
      \end{align*}
    \item[Predecessor]
      \begin{align*}
        & \mathtt{pred} && \defeq\; && \lambda n.\, \lambda f\,x.\, ? \\
        & \mathtt{pred}\;\bc_{0}  && \reduce\; && \bc_{0} \\
        & \mathtt{pred}  \;\bc_{n+1} && \reduce\; && \bc_{n}
      \end{align*}
  \end{description}
\end{frame}

\begin{frame}{Exercise}
  \begin{enumerate}
    \item Define the \emph{flip} operation, i.e.\ a $\lambda$-term~$\mathtt{flip}$
      such that
      \[
        \mathtt{flip}\;M\;N\;P
        =_\beta M\;P\;N
      \]
    \item Define Boolean operations $\mathtt{not}$, $\mathtt{and}$, and $\mathtt{or}$.
    \item Evaluate $\mathtt{succ}\;\bc_0$ and $\mathtt{add}\;\bc_{1}\;\bc_{2}$. 
    \item Define the multiplication $\mathtt{mult}$ over Church numerals.
    \item (Hard) Define $\mathtt{pred}$ so that $\mathtt{pred}\;\bc_0 =_\beta \bc_{0}$ and $\mathtt{pred}\;\bc_{n+1} =_\beta \bc_{n}$.
  \end{enumerate}
\end{frame}
\begin{frame}{General Recursion, informally}

  The summation $\sum_{i = 0}^{n} i$ for $n \in \mathbb{N}$ \emph{can be} defined as
\[
  \mathtt{sum}(n) =
    \begin{cases} 
     0 & \text{if } n = 0 \\
     n + \mathtt{sum}(n - 1)  & \text{otherwise}.
    \end{cases}
\]
Can we avoid the self-reference? Consider the function
$G\colon (\mathbb{N} \to \mathbb{N}) \to (\mathbb{N} \to \mathbb{N})$
defined by
\begin{equation} \label{eq:summation}
  (Gf)(n) \defeq
  \begin{cases}
     0 & \text{if } n = 0 \\
     n + f(n - 1)  & \text{otherwise}.
  \end{cases}
\end{equation}
If $G(\mathtt{sum}') =\mathtt{sum}'$, then $\mathtt{sum}' = \mathtt{sum}$. 

\end{frame}

\begin{frame}{Curry's paradoxical combinator}
\begin{proposition}
  Define
  \[
    \mathbf{Y} \defeq \lambda f.\, (\lambda x.\, f\,(x\,x))\,(\lambda
    x.\,f\,(x\,x)).
  \]
  Then, 
  \begin{align*}
    \mathbf{Y}F
    & \onereduce {\color{red} (\lambda x.\,F\,(x\,x))\,(\lambda x.\, F\,(x\,x))} \\
    & \onereduce F\,({\color{red}(\lambda x.\,F\,(x\,x))\,(\lambda x.\, F\,(x\,x))}) \\
  \end{align*}
  for every $\lambda$-term~$F$. 
\end{proposition}
\end{frame}


\begin{frame}{Summation, formally}
  Using the combinators we have known so far, the equation~\eqref{eq:summation}
  can be defined as $\lambda$-terms:
  \begin{align*}
    G \defeq{} & 
    \lambda f\,n.\, \mathtt{ifz}\;n\;\bc_0\;
    (\add\;n\;(f\;(\mathtt{pred}\;n))) \\
    \mathtt{sum} \defeq{} & \mathbf{Y}G
  \end{align*}
  For example
  \begin{align*}
    \mathtt{sum}\;\bc_1
      &\equiv  (\mathbf{Y}G)\;\bc_1 \\
      &\onereduce G'\;\bc_1 \\
      &\onereduce G\;G'\;\bc_1 \\
      &\onereduce (\lambda n.\, \mathtt{ifz}\;n\;\bc_0\;
      (\add\;n\;(G'\;(\mathtt{pred}\;n))))\;\bc_1 \\
      &\onereduce \mathtt{ifz}\;\bc_1\;\bc_0\;
      (\add\;\bc_1\;(G'\;(\mathtt{pred}\;\bc_1))) \\
      &\onereduce \dots
  \end{align*}
  where
  $G' \defeq ((\lambda x.\, G\;(x\;x))\;(\lambda x.\,G\;(x\;x)))$.
\end{frame}
  
\begin{frame}{Turing's fixed-point combinator}
Here is a fixed-point operator such that $\Theta F \reduce F(\Theta F)$.
\begin{proposition}
  Define 
  \[
    \Theta \defeq 
    (\lambda x\,f.\,f\,(x\,x\,f))\;
    (\lambda x\,f.\,f\,(x\,x\,f))
  \]
  Then, 
  \[
    \Theta F \reduce F (\Theta F)
  \]
\end{proposition}
Try Turing's fixed-point combinator with $G$ to define $\sum_{i=0}^n i$.
\begin{align*}
  G \defeq{} &
  \lambda f\,n.\,
  \mathtt{ifz}\;n\;\bc_0\;
  (\add\;n\;(f\,(\mathtt{pred}\;n))) \\
  \mathtt{sum} \defeq\; & \Theta\,G 
\end{align*}
\end{frame}

\begin{frame}{Exercise}
  \begin{enumerate}
    \item Evaluate $\mathtt{sum}\;\bc_1$ to its normal form in detail.
    \item Define the factorial $n!$ on Church numerals with Turing's fixed-point combinator.
  \end{enumerate}
\end{frame}


\section{Properties of Lambda Calculus}

\begin{frame}
  \begin{example}
    Suppose $M \in \Lambda$ and $y \not\in \FV(M)$. 
    Then, consider 
    \[
      (\lambda y.\, M)\, ((\lambda x.\, x\,x)(\lambda x.\, x\, x))
    \]
  \end{example}
  \mode<presentation>{\vfill}
  Observations:
  \begin{itemize}
    \item Some evaluation may diverge while some may converge.
    \item Full $\beta$-reduction lacks for determinacy. 
  \end{itemize}
  Question:
  \begin{itemize}
    \item Does every path give the same evaluation?
  \end{itemize}
\end{frame}

\begin{frame}{Confluence}
\begin{theorem}[Church-Rosser]
  Given $N_1$ and $N_2$ with $M \reduce N_1$ and $M\reduce N_2$, there is $L$
  such that
  \[
    \xymatrix{
      & M \ar[rd]^{\beta*} \ar[ld]_{\beta*} \\
      N_1 \ar@{-->}[rd]_{\beta*} & & N_2 \ar@{-->}[ld]^{\beta*} \\
      & L
    }
  \]
\end{theorem}
\end{frame}

\begin{frame}{Normal form}
  $M$ is in \emph{normal form} if $M \not\onereduce$. 

  \begin{lemma}
    Suppose that $M$ is in normal form. Then 
    \[
      M \onereduce N
      \implies M =_\alpha N
    \]
    
  \end{lemma}

  \begin{corollary}[Uniqueness of normal forms]\label{coro:uniqueness-normal}
    Suppose that $N_1$ and $N_2$ are in normal form. Then, 
    \[
      \xymatrix{
        & M \ar[rd]^{\beta*} \ar[ld]_{\beta*} \\
        N_1 \ar@{}[r]|{=_{\alpha}} & N & N_2 \ar@{}[l]|{=_\alpha} \\
      }
    \]
  \end{corollary}
\end{frame}

\begin{frame}{Computationally equal terms have a confluent term}

  \begin{corollary}\label{coro:computational-meanting-confluence}
    If $M =_\beta N$, then there exists $L$ satisfying
  \[
    \xymatrix{
      M \ar@{-->}[rd]_{\beta*} \ar@{}[rr]|{=_\beta} & & N \ar@{-->}[ld]^{\beta*} \\
      & L
    }
  \]
  \end{corollary}
  \begin{proof}[Proof sketch]
    By induction on the derivation of $M =_\beta N$. 

    For example, if $M \onereduce N$, then choose $L$ as $N$. 
  \end{proof}

%\begin{corollary}
%  Let $=_\beta$ denote the congruence closure of $\onereduce$. 
%  \begin{enumerate}
%    \item If $M =_\beta N$, then there exists $L$ such that 
%      \[
%        M \reduce L
%        \; _{\beta*}\longleftarrow N
%      \]
%    \item If in addition $N$ is in normal form, then 
%      $M \reduce N$. 
%  \end{enumerate}
%\end{corollary}
%\begin{block}{Homework}
%  Show this corollary.
%\end{block}
\end{frame}

\begin{frame}[allowframebreaks]{Evaluation Strategies}
An evaluation strategy is a procedure of selecting $\beta$-redexes
to reduce. It is a subset $\longrightarrow_{\mathrm{ev}}$ of the full
$\beta$-reduction $\onereduce$.

\begin{description}
  \item[Innermost $\beta$-redex] does not contain any $\beta$-redex.
  \item[Outermost $\beta$-redex] is not contained in any other $\beta$-redex.
\end{description}

\begin{description}
  \item[the leftmost-outermost strategy] reduces the leftmost outermost
    $\beta$-redex in a $\lambda$-term first. For example, 
    \begin{align*}
      & 
      \underline{(\lambda x.\, (\lambda y.\, y)\,x)}\quad
      \underline{(\lambda x.\, (\lambda y.\, y\,y)\,x)}
      \\
      \onereduce &
      \underline{(\lambda y.\, y)}\quad
      \underline{(\lambda x.\, (\lambda y.\, y\,y)\, x)} \\
      \onereduce &
      \lambda x.\, \underline{(\lambda y.\, y\,y)}\quad
      \underline{\vphantom{(\lambda}x} \\
      \onereduce & (\lambda x.\, x\,x) \\
      \not\onereduce
    \end{align*}
  \item[the leftmost-innermost strategy] reduces the leftmost innermost
    $\beta$-redex in a $\lambda$-term first. For example, 
    \begin{align*}
      & (\lambda x.\, \underline{(\lambda y.\,
        y)}\;\;\underline{\vphantom{(\lambda} x})\,
      (\lambda x.\, (\lambda y.\, y\, y)\,x) \\
      \onereduce & (\lambda x.\,x)\;
      (\lambda x.\, \underline{(\lambda y.\, y\,
        y)}\;\;\underline{\vphantom{(\lambda}x}) \\
      \onereduce & \underline{(\lambda x.\,x)}\;\;\underline{(\lambda x.\, x\,x)} \\
      \onereduce & (\lambda x.\, x\, x) \\
      \not\onereduce
    \end{align*}
  \item[the rightmost-innermost/outermost strategy]
    are defined similarly where $\lambda$-terms are reduced from right to left
    instead.
\end{description}
\end{frame}

\begin{frame}{CBV versus CBN}
\begin{description}
  \item[Call-by-value strategy]
    rightmost-outermost but not under any $\lambda$-abstraction
  \item[Call-by-name strategy]
    leftmost-outermost but not under any $\lambda$-abstraction
\end{description}

  \mode<presentation>{\vfill}
\begin{proposition}[Determinacy]
  Each of evaluation strategies is deterministic, i.e.\ 
  if $M \onereduce N_1$ and $M \onereduce N_2$ then $N_1 = N_2$.
\end{proposition}
  \mode<presentation>{\vfill}
We will see the modern approach to evaluation strategies later.
\end{frame}

\begin{frame}{Exercise}
  Define following terms
\begin{align*}
  \Omega &\defeq (\lambda x.\, x\,x)\;(\lambda x.\,x\,x) \\
  \mathbf{K}_1 & \defeq \lambda x\,y.\;x
\end{align*}
Evaluate 
\[
  \mathbf{K}_1\;z\;\Omega
\]
using the call-by-value and the call-by-name strategy respectively.
\end{frame}

\begin{frame}[allowframebreaks]{Normalising}
\begin{definition}
  \begin{enumerate}
    \item $M$ is in \emph{normal form} if $M \not\onereduce N$ for any $N$. 
    \item $M$ is \emph{weakly normalising} if $M \reduce N$ for some $N$ in
      normal form.
  \end{enumerate}
\end{definition}
%
  \begin{enumerate}
    \item $\Omega$ does not have a normal form.
    \item $\mathbf{K}_1$ is normal and thus weakly normalising.
    \item $\mathbf{K}_1\;z\; \Omega$ is weakly normalising.
  \end{enumerate}

\begin{definition}
  A \alert{value} is just a $\lambda$-abstraction (for untyped $\lambda$-calculus). 
\end{definition}

\begin{theorem}%[See {\cite[Theorem 13.2.2]{Barendregt1984a}}]
  The leftmost-outermost strategy reduces every weakly
  normalising $\lambda$-term~$M$ to its value.
\end{theorem}

\begin{proposition}
  Under call-by-name and call-by-value strategies, 
  every value is in normal form.
\end{proposition}
\end{frame}

\begin{frame}{Homework}
  \begin{enumerate}
    \item Prove Corollary~\ref{coro:uniqueness-normal}
    \item Prove Corollary~\ref{coro:computational-meanting-confluence}.
    \item Evaluate $1!$ in $\lambda$-calculus
      using Church numerals and the factorial $n!$ you defined during the exercise session.
  \end{enumerate}
\end{frame}
%\begin{frame}{Remark}
% The definition of capture-avoiding substation is widely adopted, it is still
% ill-defined.  Recursion is always a total function. Advanced
% mathematics~\cite{Pitts2013} is needed to resolve this issue.
%
%  \mode<presentation>{\vfill}
% Issues with named variables may be lifted by using nameless representation of
% terms. For example, in de Bruijin's representation every $\lambda$-term 
% has a canonical form, see \cite[Chapter 6]{Pierce2002}.
%
%  \mode<presentation>{\vfill}
%  In fact, every computable function on natural numbers is definable in terms of
%  $\lambda$-terms. For interested readers, see~\cite[Chapter 3]{Barendregt1984}
%  for further detail. Therefore, $\lambda$-calculus is Turing-complete. 
%\end{frame}


%\begin{frame}[allowframebreaks]{References}
%  \bibliographystyle{amsalpha}
%  \bibliography{library} 
%\end{frame}

\end{document}
