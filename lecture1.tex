%!TEX TS-program = xelatex
%! TEX root = lecture1_slide.tex
\title{Lambda Calculus and Types}
\subtitle{Untyped Lambda Calculus}
\begin{document}
\begin{frame}
\maketitle
\end{frame}

\begin{frame}{Introduction}
  $\lambda$-calculus \dots
  \begin{enumerate}
    \item was developed by Alonzo Church (Turing's PhD supervisor)
    \item is a model of computation
    \item is a backbone of programming languages
  \end{enumerate}
  \mode<presentation>{\vfill}
  As a programming language, it only supports $3$ constructs 
  \begin{enumerate}
    \item variable
    \item function definition
    \item function application
  \end{enumerate}
  \mode<presentation>{\vfill}
\end{frame}

\section{Syntax of Lambda Calculus}

\begin{frame}{Terms of $\lambda$-calculus}
\begin{definition}[Syntax of $\lambda$-calculus]
  Let $V \defeq \{ x, y, z, \ldots \}$ be a countably infinite set of
  \emph{variables}. The set~$\Lambda$ of \alert{$\lambda$-terms} is defined
  by
%  \begin{multicols}{2}
    \begin{prooftree}
      \AXC{$x \in V$}
      \RightLabel{(var)}
      \UIC{$x \in \Lambda$}
    \end{prooftree}
%    \columnbreak%
    \begin{prooftree}
      \AXC{$M \in \Lambda$}
      \AXC{$N \in \Lambda$}
      \RightLabel{(app)}
      \BIC{$(M\, N) \in \Lambda$}
    \end{prooftree}
%    \columnbreak%
    \begin{prooftree}
      \AXC{$M \in \Lambda$}
      \AXC{$x \in V$}
      \RightLabel{(abs)}
      \BIC{$\lambda x.\, M \in \Lambda$}
    \end{prooftree}
%  \end{multicols}
\noindent Application is left associative; abstraction is right associative. 
\end{definition}
\end{frame}

\begin{frame}{Meta-Language and Object-Language}
  \begin{itemize}
    \item \emph{Meta-language} is the language we use to describe the object of
      study. E.g., naive set theory. 
    \item \emph{Meta-variable} is a placeholder \emph{in} the meta-language. 
    \item \emph{Object-language} is the object of study. E.g., arithmetic
      expressions, $\lambda$-terms, etc.
    \item \emph{Variable} refers to some variable of $\lambda$-calculus.
  \end{itemize}
  \begin{example}
  As \emph{naming} a function is not supported in $\lambda$-calculus, we do so
  in the meta-language:
    \[
      \mathtt{id} \defeq \lambda x.\, x
    \]
    $\mathtt{id}$ is a synonym of the $\lambda$-term on RHS in the meta-language.
  \end{example}
 
\end{frame}

\begin{frame}
  \begin{example}[Projections]
  The first and the second projections are made of abstractions and
  variables only:
  \[
    \mathtt{fst} \defeq \lambda x.\, \lambda y.\, x
    \quad\text{and}\quad \mathtt{snd} \defeq \lambda x.\, \lambda y.\, y
  \]
  \end{example}
  For brevity $\lambda x_1\,x_2\,\ldots x_n.\, M \defeq \lambda x_1.\,(\lambda
  x_2.\,(\ldots (\lambda x_n.\, M)\ldots))$.\\
  \mode<presentation>{\vfill}
  Hence, projections are equal to
  \[
    \lambda x\,y.\, x \quad\text{and}\quad 
    \lambda x\,y.\, y
  \]
  \mode<presentation>{\vfill}
  In Haskell: $\mathtt{\backslash x\,y \to \,x}$ and
  $\mathtt{\backslash x\,y \to \,y}$. However, $\mathtt{fst} =
  \mathtt{\backslash x\,y \to \,x}$ is a proper term in Haskell
  (object-language).
\end{frame}

\begin{frame}{$\alpha$-equivalence, informally}
  \begin{definition}
    Two $\lambda$-terms are \alert{$\alpha$-equivalent} if variables
    \emph{bound} by abstractions can be renamed to derive the same term. 
  \end{definition}
  \begin{example}
    \begin{enumerate}
      \item $\lambda x.\, x \neq \lambda y.\, y$  but $\lambda x.\, x
        \equiv_\alpha \lambda y.\, y$. 
      \item $\lambda x.\, \lambda y.\, y \equiv_\alpha 
        \lambda z.\, \lambda y.\, y$. 
      \item $\lambda x.\, \lambda y.\, x \mathrel{\not\equiv}_\alpha
        \lambda x.\, \lambda y.\, y$. 
    \end{enumerate}
  \end{example}
  $\alpha$-equivalent terms are considered `programs of the same structure'.
  Renaming variables do not change program behaviour but
  readability.
\end{frame}

\begin{frame}{Concrete and Abstract Syntax}
  \begin{block}{Concrete Syntax}
    A string possibly annotated with brackets and other delimiters.
  \end{block}
  
  \begin{block}{Abstract Binding Tree}
    A tree structure with pointers where each node is an operator 
    with arguments as its sub-trees.
  \end{block}

  \[
    (\lambda x\, y.\, (\lambda z.\, z\, x)\, y)
  \]
\end{frame}

\section{Operational Semantics}
\begin{frame}{Evaluation, informally}
  \begin{itemize}
    \item The term $(M\;N)$ is understood as a function application where
      $N$ is the argument for the `function' $M$.
    \item The term $\lambda x.\; L$ is understood as a function with a parameter
      $x$ and the function body~$L$. 
    \item The only \alert{`computation'} in $\lambda$-calculus
      is function application:
    \[
      (\lambda x.\, M)\,N \longrightarrow M[x \mapsto N]
    \]
    where $M[x \mapsto N]$ means that $x$ in $M$ is substituted for $N$. 
  \end{itemize}
  How to evaluate the following terms? 
  \begin{enumerate}
    \item $(\lambda x.\,\lambda y.\,x)\,M\,N$
    \item $(\lambda y.\,\lambda y.\,y)\,M$
    \item $(\lambda x.\, \lambda y.\,x)\,y$
  \end{enumerate}
%\[
%  \lambda x_1\,x_2\,\dots x_n.\; M
%  \defeq \lambda x_1.\,(\lambda x_2.\, (\dots (\lambda x_n.\; M)\dots))
%\]
%and
%\[
%  M_0\;M_1\;\dots M_n \defeq (\dots ((M_0\;M_1)\;M_2)\;\dots M_n)
%\]
\end{frame}
%
\begin{frame}[allowframebreaks]{Naive Substitution}
  \begin{definition}
    For $x \in V$ and $M \in \Lambda$, the substitution of $x$ for $M$ is
    defined by
    \begin{align*}
      x[x \mapsto M] & = M \\
      y[x \mapsto M] & = M & \text{if $x \neq y$} \\
      (M\,N)[x \mapsto L] & = (M[x \mapsto L]\,N[x \mapsto L]) \\
      (\lambda x.\, M)[y \mapsto N] & = \lambda x.\, M[y \mapsto N] 
    \end{align*}
  \end{definition}
    A bound variable may become free. 
    \[
      (\lambda x.\, x)[x \mapsto y] = \lambda x.\, y
    \]
  \begin{definition}
    For $x \in V$ and $M \in \Lambda$, the substitution
    of $x$ for $M$ is defined by
    \begin{align*}
      x[x \mapsto M] & = M \\
      y[x \mapsto M] & = M && \text{if $x \neq y$} \\
      (M\,N)[x \mapsto L] & = (M[x \mapsto L]\,N[x \mapsto L]) \\
      (\lambda x.\, M)[y \mapsto N] & = \lambda x.\, M[y \mapsto N] 
                                    && \text{if $x \neq y$} \\
      (\lambda x.\, M)[y \mapsto N] & = \lambda x.\, M
                                    && \text{if $x = y$} 
    \end{align*}
  \end{definition}
    A variable may be captured by an abstraction. 
    \[
      (\lambda x. y)[y \mapsto x] = \lambda x.\, x
    \]
\end{frame}

\begin{frame}{Free and Bound Variables}
\begin{definition}
  The set $\FV$ of free variables of a term $M$ is defined by
  \begin{align*}
    \FV(x) & = \{x\} \\
    \FV(\lambda x.\, M) & = \FV(M) - \{x\} \\
    \FV(M\;N) & = \FV(M) \cup \FV(N)
  \end{align*}
  A variable $y$ which occurs in $M$ is \alert{free} if $y \in \FV(M)$;
  \alert{bound} otherwise. 
  A $\lambda$-term~$M$ is \alert{\emph{closed}} or a \alert{\emph{combinator}}
  if $\FV(M) = \emptyset$. 
\end{definition}
  \begin{enumerate}
    \item A \alert{variable} can be \emph{free} and \emph{bound} at the same
      time. 
    \item An \alert{occurrence} of a variable can only be
      either \emph{free} or \emph{bound}. 
  \end{enumerate}
\end{frame}
\begin{frame}
  \begin{center}
    \includegraphics[height=7cm]{img/e0a.jpg}
  \end{center}
  \begin{enumerate}
    \item A \alert{variable} can be \emph{free} and \emph{bound} at the same
      time. 

    \item An \alert{occurrence} of a variable is either \emph{free} or
      \emph{bound}. 
  \end{enumerate}

\end{frame}
\begin{frame}{Capture-Avoiding Substitution}
  Capture-avoiding substitution of $L$ for the \alert{free occurrences} of $x$ is a
  \emph{partial} function from $\lambda$-terms to $\lambda$-terms defined by
  \begin{align*}
    x\subst{x}{L} & = L \\
    y\subst{x}{L} & = y && \text{if $x \neq y$} \\
    (M\, N)\subst{x}{L} & = (M\subst{x}{L}\; N\subst{x}{L}) \\
    (\lambda x.\, M)\subst{x}{L} & = \lambda x.\, M \\
    (\lambda y.\, M)\subst{x}{L} & = \lambda y.\, M\subst x L
                                 &&\text{if $x \neq y$ and $y \not\in \FV(L)$}
  \end{align*}
  \begin{definition}[Freshness]
    A variable $y$ is \alert{fresh} for $L$ if $y \notin \FV(L)$.  
  \end{definition}
\end{frame}
\begin{frame}{Congruence}
\begin{definition}
  \begin{enumerate}
    \item A \alert{congruence} on $\lambda$-terms is a 
      relation~$R$ on $\lambda$-terms subject to following rules
      \begin{multicols}{2}
        \begin{prooftree}
          \AXC{$M_1 \mathrel{R} M_2$}
          \AXC{$N_1 \mathrel{R} N_2$}
          \BIC{$(M_1\,N_1) \mathrel{R} (M_2\,N_2)$}
        \end{prooftree}
        \begin{prooftree}
          \AXC{$M_1 \mathrel{R}M_2$}
          \UIC{$(\lambda x.\, M_1) \mathrel{R}(\lambda x.\, M_2)$}
        \end{prooftree}
      \end{multicols}
    \item The \alert{congruence closure}~$\overline{R}$ of a relation~$R$ is
      the smallest congruence containing~$R$. 
    \item The \alert{congruence and equivalence closure} of a relation $R$ 
      is the smallest congruence and equivalence relation
      containing~$R$. 
  \end{enumerate}
\end{definition}
\vfill
  \begin{block}{Homework}
    Define congruence closure using inference rules. 
  \end{block}
\end{frame}

\begin{frame}{Renaming of Bound Variables}
If a variable $y$ is \emph{fresh} for $M$, the bound variable $x$ of~$\lambda
x.\, M$ to~$y$ can be renamed without changing the meaning. 
\begin{definition}[$\alpha$-conversion]
  $\alpha$-conversion is a relation $\to_\alpha$ defined by
  \[
    (\lambda x.\, M) \longrightarrow_\alpha \lambda y.\, M\subst{x}{y}
    \quad\text{if $y$ is fresh for $M$}.
  \]
\end{definition}
Let $=_\alpha$ be the congruence and equivalence closure
of~$\longrightarrow_\alpha$. We say that $M$ and $N$ are
\emph{$\alpha$-equivalent} if $M =_\alpha N$.

  \mode<presentation>{\vfill}
\begin{block}{Convention}
  $\lambda$-terms are equal up to $\alpha$-equivalence/renaming of bound variables. 
\end{block}
\end{frame}

\begin{frame}{$\eta$-conversion}
  Pointy style is \emph{assumed} to be equivalent to point-free style. 
  \begin{description}
    \item[$\eta$-reduction]
      $\lambda x.\,(M\,x) \longrightarrow_\eta M$
    \item[$\eta$-expansion]
      $M \longrightarrow_\eta \lambda x.\,(M\,x) $
      where $x$ is fresh for $M$.
    \item[$\eta$-equivalence]
      the congruence and equivalence closure of $\longrightarrow_\eta$. 
  \end{description}
  \mode<presentation>{\vfill}
  $\eta$-equivalence is a form of \emph{extensionality} limited to
  $\lambda$-terms.
  \[
    f = g \iff \forall x.\, f(x) = g(x)
  \]
\end{frame}

\begin{frame}[allowframebreaks]{Evaluation}
\begin{definition}[$\beta$-conversion]
  $\beta$-conversion is a relation defined by
  \[
    (\lambda x.\, M)\, N \longrightarrow_{\beta} M\subst{x}{N}
  \]
  Any term of this form $(\lambda x.\, M)\, N$ is called a \emph{$\beta$-redex}.
\end{definition}
  Good: 
    \[
      ((\lambda x.\,\lambda y.\,x)\,M)
      \longrightarrow_{\beta}
      (\lambda y.\, x)\subst{x}{M}
      = \lambda y.\, x\subst{x}{M}
      = \lambda y.\, M
    \]
  Bad: 
  \[
    (\lambda x.\, \lambda y.\, x)\, M\,N
    \longrightarrow_{\beta} \, ?
  \]
\framebreak
\begin{definition}
  The \alert{full $\beta$-reduction} is a relation on $\lambda$-terms
  defined by
  \begin{multicols}{2}
    \begin{prooftree}
      \AXC{$M_1 \longrightarrow_{\beta} M_2$}
      \UIC{$M_1 \onereduce M_2$}
    \end{prooftree}
    \begin{prooftree}
      \AXC{$M_1 \onereduce M_2$}
      \UIC{$\lambda x.\, M_1 \onereduce \lambda x.\, M_2$}
    \end{prooftree}
    \begin{prooftree}
      \AXC{$M_1 \onereduce M_2$}
      \UIC{$M_1\,N \onereduce M_2\,N$}
    \end{prooftree}
    \begin{prooftree}
      \AXC{$N_1\onereduce  N_2$}
      \UIC{$M\,N_1 \onereduce M\,N_2$}
    \end{prooftree}
  \end{multicols}
\end{definition}
  Now fixed:
  \[
    (\lambda x.\, \lambda y.\, x)\, M\,N
    \onereduce (\lambda y.\, M)\,N
    \onereduce M\subst{y}{N} \onereduce \ldots
  \]
\end{frame}

\section{Programming in Lambda Calculus}
\begin{frame}[allowframebreaks]{Programming in $\lambda$-calculus}
  
Boolean values and conditional can be encoded as closed $\lambda$-terms. 
  
\begin{description}
  \item[Boolean]
    \begin{align*}
      & \mathtt{True}  && \defeq && \lambda x\,y.\, x \\
      & \mathtt{False} && \defeq && \lambda x\,y.\, y
    \end{align*}

  \item[Conditional]
    \begin{align*}
      \mathtt{if}\_\mathtt{then}\_\mathtt{else}\_
      & \defeq \lambda b\;x\;y.\;b\,x\,y  \\
      \mathtt{if} \quad\mathtt{True} \quad & \mathtt{then}\quad M\quad\mathtt{else}\quad
      N  \reduce M \\
      \mathtt{if} \quad\mathtt{False} \quad &\mathtt{then}\quad M\quad\mathtt{else}\quad
      N \reduce N
    \end{align*}
    for any two $\lambda$-terms $M$ and~$N$.
\end{description}

\framebreak
  Natural numbers can be encoded as $\lambda$-terms, so can arithmetic operations. 
  \begin{description}
    \item[Church numerals] 
      \begin{align*}
        &\bc_0 &&\defeq\;&& \lambda f\,x.\, x \\
        &\bc_{1} &&\defeq\;&& \lambda f\,x.\, f\,x \\
        &\bc_{n+1} &&\defeq\;&& \lambda f\,x.\, f^{n+1}\,(x)
      \end{align*}
      for $n > 0$ where $f^{n+1}(M)  \defeq f (f^n\, M))$.
    \item[Successor]
      \begin{align*}
        & \mathtt{succ} && \defeq\; && \lambda n.\,\lambda f\,x.\, f\,(n\,f\,x) \\
        & \mathtt{succ}\;\bc_n && \reduce\; && \bc_{n+1}
      \end{align*}
      for any natural number~$n \in \mathbb{N}$.
    \item[Predecessor]
      \begin{align*}
        & \mathtt{pred} && \defeq\; && \lambda n.\, \lambda f\,x.\, ? \\
        & \mathtt{pred}\;\bc_{0}  && \reduce\; && \bc_{0} \\
        & \mathtt{pred}  \;\bc_{n+1} && \reduce\; && \bc_{n}
      \end{align*}
    \item[Addition]
      \begin{align*}
        & \mathtt{add} && \defeq\; && \lambda n\,m.\,\lambda f\,x.\;
        m\;f\;(n\;f\;x)  \\ & \mathtt{add}\;\bc_{n}\;\bc_{m}\;
                            && \reduce\; && \bc_{n + m}
      \end{align*}
    \item[Conditional]
      \begin{align*}
        & \mathtt{ifz} && \defeq \; \lambda n\;x\;y.\;n\;(\lambda z.\;y)\;x 
        \\
        & \mathtt{ifz}\;\bc_0\;M\;N && \reduce \; M \\
        & \mathtt{ifz}\;\bc_{n+1}\;M\;N && \reduce \; N
      \end{align*}
  \end{description}
  \framebreak

  Here is a list of common combinators:
\begin{enumerate}
  \item $\omega \defeq \lambda x.\, x\,x$
  \item $\Omega \defeq (\lambda x.\, x\,x)\,(\lambda x.\,x\,x)
    = \omega\,\omega$
  \item $\mathbf{I} \defeq \lambda x.\, x$, the \emph{identity}.
  \item $\mathbf{S}\defeq\lambda f\,g\,x.\, 
    (f\,x)\,(g\,x)$.
\end{enumerate}
  \begin{block}{Exercise}
    \begin{enumerate}
      \item Evaluate $\mathtt{succ}\,\bc_0$ and $\mathtt{add}\,\bc_{1}\,\bc_{2}$. 
      \item Define $\mathtt{pred}$. 
      \item Define Boolean operations, i.e.\ $\mathtt{not}$, $\mathtt{and}$,
        and $\mathtt{or}$. 
      \item Is $\omega$ allowed in Haskell? 
    \end{enumerate}
  \end{block}
\end{frame}

\begin{frame}[allowframebreaks]{General Recursion}

  The summation $\sum_{i = 0}^{n} i$ for $n \in \mathbb{N}$ \emph{can
  be} defined as
\[
  \mathtt{sum}(n) =
    \begin{cases} 
     0 & \text{if } n = 0 \\
     n + \mathtt{sum}(n - 1)  & \text{otherwise}.
    \end{cases}
\]
Can we avoid the self-reference? Consider the function
$G\colon (\mathbb{N} \to \mathbb{N}) \to (\mathbb{N} \to \mathbb{N})$
defined by
\begin{equation} \label{eq:summation}
  (Gf)(n) \defeq
  \begin{cases}
     0 & \text{if } n = 0 \\
     n + f(n - 1)  & \text{otherwise}.
  \end{cases}
\end{equation}
Assuming that $\mathtt{sum}'$ is a fixed-point of $G$, i.e.\
$G(\mathtt{sum}') =\mathtt{sum}'$, we can show that $\mathtt{sum}' =
\mathtt{sum}$ by induction. 

\framebreak
\begin{proposition}[Curry's paradoxical combinator]
  Define
  \[
    \mathbf{Y} \defeq \lambda f.\, (\lambda x.\, f\,(x\,x))\,(\lambda
    x.\,f\,(x\,x)).
  \]
  Then, 
  \begin{align*}
    \mathbf{Y}F
    & \onereduce \underline{(\lambda x.\,F\,(x\,x))\,(\lambda x.\, F\,(x\,x))} \\
    & \onereduce F\,(\underline{(\lambda x.\,F\,(x\,x))\,(\lambda x.\,
      F\,(x\,x))})
  \end{align*}
  for every $\lambda$-term~$F$. 
\end{proposition}

\framebreak
\begin{example}[Summation, formally]
  Using the combinators we have known so far, the equation~\eqref{eq:summation}
  can be defined as $\lambda$-terms:
  \begin{align*}
    G \defeq{} & 
    \lambda f\,n.\, \mathtt{ifz}\;n\;bc_0\;
    (\add\;n\;(f\;(\mathtt{pred}\;n))) \\
    \mathtt{sum} \defeq{} & \mathbf{Y}G
  \end{align*}
\end{example}
  Try to evaluate $\mathtt{sum}$ with, say, $\bc_3$.

\framebreak
Here is a fixed-point operator such that $\Theta F \reduce F(\Theta F)$.
\begin{proposition}[Turing's fixed-point combinator]
  Define 
  \[
    \Theta \defeq 
    (\lambda x\,f.\,f\,(x\,x\,f))\;
    (\lambda x\,f.\,f\,(x\,x\,f))
  \]
  Then, 
  \[
    \Theta F \reduce F (\Theta F)
  \]
\end{proposition}
Try Turing's fixed-point combinator with $G$ to define $\sum_{i=0}^n i$.
\begin{align*}
  G \defeq{} &
  \lambda f\,n.\,
  \mathtt{ifz}\;n\;bc_0\;
  (\add\;n\;(f\,(\mathtt{pred}\;n))) \\
  \mathtt{sum} \defeq\; & \Theta\,G 
\end{align*}

\begin{block}{Exercise}
\begin{enumerate}
  \item Define the \emph{flip} operation, i.e.\ a $\lambda$-term~$\mathtt{flip}$
    such that
    \[
      \mathtt{flip}\;M\;N\;P
      \reduce M\;P\;N
    \]
  \item Define the multiplication $m \times n$ on Church numerals.
  \item Define the factorial $n!$ on Church numerals.
\end{enumerate}
\end{block}

\end{frame}
\section{Properties of Lambda Calculus}
\begin{frame}[allowframebreaks]{Church-Rosser Property}
  \begin{example}
    Suppose $M \in \Lambda$ and $y \not\in \FV(M)$. 
    Then, consider 
    \[
      (\lambda y.\, M)\, ((\lambda x.\, x\,x)(\lambda x.\, x\, x))
    \]
  \end{example}
  \mode<presentation>{\vfill}
  Observations:
  \begin{itemize}
    \item Some evaluation may diverge while some may converge.
    \item Full $\beta$-reduction lacks for determinacy. 
  \end{itemize}
  Question:
  \begin{itemize}
    \item Does every path give the same evaluation?
  \end{itemize}

  \framebreak

  Let $\reduce$ denote the reflexive and transitive closure of $\onereduce$.
%  \begin{multicols}{2}
%    \begin{prooftree}
%      \AXC{\phantom{$M \reduce N$}}
%      \UIC{$M\reduce N$}
%    \end{prooftree}
%    \begin{prooftree}
%      \AXC{$M \onereduce N$}
%      \AXC{$N \reduce L$}
%      \BIC{$M\reduce L$}
%    \end{prooftree}
%  \end{multicols}
\begin{theorem}[Church-Rosser Property]
  Given $N_1$ and $N_2$ with $M \reduce N_1$ and $M\reduce N_2$, there is $L$
  such that
  \[
    \xymatrix{
      & M \ar[rd]^{\beta*} \ar[ld]_{\beta*} \\
      N_1 \ar@{-->}[rd]_{\beta*} & & N_2 \ar@{-->}[ld]^{\beta*} \\
      & L
    }
  \]
\end{theorem}
\framebreak
A term is in normal form if $M \not\onereduce$. 

\begin{corollary}[Uniqueness of normal forms]\label{coro:uniqueness-normal}
  Suppose that $N_1$ and $N_2$ are in normal form. Then, 
  \[
    \xymatrix{
      & M \ar[rd]^{\beta*} \ar[ld]_{\beta*} \\
      N_1 \ar@{}[r]|{=} & N & N_2 \ar@{}[l]|{=} \\
    }
  \]
\end{corollary}

\begin{block}{Homework}
  Show this corollary.
\end{block}

\begin{corollary}
  Let $=_\beta$ denote the congruence closure of $\onereduce$. 
  \begin{enumerate}
    \item If $M =_\beta N$, then there exists $L$ such that 
      \[
        M \reduce L
        \; _{\beta*}\longleftarrow N
      \]
    \item If in addition $N$ is in normal form, then 
      $M \reduce N$. 
  \end{enumerate}
\end{corollary}
\begin{block}{Homework}
  Show this corollary.
\end{block}
\end{frame}

\begin{frame}[allowframebreaks]{Evaluation Strategies}
An evaluation strategy is a procedure of selecting $\beta$-redexes
to reduce. It is a subset $\longrightarrow_{\mathrm{ev}}$ of the full
$\beta$-reduction $\onereduce$.

\begin{description}
  \item[Innermost $\beta$-redex] does not contain any $\beta$-redex.
  \item[Outermost $\beta$-redex] is not contained in any other $\beta$-redex.
\end{description}

\begin{description}
  \item[the leftmost-outermost strategy] reduces the leftmost outermost
    $\beta$-redex in a $\lambda$-term first. For example, 
    \begin{align*}
      & 
      \underline{(\lambda x.\, (\lambda y.\, y)\,x)}\quad
      \underline{(\lambda x.\, (\lambda y.\, y\,y)\,x)}
      \\
      \onereduce &
      \underline{(\lambda y.\, y)}\quad
      \underline{(\lambda x.\, (\lambda y.\, y\,y)\, x)} \\
      \onereduce &
      \lambda x.\, \underline{(\lambda y.\, y\,y)}\quad
      \underline{\vphantom{(\lambda}x} \\
      \onereduce & (\lambda x.\, x\,x) \\
      \not\onereduce
    \end{align*}
  \item[the leftmost-innermost strategy] reduces the leftmost innermost
    $\beta$-redex in a $\lambda$-term first. For example, 
    \begin{align*}
      & (\lambda x.\, \underline{(\lambda y.\,
        y)}\;\;\underline{\vphantom{(\lambda} x})\,
      (\lambda x.\, (\lambda y.\, y\, y)\,x) \\
      \onereduce & (\lambda x.\,x)\;
      (\lambda x.\, \underline{(\lambda y.\, y\,
        y)}\;\;\underline{\vphantom{(\lambda}x}) \\
      \onereduce & \underline{(\lambda x.\,x)}\;\;\underline{(\lambda x.\, x\,x)} \\
      \onereduce & (\lambda x.\, x\, x) \\
      \not\onereduce
    \end{align*}
  \item[the rightmost-innermost/outermost strategy]
    are defined similarly where $\lambda$-terms are reduced from right to left
    instead.
\end{description}
\begin{description}
  \item[Call-by-value strategy]
    rightmost-outermost but not inside any $\lambda$-abstraction
  \item[Call-by-name strategy]
    leftmost-outermost but not inside any $\lambda$-abstraction
\end{description}

\begin{proposition}[Determinacy]
  Each of evaluation strategies is deterministic. 
\end{proposition}
\begin{block}{Exercise}\label{ex:omega}
  Evaluate $\Omega$ and $\mathbf{K}_1\, (\lambda x.\, x)\, \Omega$ respectively 
  using call-by-value and call-by-name strategy
  where 
  \begin{align*}
    \Omega &\defeq (\lambda x.\, x\,x)(\lambda x.\,x\,x) \\
    \mathbf{K}_1 & \defeq \lambda x\,y.\;x
  \end{align*}
\end{block}
\begin{block}{Homework}
  Draw and evaluate above $\lambda$-terms in abstract syntax tree
  with the rightmost-innermost/outermost strategies.
\end{block}
\end{frame}
\begin{frame}[allowframebreaks]{Normalising}
\begin{definition}
  \begin{enumerate}
    \item $M$ is in \emph{normal form} if $M \not\onereduce N$ for any $N$. 
    \item $M$ is \emph{weakly normalising} if $M \reduce N$ for some $N$ in
      normal form.
  \end{enumerate}
\end{definition}
%
  \begin{enumerate}
    \item $\Omega$ does not have a normal form.
    \item $\mathbf{K}_1$ is normal and thus weakly normalising.
    \item $(\mathbf{K}_1 z)\,\Omega$ is weakly normalising.
  \end{enumerate}

\begin{theorem}%[See {\cite[Theorem 13.2.2]{Barendregt1984a}}]
  The leftmost-outermost strategy reduces every weakly
  normalising $\lambda$-term~$M$ to its normal form $N$.
\end{theorem}

\begin{definition}
  For $\lambda$-calculus, a \alert{value} is just a $\lambda$-abstraction. 
\end{definition}
\begin{proposition}
  Under call-by-name and call-by-value strategies, 
  every value is in normal form.
\end{proposition}
\end{frame}


\begin{frame}{Remark}
 The definition of capture-avoiding substation is widely adopted, it is still
 ill-defined.  Recursion is always a total function. Advanced
 mathematics~\cite{Pitts2013} is needed to resolve this issue.

  \mode<presentation>{\vfill}
 Issues with named variables may be lifted by using nameless representation of
 terms. For example, in de Bruijin's representation every $\lambda$-term 
 has a canonical form, see \cite[Chapter 6]{Pierce2002}.

  \mode<presentation>{\vfill}
  In fact, every computable function on natural numbers is definable in terms of
  $\lambda$-terms. For interested readers, see~\cite[Chapter 3]{Barendregt1984}
  for further detail. Therefore, $\lambda$-calculus is Turing-complete. 
\end{frame}


\begin{frame}[allowframebreaks]{References}
  \bibliographystyle{amsalpha}
  \bibliography{library} 
\end{frame}

\end{document}
